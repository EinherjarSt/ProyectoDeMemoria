\documentclass[11pt,letterpaper]{article}
\usepackage{pslatex}
\usepackage[spanish]{babel}
\usepackage[utf8]{inputenc} % Caracteres con acentos.
\usepackage{latexsym}
\usepackage{amssymb} 
\usepackage{amsmath}
\usepackage{epsfig}
\usepackage{url}

\begin{document}

\pagestyle{empty}

\title{
Herramienta para la optimización de redes de distribución de agua potable\\
(Propuesta de proyecto final de carrera)
}
\author{
Gabriel Sanhueza Fuentes (Estudiante)\\
Jimmy Gutierrez Bahamondes (Profesor guía)\\
Carrera de Ingeniería Civil en Computación\\ 
Universidad de Talca}
\date{6 de septiembre de 2019}

\maketitle


\section{Descripción de la propuesta}
%\emph{(Esta sección debe incluir una presentación general del problema a investigar y/o idea a desarrollar. En esta sección se debe incluir aquellas referencias bibliográficas vinculadas al contexto del proyecto. Para esto último se recomienda el uso de un archivo *.bib, el cual usa el formato BibTex \cite{1} para codificar  referencias sobre libros \cite{2}, artículos en revistas científicas \cite{3}, artículos en conferencias o workshops \cite{4}, reportes técnicos \cite{5}, capítulos en libros \cite{6}, y páginas Web \cite{7}.
%La longitud máxima de esta sección es de 2.5 páginas.)

% Agregar una introducción a lo que viene abajo

En el presente capitulo se presentan los conceptos básicos, el problema, su contexto, y la propuesta de solución. 

\subsection{Conceptos básicos del proyecto} 

En esta sección se definen algunos elementos básicos que serán necesarios para la realización del proyecto.

\begin{itemize}
\item \textbf{Optimización}: La optimización consiste en maximizar o minimizar un conjunto de funciones que matemáticamente pueden ser expresadas de la siguiente forma:
$$f_1(x),f_2(x), ..., f_N(x),\ x=(x_1,...,x_d) | x \in X$$
sujeto a una serie de condiciones
$$h_j(x) = 0, j=1,2,...,J$$
$$g_k(x) \leq 0, k=1,2,...,K$$
siendo $f_1,...,f_N$ funciones objetivos y   $h_j$ junto con $g_k$ una serie de restricciones \cite{Yang2015}. En donde de acuerdo a \cite{Lopez2013},donde $X$ corresponde al conjunto de espacio de búsqueda, mientras que los valores obtenidos $f_n(x)$ representa a los valores alcanzados por cada uno de los objetivos. De acuerdo a la cantidad de funciones objetivos que se tenga, se establece que si $N=1$ la optimización es \textbf{mono-objetivo}, mientras que para $N\geq 2$ es \textbf{multi-objetivo} \cite{Yang2015}. En este punto se debe tener en cuenta que los objetivos planteados deben encontrarse en contradicción. 

Debido a la definición de las restricciones es posible dividir el espacio de búsqueda en dos regiones \cite{Lopez2013}:
\begin{itemize}
	\item Soluciones factibles: Compuesto por los elementos pertenecientes al espacio de búsqueda que satisfacen todas las restricciones.
	\item Soluciones no factibles: Integrado por aquellos elementos que no complen todas las restricciones.
\end{itemize}
%\item \textbf{Optimización Mono-objetivo}: 


\item \textbf{Heurísticas}: Métodos que permiten obtener soluciones de un problema haciendo uso de un bajo nivel de recursos y en un tiempo reducido \cite{Lopez2013}.


\item \textbf{Metaheuristica}: Heuristicas generalizadas para ser aplicadas en un amplio conjunto de problemas. \cite{Lopez2013}.

\item \textbf{Algoritmos genéticos}:Estrategia de búsqueda de soluciones basada en la teoría de la evolución de Darwin. Para realizar esto, el algoritmo  parte desde un conjunto de soluciones denominada población he iterativamente, lleva a cabo un proceso de reproducción, generando nuevas soluciones \cite{Heiss-Czedik1997}.
%\item \textbf{Optimización Multi-objetivo}: Problemas que poseen múltiples funciones objetivos.

\item \textbf{NSGA-II}: Algoritmo que utiliza la cruza, mutación y reproducción para encontrar un conjunto de soluciones optimas a problemas que cuentan con mas de un objetivo \cite{Deb2002}. 

%\item \textbf{JMetal}: Framework que incorpora y permite la utilización de algoritmos metahuristicos para la optimización multiobjetivo. \cite{9}
\item \textbf{Epanet}: Software que permite simular el comportamiento hidráulico y la calidad del agua en redes de distribución de aguas compuesta por tuberías, nodos, bombas, válvulas y tanques de almacenamiento \cite{Rossman2017}. 
\item \textbf{Red de distribución de agua potable}: Conjunto de elementos enlazados de tal manera que permite suministrar cierta cantidad de agua a una presión establecida \cite{Doctoral2012}.
 
\end{itemize}

\subsection{Contexto del proyecto} 
%\emph{Esta sección debe incluir el marco en el cual se presenta el problema y/o proyecto a desarrollar, incluyendo los fundamentos teóricos y/o prácticos necesarios para el desarrollo del proyecto.) Esta sección responde a la pregunta ¿Dónde surge el problema?}

La escasez de agua potable es sin duda una problemática a nivel mundial. Dentro de este contexto, la optimización de los sistemas de distribución de agua potable es un problema sin resolver hasta la fecha. 

A lo mencionado anteriormente, también se suma la escasez energética. Puesto que, es requerida para el tratamiento y distribución del agua. Es por esto que, es importante hacer un uso eficiente de la energía durante la operación del sistema.

La optimización de estos sistemas, a la vez, involucra la participación de múltiples criterios que deben ser tomados en cuenta a la hora de decidir. Sin embargo, la incorporación de estos criterios, involucra la generación de modelos cada vez más complejos.

Los algoritmos metaheurísticos han demostrado ser un mecanismo eficiente ante problemas de este tipo. Ya que, estos reducen el tiempo necesario que toma el evaluar todas las configuraciones posibles. Dado que, juzgan un conjunto menor de valores logrando como resultado una aproximación a la solución óptima.

\subsection{Definición del problema} 
%\emph{(Esta sección debe incluir la descripción del problema resolver o idea a desarrollar, y la motivación para hacerlo. Es decir, cual es la importancia, innovación, aporte, y/o beneficio para la ciencia y/o la humanidad). Esta sección responde a la pregunta ¿Cuál es el problema que voy a resolver?}

Los encargados de implementar sistemas de distribución de agua potable, no cuentan con suficientes herramientas y  tiempo para su correcta gestión. Por lo tanto, no es posible utilizar los recursos asociados de forma eficiente. Además, las herramientas existentes no satisfacen las necesidades de usabilidad y costo, debido a que son poco intuitivas y de pago.

El escoger las especificaciones de una red de agua potable ya es de por sí difícil debido a que hay que evaluar el rendimiento general del sistema alternando entre distintas configuraciones en busca de una solución que sea eficaz. Debido a esto, el uso de herramientas automatizadas que evalúen el rendimiento de las diversas combinaciones posibles viene a ser necesario.

A lo anterior se suma el hecho de que los interesados en esta área no manejan herramientas informáticas.

%Es por esto que con la realización de este trabajo se busca lograr la implementación de  una herramienta que ayude a los encargados a evaluar el diseño y la operación de la red de agua potable utilizando un enfoque especifico, contribuyendo a la ves al uso eficiente del agua y energía.



\subsection{Trabajo relacionado} 
%\emph{(Esta sección debe incluir los enfoques usados actualmente para resolver el problema. Esta sección debe contener referencias bibliográficas a trabajos relacionados al proyecto.) Esta sección responde a la pregunta ¿Qué se ha hecho para resolver el problema?}

En este apartado se mostraran algunos trabajos parecidos al que tiene como resultado este proyecto.

\begin{itemize}
	\item \textbf{Magmoredes}: En \cite{Edwin2017} se describe la existencia de un software de diseño basado en micro-algoritmos genéticos multiobjetivos, que de acuerdo al autor, tiene un mejor rendimiento y es más eficiente debido a que requiere una menor cantidad de memoria y tiene un tiempo de computo mejor que el algoritmo NSGA-II. Este programa puede cargar cualquier red y realiza los cálculos utilizando librerías de java. Las funciones objetivos que este sistema intenta resolver son la optimización de los costo y la confiabilidad final de la red.
	
	
	\item \textbf{Hidra software}: Es un Software de paga que permite el diseño y la realización de distintos tipos de cálculos, como la capacidad de estanque, los recursos requeridos para la construcción de la red, demanda, diámetros, etc.
	
	
	\item \textbf{WaterGEMS}: De acuerdo a \cite{Bentley2017}, este software permite la  construcción de modelos geoespaciales; optimizacion de diseño, ciclos de bombeo y calibración automática del modelo; y la  gestión de activos. 
	
	Los programas presentados anteriormente, resuelven problemas particulares que se dan en una red de distribución de agua potable.
\end{itemize}

\subsection{Propuesta de solución}
%\emph{(Esta sección debe incluir el planteamiento y justificación de la solución y/o idea, incluyendo aspectos novedosos.) Esta sección responde a la pregunta ¿Cómo voy a resolver el problema planteado?}

La solución que se propone para solventar el problema que motiva este trabajo es el desarrollo de una aplicación que sirva como una base sobre la cual poder agregar nuevas funcionalidades. Para esto la aplicación debe quedar bien documentada. Por lo tanto, lo que se tendrá al termino del desarrollo del proyecto será un software escalable que tratara con dos problemas relacionados a la distribución de agua potable y que podrá ser ampliado en futuros trabajos.

Los problemas que se abordaran en el contexto de optimización de redes de agua potable serán:
\begin{itemize}
	\item \textbf{Problema de diseño:} Se trata este problema desde un enfoque monoobjetivo implementando un algoritmo genético que buscara la optimización de los costo de inversión variando el diámetro de las tuberías.  
	\item \textbf{Problema de operación:} Este problema se abordara desde el enfoque multiobjetivo. Para esto se implementara el algoritmo NSGA-II y se buscara la optimización de los objetivos de costos energetivos y el numero de encendidos y apagados de las bombas.
\end{itemize}

Tanto el algoritmo genético como el NSGA-II, permiten la utilización de distintos operadores de cruza y mutación que también serán implementados para ser utilizados. 

La solución planteada supone ademas el diseño e implementación de una interfaz gráfica para ayudar al usuario en el uso de la herramienta.
%\section{Hipótesis}
%\emph{(En esta sección se deben incluir una lista de afirmaciones o suposiciones las cuales se esperar responder con el desarrollo del proyecto. La longitud máxima de esta sección es de 1/2 página.)}
%\begin{itemize}
%\item El uso de ... puede facilitar ....
%\item El problema de .... puede estudiarse como ... 
%\item Las técnicas usadas en ... pueden ser aplicables para resolver el problema de ...
%\end{itemize}

\section{Objetivos}
%\emph{(En esta sección se deben especificar el objetivo general y los objetivos específicos del proyecto. Los objetivos deben reflejar lo que se espera lograr con el proyecto, evitando incluir características específicas de la solución. La longitud máxima de esta sección es de 1 página.)}

A continuación se darán a conocer el objetivo general y los objetivos específicos que se desean lograr con el desarrollo del proyecto.

\paragraph{Objetivo general}
%\emph{(Debe ser una sola frase que resuma lo que se quiere lograr.)} 

\begin{itemize}
%\item Desarrollar un sistema web que permita administrar la asignación de salas de clase de forma automática y eficiente.
\item Diseñar y desarrollar una aplicación de apoyo a la toma de decisiones, integrando algoritmos de optimización aplicados al problema de diseño y operación en redes de distribución de agua potable.
\end{itemize}

\paragraph{Objetivos específicos}% \emph{(Una lista de puntos que detallan el objetivo general.)}
\begin{enumerate}
%\item Especificar los requerimientos del proyecto.
%\item Diseñar un algoritmo para ...
%\item Construir una herramienta para ...
%\item Evaluar el sistema a través de ... 
%\item Especificar los requerimientos del proyecto.
%\item Definir la manera de representar la red de agua potable y los elementos que la componen.
%\item Diseñar e implementar una interfaz que permita la carga y visualización de una red de agua potable.
%\item Integrar el algoritmo multiobjetivo Nondominated Sorting Genetic Algorithm (NSGA-II) para resolver el problema de la programación de bombas.
%\item Integrar distintos operadores de cruza y mutación para utilizar en el algoritmo multiobjetivo.
%\item Integrar Algoritmo Genético para resolver el problema de diseño de la red.
%\item Integrar distintos operadores de cruza y mutación para utilizar en el algoritmo mono-objetivo.
%\item Permitir almacenar los resultados obtenidos al ejecutar los algoritmos.
%\item Evaluar el sistema a través de pruebas de usabilidad.
\item Generar soluciones frente al problema monoobjetivo de diseño de redes de distribución de agua potable a través de la implementación de algoritmos genéticos.
\item Generar soluciones frente al problema multiobjetivo de operación de redes de distribución de agua potable a través de la implementación de NSGA-II.
\item Diseñar he implementar interfaz grafica del sistema. 
\item Evaluar el desempeño de los algoritmos, contrastando los resultados obtenidos en redes de benchmarking con óptimos conocidos.

\end{enumerate}



\section{Alcances}
%\emph{(En esta sección se debe incluir una lista de puntos que definen los límites del trabajo. La longitud máxima de esta sección es de 1/2 página.)}
\begin{itemize}
%\item En este trabajo se espera implementar un prototipo funcional de la idea a desarrollar, por lo tanto se propone desarrollar una interfaz de línea de comando simple en lugar de una interfaz de usuario gráfica. 
%\item En este trabajo no se crearán algoritmos para ... 
%\item Este trabajo se limita a ...
\item Este trabajo no contempla la creación de la red, por lo que estas serán ser ingresadas como entradas.
\item Este trabajo solo contempla la utilización e implementación del algoritmo genético y NSGA-II.
\item El producto obtenido a través de la realización de este trabajo solo sera compatible con el sistema operativo Windows. Esto se debe a que harán llamadas a librerías nativas para realizar los cálculos de las redes.

\end{itemize}



\section{Metodología}
%\emph{(En esta sección se deben describir y justificar los métodos que se usarán en el desarrollo del proyecto de titulación. Los métodos obligatorios que debe incluir esta seccion es: Metodología de Desarrollo/Investigación y Metodología de Evaluación.)}

En esta sección se describirá y justificara la metodologías de desarrollo y evaluación que se usaran durante el desarrollo del proyecto.. 
\subsection{Metodología de desarrollo}

%\emph{(En esta subsección se deben describir y justificar los metodos de desarrollo y/o investigación que se aplicaran a lo largo del desarrollo del proyecto.)}

Esta sección describe y justifica el uso de la metodología iterativo incremental.
\subsubsection{Iterativo incremental}

Esta metodología lleva a cabo el desarrollo de un proyecto de software dividiéndolo en iteraciones que generan un incremento, el cual contribuye en el desarrollo del producto final. Cada iteración se compone de las fases de análisis, diseño, implementación y testing. La fase de análisis lleva a cabo la obtención y definición de los requerimientos del software. La etapa de diseño se encarga de la conceptualización del software basado en los requerimientos definidos anteriormente. Durante la implementación se codifican las funcionalidades siguiendo las directivas establecidas durante el diseño, con el fin de satisfacer los requerimientos. Y finalmente, durante la fase de testing, se valida y verifica la correctitud de las funcionalidades implementadas, asi como el cumplimiento de los requisitos. El hecho de llevar a cabo un desarrollo iterativo permite la obtención de retroalimentación del producto que se esta desarrollando tempranamente y de esta manera poder refinar el trabajo en etapas posteriores del desarrollo. \cite{Victor2003, Mitchell2009, Martin1999,Alshamrani2015}.

Debido a que la metodología esta pensada para ser llevada a cabo por un equipo de trabajo se adaptara la metodología para poder ser aplicada en el desarrollo llevado a cabo por una sola persona. Esta adaptación consiste en la disminución de la cantidad de la documentación generada, permitir llevar a cabo mas de una fase de la iteración al mismo tiempo y los roles de analista, diseñador, implementador y tester sera realizado por una sola persona. Los documentos a generar por cada fase serán:

\paragraph{Análisis:} El producto generado por esta fase sera un documento de especificación de requisitos que constara de:
\begin{itemize}
	\item Introducción
	\item Requisitos de usuario
	\item Requisito de sistema
	\item Matriz de trazado requisitos de usuario vs sistema.
\end{itemize}
\paragraph{Diseño:} Esta fase generara como producto un documento de diseño que contara con los siguientes diagramas:
\begin{itemize}
	\item Casos de uso
	\item Arquitectura lógica
	\item Diagrama de componentes
	\item Diagrama de clases
\end{itemize}
\paragraph{Implementación:} Esta fase generara el código fuente de la aplicación y un manual de usuario de la aplicación.
\paragraph{Pruebas:} Durante esta fase de realizaran la documentación y la realización de las pruebas sobre la aplicación.
\begin{itemize}
	\item Se documentara las pruebas unitarias realizadas y sobre que componentes.
	\item Se documentara las pruebas de integración y que caso de uso cubren.
\end{itemize}

La razón por la que se utilizará esta metodología sobre otras es porque el producto resultante de este proyecto esta pensado para servir como base para futuros trabajos. Debido a esto es necesario documentar correctamente para que otros programadores puedan continuar con su desarrollo más adelante. Aunque existen otras metodologías como cascada u otras tradicionales, estas son difíciles de llevar a cabo por la cantidad de documentación que se requiere, mientras que metodologías de desarrollo ágil carecen en cuanto a la documentación que se necesita para el sistema a desarrollar. Ademas, esta metodología nos permite obtener una retroalimentacion al final de la iteración, obtener nuevos requisitos que no hayan quedado definidos en etapas anteriores o refinar los requisitos y el diseño ya existente, permitiendo así mejorar la calidad del producto final.

La implementación de esta metodología para el desarrollo del proyecto se llevara a cabo repartiendo las tareas necesarias para el cumplimiento de los objetivos en iteraciones. De este modo al final de cada iteración se contará con un prototipo funcional de la aplicación sobre el que se agregaran las nuevas funcionalidades en las iteraciones siguientes.

\subsection{Metodología de evaluación del proyecto}
%\emph{(En esta subsección se deben describir y justificar los metodos de evaluación/validación que se aplicarán a lo largo del desarrollo del proyecto.)}

Esta sección describe las metodologías que se aplicaran para la evaluación del proyecto. Estas serán pruebas unitarias, de integración y caso de estudio.

\subsubsection{Pruebas unitarias}
Pruebas realizadas sobre un componente del programa, ya sean métodos u objetos. Estas pruebas se llevan a cabo variando los parámetros de entrada de los componentes para comprobar que la funcionalidad haya sido correctamente implementada. Ademas, deben ser realizadas aisladamente sin interacción con otros componentes o sistemas.\cite{Sommerville2010}.

El motivo para realizar estas pruebas sobre los componentes del sistema, es para poder detectar defectos y comprobar que el modulo se comporte de la manera esperada.

\subsubsection{Pruebas de integración}
Las pruebas de integración son utilizadas para comprobar las interfaces y las interacciones entre los módulos que conforman la aplicación \cite{Leung1990}. Este tipo de pruebas se realizan después de la realización de las pruebas unitarias y permitirá verificar que  los componentes interactúan  adecuadamente. 

\subsubsection{Caso de estudio}
 %\textbf{(Wholin Empirical software engineering reasearch, Case studies Yin)}
La metodología de evaluación \cite{Yin2009,Shull2008} que se utilizara para la investigación es mediante casos de estudio. Un caso de estudio permite aplicar los resultados de implementación en un contexto real con el objetivo de responder la pregunta de investigación planteada. Esta metodología de evaluación considera aspectos formales para obtener evidencia. Los principales aspectos son:

\begin{itemize}
\item Describir el contexto de aplicación del caso
\item Definición de objetivos experimentales
\item Definición de características a evaluar
\item Definición de sujetos de prueba
\item Definición de un protocolo para conducir el caso de estudio
\item Aplicación de caso de estudio en un conjunto de sesiones no controladas
\item Aplicación de herramientas de obtención de evidencia empírica
\item Análisis y evaluación de datos empíricos
\end{itemize}




%\paragraph{Objetivo 1:} ``Comparación de algoritmos para ..."
%\begin{itemize}
%\item Estudiar ...
%\item Seleccionar ...
%\item Comparar ...
%\end{itemize}
%
%\paragraph{Objetivo 2:} ``Especificación de requisitos del software"
%\begin{itemize}
%\item Analizar ...
%\item Clasificar ...
%\item Especificar ...
%\end{itemize}

\section{Plan de trabajo}
%\emph{(En esta sección se debe definir como organizar y planificar, en términos de etapas y tiempo, las actividades a desarrollar así como los resultados a obtener.  Cada actividad debe generar un entregable. Se entiende como entregable a la generación de artefactos que sean útiles al desarrollo del proyecto, por ejemplo: documentos bibliográficos con resumen, documentos de desarrollo de software, código fuente, planificaciones, diagramas, esquemas, algoritmos, etc.)}

%\emph{(Esta sección debe incluir además una Carta Gantt, la cual define fechas de inicio y término. La longitud máxima de esta sección es de 1 página, sin considerar la Carta Gantt.)}

En esta sección de detallara las tareas a realizar para cumplir con cada uno de los objetivos así como la fecha en que estas serán realizadas. Ademas, junto a este documento se adjuntada una carta Gantt que refleja este plan de trabajo.

\paragraph{Etapa 1:} Desarrollar el objetivo 1 (12/08-07/10)
\begin{itemize}
	\item Capturar y especificación de los requisitos (26/08-09/09)
	\item Construir planificación (12/08-09/09)
	\item Crear la arquitectura lógica (09/09-16/09)
	\item Especificar casos de uso (09/09-16/09)
	\item Crear diagrama de clases (09/09-16/09)
	\item Leer red de distribución de agua desde un archivo .inp (16/09-30/09)
	\item Parsear entrada del archivo y crear una representación de la red utilizando estructuras de datos. (16/09-30/09)
	\item Realizar pruebas unitarias (23/09-07/10)
	\item Realizar pruebas de integración (23/09-07/10)
	\item Entregar de incremento y primera versión de los documentos generados en cada fase, este cargara la red y la representara usando un conjunto de clases. (07/10)
\end{itemize}

\paragraph{Etapa 2:} Desarrollar el objetivos 2 (07/10-25/11)
\begin{itemize}
	\item Capturar y especificación y refinación de los requisitos (07/10-14/10)
	\item Revisión la arquitectura lógica (07/10-14/10)
	\item Especificar casos de uso (07/10-14/10)
	\item Revisión y adaptación diagrama de clases (07/10-14/10)
	\item Corregir y refactorizar el código de la aplicación (07/10-28/10)
	\item Integrar algoritmo GA y sus operadores (07/10-11/11)
	\item Realizar pruebas unitarias (23/09-07/10)
	\item Realizar pruebas de integración (28/09-25/11)
	\item Entregar de incremento y segunda versión de los documentos generados en cada fase, este permitirá la ejecución del algoritmo GA a través de la terminal. (25/11)
\end{itemize}


\paragraph{Etapa 3:} Desarrollar el objetivos 3 (02/12-27/04)
\begin{itemize}
	\item Capturar y especificación y refinación de los requisitos (02/12-09/12)
	\item Revisión la arquitectura lógica (02/12-09/12)
	\item Especificar casos de uso (02/12-09/12)
	\item Revisión y adaptación diagrama de clases (02/12-09/12)
	\item Corregir y refactorizar el código de la aplicación (09/12-23/12) retomando el (02/03-16/03)
	\item Integrar algoritmo NSGA-II y sus operadores (02/03-27/04)
	\item Realizar pruebas unitarias (06/04-27/04)
	\item Realizar pruebas de integración (06/04-27/04)
	\item Entregar de incremento y tercera versión de los documentos generados en cada fase, este permitirá la ejecución del algoritmo GA y NSGA-II desde la terminal (27/04)
\end{itemize}

\paragraph{Etapa 4:} Desarrollar el objetivos 4 (27/04-06/07)
\begin{itemize}
	\item Capturar y especificación y refinación de los requisitos (27/04-04/05)
	\item Revisión la arquitectura lógica (27/04-04/05)
	\item Especificar casos de uso (27/04-04/05)
	\item Revisión y adaptación diagrama de clases (27/04-04/05)
	\item Diseño de interfaces graficas (27/04-04/05)
	\item Corregir y refactorizar el código de la aplicación (27/04-18/05)
	\item Implementar Interfaz y los menús de interacción (04/05-18/05)
	\item Mostrar la representación de la red en la interfaz (11/05-01/06)
	\item Implementar interfaz para visualizar resultados de la simulación (11/05-01/06)
	\item Guardar resultados de la simulación (11/05-01/06)
	\item Realizar pruebas unitarias (25/05-08/06)
	\item Realizar pruebas de integración (25/05-08/06)
	\item Entregar de incremento y cuarta y última versión de los documentos generados en cada fase, este permitirá la ejecución del algoritmo GA y NSGA-II desde una interfaz gráfica (08/06)
	\item Realizar el caso de estudio (08/06-13/07)
\end{itemize}


\bibliographystyle{unsrt}

\bibliography{referencias2}


\end{document}