\documentclass[11pt,letterpaper]{article}
\usepackage{pslatex}
\usepackage[spanish]{babel}
\usepackage[utf8]{inputenc} % Caracteres con acentos.
\usepackage{latexsym}
\usepackage{amssymb} 
\usepackage{amsmath}
\usepackage{epsfig}
\usepackage{url}

\begin{document}

\pagestyle{empty}

\title{
$<$Título de la propuesta de proyecto$>$\\
(Propuesta de proyecto final de carrera)\footnote{Los comentarios de este documento (texto en itálica), incluyendo esta nota a pie de página, deben ser removidos por el alumno al momento de elaborar su propuesta de proyecto.}}
\author{
Gabriel Sanhueza Fuentes (Estudiante)\\
Jimmy Gutierrez Bahamondes (Profesor guía)\\
Carrera de Ingeniería Civil en Computación\\ 
Universidad de Talca}
\date{6 de septiembre de 2019}

\maketitle


\section{Descripción de la propuesta}
\emph{(Esta sección debe incluir una presentación general del problema a investigar y/o idea a desarrollar. En esta sección se debe incluir aquellas referencias bibliográficas vinculadas al contexto del proyecto. Para esto último se recomienda el uso de un archivo *.bib, el cual usa el formato BibTex \cite{1} para codificar  referencias sobre libros \cite{2}, artículos en revistas científicas \cite{3}, artículos en conferencias o workshops \cite{4}, reportes técnicos \cite{5}, capítulos en libros \cite{6}, y páginas Web \cite{7}.
La longitud máxima de esta sección es de 2.5 páginas.)}

\subsection{Conceptos básicos del proyecto} 
\emph{Esta sección debe incluir el marco teórico o conceptos básicos que se requiere para comprender el proyecto a desarrollar.) Esta sección responde a la pregunta ¿Qué debo conocer para comprender el proyecto?}

\emph{(La longitud máxima de esta sección es de 1 página.)}


\begin{itemize}
\item \textbf{Optimización}: 
\item \textbf{Problemas Mono-objetivos}: 
\begin{itemize}
	\item \textbf{Exactos}: 
	\item \textbf{Heurísticos}: 
	\begin{itemize}
		\item \textbf{Algoritmos genéticos}: 
	\end{itemize}
\end{itemize}
\item \textbf{Problemas Multiobjetivo}:
	\begin{itemize}
		\item \textbf{NSGA-II}: 
	\end{itemize}
\item \textbf{JMetal}:
\item \textbf{Epanet}:
\item \textbf{Red de distribución de agua}:
 
\end{itemize}

\subsection{Contexto del proyecto} 
\emph{Esta sección debe incluir el marco en el cual se presenta el problema y/o proyecto a desarrollar, incluyendo los fundamentos teóricos y/o prácticos necesarios para el desarrollo del proyecto.) Esta sección responde a la pregunta ¿Dónde surge el problema?}

La escasez de agua potable es sin duda una problemática a nivel mundial. Dentro de este contexto, la optimización de los sistemas de distribución de agua potable es un problema sin resolver hasta la fecha. 

A lo mencionado anteriormente, también se suma la escasez energética. Puesto que, es requerida para el tratamiento y distribución del agua. Es por esto que, es importante hacer un uso eficiente de la energia durante la operación del sistema.

La optimización de estos sistemas, a la vez, involucra la participación de múltiples criterios que deben ser tomados en cuenta a la hora de decidir. Sin embargo, la incorporación de estos criterios, involucra la generación de modelos cada vez más complejos.

Los algoritmos metaheurísticos han demostrado ser un mecanismo eficiente ante problemas de este tipo. Ya que, estos reducen el tiempo necesario que toma el evaluar todas las configuraciones posibles. Dado que, juzgan un conjunto menor de valores logrando como resultado una aproximación a la solución óptima.

\subsection{Definición del problema} 
\emph{(Esta sección debe incluir la descripción del problema resolver o idea a desarrollar, y la motivación para hacerlo. Es decir, cual es la importancia, innovación, aporte, y/o beneficio para la ciencia y/o la humanidad). Esta sección responde a la pregunta ¿Cuál es el problema que voy a resolver?}

Los encargados de implementar sistemas de distribución de agua potable, no cuentan con suficientes herramientas y  tiempo para su correcta gestión. Por lo tanto, no es posible utilizar los recursos asociados de forma eficiente. Además, las herramientas existentes no satisfacen las necesidades de usabilidad y costo, debido a que son poco intuitivas y de pago.

El escoger las especificaciones de una red de agua potable ya es de por sí difícil debido a que hay que evaluar el rendimiento general del sistema alternando entre distintas configuraciones en busca de una solución que sea eficaz. Debido a esto, el uso de herramientas automatizadas que evalúen el rendimiento de las diversas combinaciones posibles viene a ser necesario.

A lo anterior se suma el hecho de que los interesados en esta área no manejan herramientas informáticas.

%Es por esto que con la realización de este trabajo se busca lograr la implementación de  una herramienta que ayude a los encargados a evaluar el diseño y la operación de la red de agua potable utilizando un enfoque especifico, contribuyendo a la ves al uso eficiente del agua y energía.



\subsection{Trabajo relacionado} 
\emph{(Esta sección debe incluir los enfoques usados actualmente para resolver el problema. Esta sección debe contener referencias bibliográficas a trabajos relacionados al proyecto.) Esta sección responde a la pregunta ¿Qué se ha hecho para resolver el problema?}

\subsection{Propuesta de solución}
\emph{(Esta sección debe incluir el planteamiento y justificación de la solución y/o idea, incluyendo aspectos novedosos.) Esta sección responde a la pregunta ¿Cómo voy a resolver el problema planteado?}

%\section{Hipótesis}
%\emph{(En esta sección se deben incluir una lista de afirmaciones o suposiciones las cuales se esperar responder con el desarrollo del proyecto. La longitud máxima de esta sección es de 1/2 página.)}
%\begin{itemize}
%\item El uso de ... puede facilitar ....
%\item El problema de .... puede estudiarse como ... 
%\item Las técnicas usadas en ... pueden ser aplicables para resolver el problema de ...
%\end{itemize}

\section{Objetivos}
\emph{(En esta sección se deben especificar el objetivo general y los objetivos específicos del proyecto. Los objetivos deben reflejar lo que se espera lograr con el proyecto, evitando incluir características específicas de la solución. La longitud máxima de esta sección es de 1 página.)}

\paragraph{Objetivo general}
\emph{(Debe ser una sola frase que resuma lo que se quiere lograr.)} 
\begin{itemize}
%\item Desarrollar un sistema web que permita administrar la asignación de salas de clase de forma automática y eficiente.
\item Generar un software capaz de cargar diferentes redes de agua potable, el cual por una parte haciendo uso de algoritmos de optimización multiobjetivo (NSGA-II) obtenga el régimen de bombeo que optimiza los costos operativos y energéticos y por otra parte resuelva un problema de diseño usando el enfoque mono-objetivo configurando los diámetros de las tuberías con el fin de reducir el costo de inversión.
\end{itemize}

\paragraph{Objetivos específicos} \emph{(Una lista de puntos que detallan el objetivo general.)}
\begin{itemize}
%\item Especificar los requerimientos del proyecto.
%\item Diseñar un algoritmo para ...
%\item Construir una herramienta para ...
%\item Evaluar el sistema a través de ... 
\item Especificar los requerimientos del proyecto.
\item Definir la manera de representar la red de agua potable y los elementos que la componen.
\item Elegir algoritmos metaheurísticos a utilizar para los problemas de diseño y operación.
\item Establecer indicadores de calidad para evaluar la solución obtenida a partir de los métodos metaheurísticos.
\item Evaluar el sistema a través de pruebas de usabilidad.
\end{itemize}



\section{Alcances}
\emph{(En esta sección se debe incluir una lista de puntos que definen los límites del trabajo. La longitud máxima de esta sección es de 1/2 página.)}
\begin{itemize}
\item En este trabajo se espera implementar un prototipo funcional de la idea a desarrollar, por lo tanto se propone desarrollar una interfaz de línea de comando simple en lugar de una interfaz de usuario gráfica. 
\item En este trabajo no se crearán algoritmos para ... 
\item Este trabajo se limita a ...
\end{itemize}



\section{Metodología}
\emph{(En esta sección se deben describir y justificar los métodos que se usarán en el desarrollo del proyecto de titulación. Los métodos obligatorios que debe incluir esta seccion es: Metodología de Desarrollo/Investigación y Metodología de Evaluación.)}

\subsection{Metodología de desarrollo/investigación}

\emph{(En esta subsección se deben describir y justificar los metodos de desarrollo y/o investigación que se aplicaran a lo largo del desarrollo del proyecto.)}


\emph{(La longitud máxima de esta sección es de 1 página.)}

\textbf{Por ejemplo:}
\begin{itemize}
	\item Metodologia de desarrollo tradicional RUP
	\item Metodologia de desarrollo ágil SCRUM
	\item Metodologia de investigación cientifica basado en M. BUNGE
\end{itemize}

\subsection{Metodología de evaluación del proyecto}
\emph{(En esta subsección se deben describir y justificar los metodos de evaluación/validación que se aplicarán a lo largo del desarrollo del proyecto.)}

\textbf{Por ejemplo:}
\begin{itemize}
	\item Metodología de evaluación basado en experimentos
	\item Metodología de evaluación basado en casos de estudio
		\item Metodología de evaluación basado action research
\end{itemize}

\part{%\paragraph{Objetivo 1:} ``Comparación de algoritmos para ..."
%\begin{itemize}
%\item Estudiar ...
%\item Seleccionar ...
%\item Comparar ...
%\end{itemize}
%
%\paragraph{Objetivo 2:} ``Especificación de requisitos del software"
%\begin{itemize}
%\item Analizar ...
%\item Clasificar ...
%\item Especificar ...
%\end{itemize}
}

\section{Plan de trabajo}
\emph{(En esta sección se debe definir como organizar y planificar, en términos de etapas y tiempo, las actividades a desarrollar así como los resultados a obtener.  Cada actividad debe generar un entregable. Se entiende como entregable a la generación de artefactos que sean útiles al desarrollo del proyecto, por ejemplo: documentos bibliográficos con resumen, documentos de desarrollo de software, código fuente, planificaciones, diagramas, esquemas, algoritmos, etc.)}

\emph{(Esta sección debe incluir además una Carta Gantt, la cual define fechas de inicio y término. La longitud máxima de esta sección es de 1 página, sin considerar la Carta Gantt.)}

\paragraph{Etapa 1:} Desarrollar el objetivo 1 (inicio-término)
\begin{itemize}
\item Analizar ... (inicio-término)
\item Redactar ... (inicio-término)
\end{itemize}

\paragraph{Etapa 2:} Desarrollar los objetivos 2 y 3 (inicio-término)
\begin{itemize}
\item Diseñar ... (inicio-término)
\item Programar ... (inicio-término)
\item Redactar ... (inicio-término)
\end{itemize}


\bibliographystyle{plain}

\bibliography{referencias}


\end{document}