\section{Optimizaci�n}
% Optimizaci�n consiste en maximizar o minimizar un conjunto de funciones que matem�ticamente pueden ser expresadas de la siguiente forma:

Optimizaci�n consiste en maximizar o minimizar un conjunto de funciones, modificando una serie de variables, conocidas como las variables de decisi�n o independientes. �sta puede ser expresada matem�ticamente de la siguiente forma:

$$f_1(x),f_2(x), ..., f_N(x),\ x=(x_1,...,x_d) | x \in X$$

\noindent sujeto a una serie de restricciones

$$h_j(x) = 0, j=1,2,...,J$$
$$g_k(x) \leq 0, k=1,2,...,K$$

\noindent siendo $f_1,...,f_N$ funciones objetivos; $x_1, ..., x_d$ variables de decisi�n, pertenecientes al espacio de b�squeda $X$; y $h_j$ junto con $g_k$, una serie de restricciones~\cite{Yang2015}. De acuerdo a la cantidad de funciones objetivos ($N$) que se tenga, se establece que si $N=1$ la optimizaci�n es \textbf{monoobjetivo}, mientras que para $N\geq 2$ se conoce como \textbf{multiobjetivo}~\cite{Yang2015}. Cada uno de estos tipos de problemas debe ser abordado con metodolog�as espec�ficas para su prop�sito ya que sus objetivos son diferentes; por un lado, los problemas monoobjetivos persiguen encontrar una �nica soluci�n, mientras que los problemas multiobjetivos se enfocan en determinar un conjunto de soluciones llamado Frontera de Pareto que ser� descrito en el apartado~\ref{sec:frontera_pareto}. En este punto se debe tener en cuenta que los objetivos planteados deben encontrarse en contradicci�n. 

Debido a la definici�n de las restricciones es posible dividir el espacio de b�squeda en dos regiones~\cite{Bozorg-Haddad2017}:
\begin{itemize}
	\item Soluciones factibles: Compuesto por los elementos pertenecientes al espacio de b�squeda que satisfacen todas las restricciones.
	\item Soluciones no factibles: Integrado por aquellos elementos que no cumplen todas las restricciones.
\end{itemize}