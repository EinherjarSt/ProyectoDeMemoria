\section{Metodolog�a de evaluaci�n}
\subsection{Casos de estudio}
Un caso de estudio \cite{Runeson2009} es una metodolog�a de investigaci�n la cual en ingenier�a de software permite analizar un proyecto, un grupo de personas, un producto, etc, en un contexto real con el objetivo de responder la pregunta de investigaci�n planteada. Esta metodolog�a de evaluaci�n considera aspectos formales para obtener evidencia. Los principales aspectos son:

\begin{itemize}
	\item Describir el contexto de aplicaci�n del caso: Consiste en establecer sobre que elemento se aplicara el caso de estudio a desarrollar.
	\item Definici�n de objetivos experimentales: Consiste en indicar cual es el objetivo de la investigaci�n a realizar, si es describir, evaluar o explicar alg�n suceso.
	\item Definici�n de un protocolo para conducir el caso de estudio: Consiste en escoger las pautas para llevar a cabo el caso de estudio, que instrumentos ser�n utilizados para recolectar datos y como se realizaran el an�lisis de estos.
	\item Definici�n de caracter�sticas a evaluar: Consiste en establecer qu� es lo que estamos interesados en evaluar del elemento sobre el que se aplica el caso de estudio.
	\item Definici�n de sujetos de prueba: Consiste en indicar cual sera la fuente de datos a ser utilizada para el caso de estudio, estas pueden ser personas, datos ya recolectados, etc.
	\item Aplicaci�n de caso de estudio en un conjunto de sesiones no controladas: Consiste en llevar a cabo el caso de estudio sobre los sujeto de prueba definidos.
	\item Aplicaci�n de herramientas de obtenci�n de evidencia emp�rica: Consiste en la utilizaci�n de m�todos o t�cnicas con el fin de obtener evidencia emp�rica a partir de los datos recolectados.
	\item An�lisis y evaluaci�n de datos emp�ricos: Consiste en analizar la evidencia y evaluar la validez de los resultados obtenidos.
\end{itemize}
