\section{Herramientas para la implementaci�n del software}
A continuaci�n se presentan las herramientas utilizadas para el desarrollo del proyecto.

\subsection{Java}
Java es un lenguaje de programaci�n de alto nivel orientado a objetos y de prop�sito general. Un programa Java se ejecuta sobre la maquina virtual llamada la \textit{Java Virtual Machine}, la cual le da a este lenguaje la caracter�stica de ser multiplataforma. Adicionalmente, Java incorpora el soporte para multi-hilos, una poderosa herramienta que permite la ejecuci�n de distintas instrucciones de c�digo al mismo tiempo~\cite{Gosling2015}. Ademas. este lenguaje tambi�n incorpora una caracter�stica conocida como el recolector de basura, que se encarga de limpiar la memoria de objetos que ya no est�n siendo utilizados.  Fue anunciado por Sun Microsystems en Mayo de 1995~\cite{3Java}. 

\subsubsection{Java Reflection}
Caracter�stica de Java que permite que un programa se auto examine. Esta caracter�stica est� disponible a trav�s de la Java Reflection API, la cual cuenta con m�todos para obtener los metadatos de las clases, m�todos, constructores, campos o par�metros. Esta API tambi�n permite crear nuevos objetos cuyo tipo era desconocido al momento de compilar el programa~\cite{Braux1999}.

\subsubsection{Java Annotation}
Caracter�stica de Java para agregar metadatos a elementos del lenguaje como las clases, m�todos, par�metros, etc~\cite{Rocha2011}. Las anotaciones no tienen efecto directo sobre el c�digo. Sin embargo, combinadas con \textit{Java Reflection} permiten realizar una serie de tareas muy �tiles como crear nuevos objetos cuyo tipo no conocemos en tiempo de compilaci�n.

Las anotaciones est�n presentes en varios lenguajes como Python y C\#, siendo en este �ltimo llamadas Atributos. Generalmente, los lenguajes que incorporan anotaciones tambi�n implementan la t�cnica llamada reflexi�n.

A continuaci�n se presenta un ejemplo de la anotaci�n \textit{@Override} en Java:

\begin{center}
        \begin{minipage}[t]{10cm}
            \begin{verbatim}
                @Override
                public void toString(){}
            \end{verbatim}
        \end{minipage}
\end{center}   


% Estas herramientas pueden analizar estas anotaciones y realizar acciones en base a estas, por ejemplo, generar archivos adicionales como clases de Java, archivos xml, ser analizadas durante la ejecuci�n del programa v�a Java Reflection, para crear objetos cuyo tipo no conocemos en tiempo de compilaci�n; etc. 

\subsection{Epanet Programming Toolkit}
% Software que permite simular el comportamiento hidr�ulico y la calidad del agua en redes de distribuci�n de aguas compuesta por tuber�as, nodos, bombas, v�lvulas y tanques de almacenamiento~\cite{Rossman2017}.  Este software cuenta tambi�n con una librer�a din�mica conocida bajo el nombre de Epanet Programming Toolkit, la cual cuenta con un conjunto de funciones para realizar simulaciones desde diferentes entornos de desarrollo como C, C++, VB, Java, etc~\cite{Rossman1999}.

Librer�a de enlace din�mico que permite realizar simulaciones computaciones del comportamiento del agua en RDA. Esta librer�a es parte de Epanet, un software que permite simular el comportamiento hidr�ulico y la calidad del agua en redes de distribuci�n de aguas compuesta por tuber�as, nodos, bombas, v�lvulas y tanques de almacenamiento~\cite{Rossman2017}. Fue creada por la agencia EPA de EE.UU. La librer�a cuenta con un conjunto de funciones para realizar simulaciones desde diferentes entornos de desarrollo como C, C++, VB, Java, etc~\cite{Rossman1999}.