\section{Trabajo relacionado}
En este apartado se mostrar�n distintas herramientas y el enfoque utilizado para resolver el problema con ellas. En general, los enfoques consisten en usar software ya disponible o crear un software personalizado.

\begin{itemize}
	\item \textbf{Magmoredes}: En~\cite{Edwin2017} se describe la existencia de un software de dise�o basado en micro-algoritmos gen�ticos multiobjetivos, que de acuerdo al autor, tiene un mejor rendimiento y es m�s eficiente que el algoritmo NSGAII. Esto se debe a que requiere una menor cantidad de memoria y tiene un mejor tiempo de c�mputo. Este programa puede cargar cualquier red y realiza los c�lculos utilizando librer�as de Java. Las funciones objetivos que este sistema intenta resolver son la optimizaci�n de los costos de construcci�n de la red y la confiabilidad final de la red.
	
	\item \textbf{WaterGEMS}: Software comercial que permite la construcci�n de modelos geoespaciales; optimizaci�n de dise�o, ciclos de bombeo y calibraci�n autom�tica del modelo; y la gesti�n de activos. Este software a sido usado en~\cite{Mehta2017} para llevar a cabo las simulaciones necesarias para su estudio. La metodolog�a seguida para la utilizaci�n de este sistema consiste en ingresar los datos a WaterGEMS para correr las simulaciones. La limitaci�n de este programa es que no permite la adici�n de nuevos algoritmos por parte del usuario, en el caso de que se quiera probar o mejorar alg�n algoritmo ya existente. Sin embargo, este sistema tambi�n tiene sus ventajas, porque ya incorpora algunos algoritmos predefinidos para resolver ciertos problemas. Adem�s, posee diversas funcionalidades como conexi�n con datos externos, operaciones de an�lisis espaciales, intercambio de datos con dispositivos o programas de administraci�n, entre otros.
	
	\item \textbf{EPANET}: El enfoque seguido con la utilizaci�n de esta herramienta consiste en automatizar la ejecuci�n de los algoritmos y la posterior evaluaci�n de los resultados utilizando la librer�a EPANET Toolkit. Este es usado en~\cite{Doctoral2012} en donde se implementan ciertos algoritmos metaheur�sticos y los resultados obtenidos por estos son enviados a la EPANET Toolkit para evaluar la soluci�n y determinar la factibilidad de �sta. La ventaja de este enfoque es que permite una mayor flexibilidad en los algoritmos metaheuristicos utilizados y los problemas que se quieren resolver. Sin embargo, debido a que se necesita implementar los problemas y los algoritmos, este enfoque toma mucho tiempo.
\end{itemize}

Para el desarrollo de este proyecto se usa el enfoque basado en EPANET, puesto que es una librer�a de simulaci�n hidr�ulica ampliamente utilizada y permite enfocarnos en la resoluci�n de los problemas usando algoritmos metaheur�sticos. Tanto Magmoredes como WaterGEMS buscan resolver temas concretos en los sistemas de distribuci�n de agua potable y puesto que el c�digo de estos programas no esta disponible p�blicamente o son un sistema que se comercializa sin permitir la modificaci�n del sistema por parte de terceros, se busca con nuestro proyecto incorporar esta capacidad para que en futuros trabajos se pueda abarcar una mayor cantidad de problemas en nuestro sistema.