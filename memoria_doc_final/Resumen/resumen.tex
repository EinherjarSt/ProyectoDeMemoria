\begin{resumen}
La escasez de agua potable es un problema a nivel mundial. Cada d�a aumenta la necesidad de utilizar eficientemente los recursos h�dricos disponibles. Sin embargo, la optimizaci�n de todos los procesos involucrados en su gesti�n es una tarea compleja. Especialmente, la distribuci�n del agua es un procedimiento dif�cil de optimizar debido a que implica el modelamiento de variables f�sicas que se relacionan de manera no lineal \cite{Basha1996}. Diversos investigadores han abordado el problema desde diferentes perspectivas. Donde una gran cantidad de trabajos ha destacado a los algoritmos metaheur�sticos como herramientas eficientes frente a la resoluci�n de problemas complejos de esta �rea. Debido principalmente a su capacidad de exploraci�n del espacio de posibles soluciones en tiempos razonables. A pesar de ello, existen muy pocos sistemas computacionales que permitan a los ingenieros hidr�ulicos optimizar el dise�o y operaci�n de las redes de distribuci�n de agua (RDA) desde una interfaz gr�fica, sin tener que codificar.  

Es por ello que este proyecto busca llevar a cabo el desarrollo de una herramienta que permita la optimizaci�n de problemas presentes en RDA. En t�rminos generales, el objetivo es dise�ar e implementar una aplicaci�n de escritorio que pueda ser utilizada por personas interesadas en la labores del dise�o y operaciones de RDA. A pesar de que existe muchos problemas relacionados con esta tem�tica, este proyecto se centrar� en dos. Los cuales han sido seleccionados por un grupo de expertos que apoya este proyecto. El primero corresponde a la optimizaci�n del dise�o de RDA basado en la minimizaci�n del costo de inversi�n seg�n la selecci�n de di�metros de tuber�as. En segundo lugar, el problema de planificaciones de las operaciones de los sistemas de bombeo que optimiza simult�neamente los costos energ�ticos y de mantenimiento de los equipos. Para ello, el proyecto contempla la implementaci�n de un Algoritmo Gen�tico (GA) \cite{Heiss-Czedik1997} para la resoluci�n de problemas monoobjetivos, el algoritmo  "Nondominated Sorting Genetic Algorithm II" (NSGA-II) \cite{Deb2002} para el caso de problemas multiobjetivos y todos los m�todos que permiten realizar las simulaciones hidr�ulicas de RDA.

Adicionalmente, es importante destacar que este nuevo sistema debe poseer una arquitectura escalable que sirva como base para la adici�n de nuevos problemas y algoritmos a medida que sean requeridos.

A continuaci�n se presentan los conceptos b�sicos que son necesarios para el desarrollo de este proyecto. Posteriormente, se presenta el problema y contexto del proyecto. Finalmente, se presenta la propuesta de soluci�n al problema identificado.

Para el desarrollo de este proyecto se opto por utilizar la metodolog�a de desarrollo Iterativo e Incremental.

Para la evaluaci�n del sistema a desarrollar se usara la metodolog�a de caso de estudio cuyo resultado permitir� determinar la utilidad del software desarrollado. 
\end{resumen}