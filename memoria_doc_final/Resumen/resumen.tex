\begin{resumen}

La escasez de agua potable es un problema a nivel mundial. Cada d�a aumenta la necesidad de utilizar eficientemente los recursos h�dricos disponibles. Sin embargo, la optimizaci�n de todos los procesos involucrados en su gesti�n es una tarea compleja debido a que implica el modelamiento de variables f�sicas que se relacionan de manera no lineal~\cite{Basha1996}. Junto a lo anterior, se tiene que los encargados de implementar los sistema de distribuci�n de agua (RDA), no cuentan con las suficientes herramientas y tiempo para la correcta gesti�n de estos sistemas.

Es por ello que este proyecto busca llevar a cabo el desarrollo de una herramienta de escritorio extensible que permita la optimizaci�n de los procesos de dise�o y operaci�n en RDA. Con el fin de valorar la capacidad del programa, se han implementado dos problemas los cuales son la optimizaci�n del dise�o de RDA basado en la selecci�n del di�metro de tuber�as; y la optimizaci�n del r�gimen de bombeo a traves de un enfoque multiobjetivo. 

La metodolog�a utilizada para la implementaci�n es la iterativa e incremental. �sta metodolog�a fue escogida principalmente por la documentaci�n que su aplicaci�n genera, as� como la retroalimentaci�n e importancia de la participaci�n del cliente en el desarrollo del proyecto. 

La implementaci�n se realiz� utilizando el lenguaje de programaci�n Java. Para la implementaci�n de la herramienta fueron importantes 2 funcionalidades de este lenguaje las cuales son \textit{Java Reflection} y \textit{Java annotation}. �stas nos permitieron lograr un mejor desacoplamiento entre las clases facilitando la extensi�n de la aplicaci�n para agregar nuevos algoritmos metaheur�sticos y problemas.

Para realizar el proceso de evaluaci�n de la soluci�n se utilizaron los estudios de caso, los cuales permiten analizar el objeto de estudio en un contexto real. Este estudio se realiz� sobre profesionales con conocimientos en hidr�ulica y computaci�n, dando como resultado una favorable aceptaci�n de la herramienta.

Al finalizar el proceso de desarrollo y evaluaci�n, se concluye que los objetivos planteados fueron logrados exitosamente. Tambi�n se presentan una serie de sugerencias que pueden ser implementadas para mejorar la herramienta como la implementaci�n de nuevos algoritmos y problemas; m�todos para comparar algoritmos metaheur�sticos, entre otros.

\end{resumen}