\section{Contexto del proyecto} 
La escasez de agua potable es sin duda una problem�tica a nivel mundial y la optimizaci�n de los sistemas que permiten su distribuci�n es cada d�a m�s relevante. Existe una serie de problem�ticas asociadas a la determinaci�n de las condiciones �ptimas de operaciones y las caracter�sticas adecuadas para su construcci�n.

Las redes de agua potable (RDA) son redes que pueden ser muy extensas y complejas. Forman parte de la estructura principal de cualquier ciudad. Deben ser capaces de adaptarse a los cambios y asegurar niveles m�nimos de servicios durante las 24 horas del d�a~\cite{Pereyra2017}. Adicionalmente, dependiendo de su topolog�a, las RDA integran sistemas de bombeo que requieren gran cantidad de energ�a en horarios determinados.

La optimizaci�n de estos sistemas, a la vez, involucra la participaci�n de m�ltiples criterios que deben ser tomados en cuenta a la hora de decidir. Sin embargo, la incorporaci�n de �stos, involucra la generaci�n de modelos cada vez m�s complejos~\cite{JHawanet-2019}.

Por lo anteriormente mencionado, esta �rea, ha llamado la atenci�n de muchos investigadores que han creado diversos m�todos para resolver la problem�tica desde diferentes enfoques. Sin embargo, a�n existen muy pocas aplicaciones computacionales que permitan emplear los nuevos modelos y t�cnicas de forma pr�ctica. �sto supone un gran problema para los interesados en aplicar estos conocimientos en un contexto real. Generalmente se trata de personas instruidas en tem�ticas relacionadas con la hidr�ulica pero que poseen un escaso manejo de t�cnicas computacionales de optimizaci�n.

En este trabajo se pretende dar respuesta a esa necesidad creciente a trav�s del dise�o e implementaci�n de una aplicaci�n de escritorio. Este nuevo sistema, permitir� a a los usuarios resolver dos de los principales problemas en la optimizaci�n de RDA. En el caso de los problemas con solo un objetivo se utilizar� un Algoritmo Gen�tico, mientras que en los problemas multiobjetivos se utilizar� NSGAII.