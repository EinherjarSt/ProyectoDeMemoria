\section{Alcances}
Los alcances propuestos para este proyecto son los siguientes:

\begin{itemize}
	\item Esta herramienta s�lo puede ser ejecutada en equipos con el sistema operativo Window de 64bits. Esta limitaci�n se debe a que se realizan llamadas a librer�as nativas que fueron compiladas para sistemas de 64bits.
	\item El sistema permite la carga y la visualizaci�n de la red gr�ficamente.
	\item El sistema permite visualizar la configuraci�n b�sica almacenada en el archivo de configuraci�n de red.
	\item Este proyecto no contempla la creaci�n de la red desde el programa a desarrollar.
	\item El archivo de configuraci�n de red debe ser creado utilizando la aplicaci�n Epanet y guardado con la extension inp.
	\item El sistema �nicamente cuenta con dos algoritmos implementados por defecto los cuales son el Algoritmo Gen�tico (GA) y \textit{Non-Dominated Sorting Genetic Algorithm II} (NSGAII).
	\item El sistema s�lo resuelve dos tipos de problemas de optimizaci�n, uno monoobjetivo y el otro multiobjetivo. El Algoritmo Gen�tico se utiliza para generar soluciones para el problema monoobjetivo con el objetivo de optimizar el costo de inversi�n de las tuber�as. Por otro lado, el algoritmo NSGAII aborda el problema multiobjetivo con el fin de optimizar los costos energ�ticos y el r�gimen de bombeo.  % El problema monoobjetivo ser� el de los costos de inversi�n. En cuanto al problema multiobjetivo, este ser� el de los costos energ�ticos y el n�mero de encendidos y apagado de las bombas. 
	\item El sistema permite visualizar �l o los resultados obtenidos al finalizar la ejecuci�n del algoritmo.
	\item El sistema permite utilizar una soluci�n obtenida, a trav�s de los algoritmos metaheur�sticos, para generar un nuevo archivos de configuraci�n de red. %Debido a que el archivo .inp contiene una gran cantidad de configuraciones, las �nicas que se permitir�n modificar ser�n las configuraciones asociadas a los conexiones (junctions) y tuber�as (pipes).
	\item El sistema permite generar archivos con las soluciones obtenidas para cada problema, es decir, el valor de los objetivos y las variables de decisi�n involucradas.
	\item El sistema permite gr�ficar visualmente las soluciones en un plano cartesiano.
	\item La gr�fica �nicamente est� disponible en problemas con uno o dos objetivos.
\end{itemize}

%Este proyecto no contempla la creaci�n de la red por lo que estas deber�n ser ingresadas como entradas al programa, es decir, estas deber�n ser creadas usando EPANET o manualmente, pero siguiendo el formato establecido por EPANET. Adem�s, esta herramienta �nicamente podr� ser ocupada en equipos cuyo sistema operativo sea Windows debido a que se realizan llamadas a librer�as nativas.