\section{Alcances}
Los alcances propuestos para este proyecto ser�n los siguientes:
\begin{itemize}
	
	\item El sistema permitir� la carga y la visualizaci�n de la red gr�ficamente.
	\item El sistema permitir� visualizar la configuraci�n b�sica almacenada en el archivo inp de los elementos de la red.
	\item El sistema solo resolver� dos clases de problemas de optimizaci�n, uno monoobjetivo y el otro multiobjetivo. El problema monoobjetivo ser� el de los costos de inversi�n. En cuanto al problema multiobjetivo, este ser� el de los costos energ�ticos y el n�mero de encendidos y apagado de las bombas. 
	\item El sistema �nicamente contara con dos algoritmos implementados los cuales ser�n el algoritmo gen�tico y NSGA-II. El algoritmo gen�tico ser� el usado para tratar el problema monoobjetivo, mientras que NSGA-II ser� aplicado al multiobjetivo.
	\item El sistema permitir� visualizar el o los resultados obtenidos al finalizar la ejecuci�n del algoritmo.
	\item El sistema permitir� generar archivos .inp con la configuraci�n de la soluci�n obtenida a trav�s de los algoritmos. Debido a que el archivo .inp establece una gran cantidad de configuraciones, las �nicas que se permitir�n modificar ser�n las configuraciones asociadas a los conexiones (junctions) y tuber�as (pipes).
	\item El sistema permitir� generar archivos con las soluciones obtenidas para cada problema, es decir, el valor de los objetivos y las variables de decisi�n involucradas.
	\item El sistema permitir� graficar visualmente las soluciones en un plano cartesiano.
	\item La gr�fica �nicamente estar� disponible en problemas con uno o dos objetivos.
	\item La comparaci�n con las redes de benchmarking consistir� �nicamente, en presentar una tabla comparativa entre los resultados presentes en la literatura y los obtenidos a trav�s de nuestro sistema, para cada uno de los problemas de redes de distribuci�n de agua potable de nuestro sistema.
	
\end{itemize}
Este proyecto no contempla la creaci�n de la red por lo que estas deber�n ser ingresadas como entradas al programa, es decir, estas deber�n ser creadas usando EPANET o manualmente, pero siguiendo el formato establecido por EPANET. Adem�s, esta herramienta �nicamente podr� ser ocupada en equipos cuyo sistema operativo sea Windows debido a que se realizan llamadas a librer�as nativas.