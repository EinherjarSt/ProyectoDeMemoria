\section{Alcances}
Los alcances propuestos para este proyecto son los siguientes:

\begin{itemize}
	\item Esta herramienta s�lo puede ser ejecutada en equipos con el sistema operativo Window de 64bits. Esta limitaci�n se debe a que se realizan llamadas a librer�as nativas que fueron compiladas para sistemas de 64bits.
	\item El sistema permite la carga y la visualizaci�n de la red gr�ficamente.
	\item El sistema permite visualizar la configuraci�n b�sica almacenada en el archivo de configuraci�n de red.
	\item Este proyecto no contempla la creaci�n de la red desde el programa a desarrollar.
	\item El archivo de configuraci�n de red debe ser creado utilizando la aplicaci�n Epanet y guardado con la extension inp.
	\item El sistema implementa por defecto dos algoritmos. Uno de ellos monoobjetivo y el otro multiobjetivo. El algoritmo monoobjetivo corresponde al Algoritmo Gen�tico (GA) mientras que el multiobjetivo a \textit{Non-Dominated Sorting Genetic Algorithm II} (NSGAII).
	\item El sistema proporciona dos problemas ya implementados uno monoobjetivo y el otro multiobjetivo. El Algoritmo Gen�tico se utiliza para generar soluciones para el problema monoobjetivo con el fin de optimizar el costo de inversi�n de las tuber�as. Por otro lado, el algoritmo NSGAII aborda el problema multiobjetivo con el objetivo de optimizar los costos energ�ticos y costo de mantenimiento de los equipos de bombeo.  
	\item El sistema permite visualizar �l o los resultados obtenidos al finalizar la ejecuci�n del algoritmo.
	\item El sistema permite utilizar una soluci�n obtenida, a trav�s de los algoritmos metaheur�sticos, para generar un nuevo archivos de configuraci�n de red. 
	\item El sistema permite generar archivos con las soluciones obtenidas para cada problema, es decir, el valor de los objetivos y las variables de decisi�n involucradas.
	\item El sistema permite gr�ficar visualmente las soluciones en un plano cartesiano.
	\item La gr�fica �nicamente est� disponible en problemas con uno o dos objetivos.
\end{itemize}