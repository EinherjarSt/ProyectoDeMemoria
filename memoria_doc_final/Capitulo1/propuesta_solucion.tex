\section{Propuesta de soluci�n}
La soluci�n que se propone para abordar el problema consiste en el desarrollo de una aplicaci�n de escritorio extensible que permita optimizar procesos de dise�o y operaci�n en RDA. Este software entrega a sus usuarios la posibilidad de agregar nuevos problemas y algoritmos si se consideran necesarios.

Los problemas que se abordan en el contexto de optimizaci�n de redes de agua potable ser�n:
\begin{itemize}

	\item \textbf{Problema de dise�o:} Este tipo de problema busca optimizar las configuraciones y la disposici�n de los elementos que conforman la red previa a su construcci�n~\cite{Jimmy-2019}. 

	\item \textbf{Problema de operaci�n:} Los problemas de operaci�n buscan optimizar las configuraciones de una red ya construida~\cite{Jimmy-2019}.
\end{itemize}

Con el fin de valorar la capacidad del programa, dos problemas hidr�ulicos son implementados. Cada uno con un algoritmo af�n seg�n sus caracter�sticas.

La selecci�n de estos problemas se llevo a cabo teniendo en cuenta la necesidad de los interesados en el estudio de redes de distribuci�n de agua. Espec�ficamente, los participantes del proyecto de investigaci�n \textit{Optimization of real-world water distribution systems and hydraulic elements using computational fluid dynamics (cfd) and evolutionary algorithms} financiado por la Agencia Nacional de Investigaci�n y Desarrollo ANID, Chile y sus colaboradores en la Universidad Polit�cnica de Valencia, Espa�a.

Seg�n lo anteriormente expuesto, los problemas seleccionados son los siguientes:

\begin{enumerate}
	\item Problema de dise�o de RDA basado en el costo de tuber�as.  El cual consiste en la optimizaci�n de los costo de inversi�n variando el di�metro de las tuber�as (\textit{Pipe Optimizing})~\cite{Iglesias-2004, Pereyra2017}. Este problema sera abordado utilizando el Algoritmo Gen�tico (GA). 
	\item Problema de operaci�n basado en el r�gimen de bombeo. �ste consiste en la optimizaci�n de costos energ�ticos y el costo de mantenimiento de las bombas (\textit{Pumping Schedule}) \cite{JHawanet-2019}. Este problema se aborda desde el enfoque multiobjetivo utilizando el algoritmo\textit{Non-Dominated Sorting Genetic Algorithm}, versi�n II (NSGAII). 
\end{enumerate}

Adicionalmente, se genera una gu�a para que los usuarios puedan implementar nuevos problemas, algoritmos y operadores. 