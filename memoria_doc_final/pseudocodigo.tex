\documentclass[11pt,letterpaper]{article}
\usepackage{pslatex}
\usepackage[spanish]{babel}
\usepackage[utf8]{inputenc} % Caracteres con acentos.
\usepackage{latexsym}
\usepackage{amssymb} 
\usepackage{amsmath}
\usepackage{epsfig}
\usepackage{url}
\usepackage{titlesec}
\usepackage{float}
\usepackage[onelanguage, commentsnumbered, linesnumbered, boxed, ruled]{algorithm2e}
\usepackage{pdfpages}
\usepackage{amsfonts}
\usepackage{amssymb}

\begin{document}
	\begin{figure}[h]
		\begin{algorithm}[H]
			\caption{Algoritmo Evolutivo}
			\DontPrintSemicolon
			\SetKwFunction{CreateInitialPopulation}{crearPoblaciónInicial}
			\SetKwFunction{EvaluatePopulation}{evaluarPoblación}
			\SetKwFunction{Selection}{selección}
			\SetKwFunction{Crossover}{cruzamiento}
			\SetKwFunction{Mutation}{mutación}
			\SetKwFunction{Replace}{remplazar}
			\SetKwData{PopulationVar}{población}
			\SetKwData{SelectionVar}{poblacionSeleccionada}
			\SetKwData{OffspringVar}{poblaciónDecendiente}
			
			\PopulationVar $\leftarrow$ \CreateInitialPopulation{}\;
			\EvaluatePopulation{\PopulationVar}\;
			
			\While{la condición de termino no ha sido alcanzada}{
				\SelectionVar $\leftarrow$ \Selection{\PopulationVar}\;
				\OffspringVar $\leftarrow$ \Crossover{\SelectionVar}\;
				\OffspringVar $\leftarrow$ \Mutation(\OffspringVar)\;
				\OffspringVar $\leftarrow$ \EvaluatePopulation(\OffspringVar)\;
				\PopulationVar $\leftarrow$ \Replace(\PopulationVar, \OffspringVar)\;
			}
			
		\end{algorithm}
		\caption{Pseudocodigo algoritmo evolutivo}
		\label{fig:EA}
	\end{figure}

\begin{figure}[h]
	\begin{algorithm}[H]
		\caption{Función de remplazo para el algoritmo NSGAII}
		\DontPrintSemicolon
		\SetKwData{PopulationVar}{población}
		\SetKwData{OffspringVar}{poblaciónDescendiente}
		\SetKwData{JoinVar}{unionPoblacion}
		\SetKwData{Front}{$F$}
		\SetKwData{NewPopulation}{nuevaPoblacion}
		
		\SetKwFunction{FastNonDominatingSorting}{ordenarPorFrentesNoDominados}
		\SetKwFunction{CrowdingDistanceAssignament}{asignarDensidad}
		\SetKwFunction{Sort}{ordernar}
		
		\SetKwProg{Fn}{Function}{}{fin}
		\Fn{remplazar(\PopulationVar, \OffspringVar)}{
			
			$\JoinVar \leftarrow \PopulationVar \cup \OffspringVar$\;
			\BlankLine
			\tcc*[h]{$F = (F_1,F_2, ...)$}\;
			\Front $\leftarrow$ \FastNonDominatingSorting(\JoinVar)\;
			\BlankLine
			\NewPopulation $\leftarrow \emptyset$ \;
			\BlankLine
			$i = 1$\;
			\BlankLine
			\tcc*[h]{Hasta que \NewPopulation este lleno}\;
			\While{$(|\NewPopulation| +|F_i|) \leq N$}{
				\tcc{Calcular y asignar la densidad a cada solución del frente $F_i$}
				\CrowdingDistanceAssignament($F_i$)\;
				\BlankLine
				\tcc{Añadir a \NewPopulation las soluciones del frente $F_i$}
				\NewPopulation $\leftarrow$ \NewPopulation $\cup$ $F_i$ \;
				\BlankLine
				$i = i+1$\;
			}
			\BlankLine
			\tcc{Ordenar el frente $F_i$ usando el comparador de densidad}
			\Sort($F_i$, $\prec_n$) \;
			\BlankLine
			\tcc{Elegir los primeros $N -|\NewPopulation|]$ }
			\NewPopulation $\leftarrow$ \NewPopulation $\cup$ $F_i[1: N -|\NewPopulation|]$ \;
			\BlankLine
			retornar \NewPopulation\;
		}
		
	\end{algorithm}
	\caption[Pseudocódigo de la función de remplazo utilizada en el algoritmo NSGAII]{Pseudocódigo de la función de remplazo utilizada en el algoritmo NSGAII~\cite{Deb2002}}
	\label{fig:RemplaceNSGAII}
\end{figure}

\begin{figure}[h]
	\begin{algorithm}[H]
		{Función de ordenación en frentes no dominados}
		\DontPrintSemicolon
		\SetKwData{PopulationVar}{población}
		\SetKwData{Front}{$F$}
		\SetKwData{NewPopulation}{nuevaPoblacion}
		
		\SetKwFunction{FastNonDominatingSorting}{ordenarPorFrentesNoDominados}
		
		
		
		\SetKwProg{Fn}{Function}{}{fin}
		\Fn{\FastNonDominatingSorting(\PopulationVar)}{
			$\Front \leftarrow \emptyset$\;
			\ForEach{$p \in \PopulationVar$}{
				\tcc*[h]{Conjunto de soluciones dominadas por $p$}\;
				$S_p = \emptyset$\; 
				\tcc*[h]{Numeros de soluciones que dominan a $p$}\;
				$n_p = 0$\;
				\ForEach{$q \in \PopulationVar$}{
					\uIf(\tcc*[h]{$p$ domina a $q$}){$p \prec q$}{
						$S_p = S_p \cup \{q\}$\;
					}\ElseIf(\tcc*[h]{$q$ domina a $p$}){$q \prec p$}{
						$n_p = n_p + 1$\;
					}
					\If{$n_p = 0$}{
						$p_{rango} = 1$\;
						$F_1 = F_1 \cup \{p\}$\;
					}	
				}
			}
			\BlankLine
			$i = 1$\;
			\BlankLine
			\While{$F_i \neq 0$}{
				$Q = \emptyset$\;
				\ForEach{$p \in \Front_i$}{
					\ForEach{$q \in S_p$}{
						$n_q = n_q - 1$\;
						\If{$n_q = 0$}{
							$q_{rango} = i + 1$\;
							$Q = Q \cup \{q\}$\;
						}
					}
				}
				
				$i = i+1$\;
				$F_i = Q$\;
			}
			
			
			
		}
		
	\end{algorithm}
	\caption[Pseudocódigo de la función de ordenamiento utilizada en NSGAII]{Pseudocódigo de la función de ordenamiento utilizada en NSGAII~\cite{Deb2002}}
	\label{fig:sortFront}
\end{figure}

\begin{figure}[h]
	\begin{algorithm}[H]
		\caption{Función de calculo de densidades}
		\DontPrintSemicolon
		\SetKwData{NDS}{$\mathcal{I}$}
		\SetKwFunction{CrowdingDistanceAssignament}{asignarDensidad}
		
		
		
		\SetKwProg{Fn}{Function}{}{fin}
		\Fn(\tcc*[h]{$\NDS$ : frente de soluciones no dominadas}){\CrowdingDistanceAssignament($\NDS$)}{
			$l = |\NDS|$ \tcc*[h]{Obtiene el tamaño del frente}\;\;
			\BlankLine
			\tcc*[h]{Inicializa la distancia para cada solución}\;
			\ForEach{$i \leftarrow 1$ \KwTo $l$}{
				$\NDS[i]_{distancia} \leftarrow 0$
			}
			\ForEach{$m \leftarrow 1$ \KwTo numero de objetivos}{
				$\NDS \leftarrow$ sort($\NDS$, m) \tcc*[h]{Ordena por el objetivo m}\; 
				\BlankLine
				\tcc*[h]{Asignar a la primera y ultima solucion el valor $\infty$ }\;
				$\NDS[1]_{distancia} \leftarrow \NDS[l]_{distancia} \leftarrow \infty$\;
				\BlankLine
				\For{$i \leftarrow 2$ \KwTo $l-1$}{
					\tcc*[h]{Asigna la distancia a las soluciones restantes}\;
					$\NDS[i]_{distancia} \leftarrow \NDS[i]_{distancia} + (\NDS[i+1].m - \NDS[i-1].m)/(f_m^{max} - f_m^{min})$\;	
				}
			}
			
		}
		
	\end{algorithm}
	\caption[Pseudocódigo de la función de asignación de densidad]{Pseudocódigo de la función de asignación de densidad~\cite{Deb2002}}
	\label{fig:CrowindDistance}
\end{figure}
\end{document}