\section{Requisitos}

Durante la fase de requisitos se llevo a cabo la captura, priorizaci�n y la especificaci�n formal de requisitos.

Los requisitos iniciales de la aplicaci�n fueron capturados a partir de una reuni�n con el profesor Jimmy Gutierrez. Posteriormente, estos requisitos fueron priorizados para finalmente ser documentados en el documento de especificaci�n formal de requisitos. 

A medida que avanzaban las iteraci�nes algunos requisitos fueron cambiando o fueron surgiendo requisitos nuevos. En el Cuadro~\ref{fig:cambios_requisitos} se detallan los cambios y actividades realizados durante cada iteraci�n.

\begin{table}[H]
  \begin{center}
    \caption{Actividades y cambios de la fase de requisitos durante cada iteraci�n}
    \bigskip
    \begin{tabular}{||c| m{2cm}| m{8cm}||} 
      \hline
      \textbf{N� Iteraci�n} & \textbf{Requisitos cubiertos} & \textbf{Tareas}\\ [0.5ex] 
      \hline\hline
      1 & 5/32 & Se capturan 5 requisitos.\newline Se crea el informe de especificaci�n de requisitos. \\
      \hline
      2 & 11/32 & Se capturan 6 nuevos requisitos. \newline Se actualiza el documento de requisitos.\\
      \hline
      3 & 16/32 & Se capturan 5 nuevos requisitos. \newline Se actualiza el documento de requisitos. \\
      \hline
      4 & 20/32 & Se capturan 4 nuevos requisitos.\newline Se actualiza el documento de requisitos. \\
      \hline
      5 & 20/32 & No hay nuevos requisitos. \\
      \hline
      6 & 32/32 & Se capturan 12 nuevos requisitos.\newline Se actualiza el documento de requisitos. \\
      \hline

    \end{tabular}
   \label{fig:cambios_requisitos}
 \end{center}
\end{table}

Los cuadros~\ref{fig:RU004},~\ref{fig:RU002} y~\ref{fig:RU020} muestran la especificaci�n formal de los requisitos relacionados a las funcionalidades escogidas anteriormente.

\begin{requisito}[fig:RU004][Especificaci�n del requisito de usuario RU004.]
  \Requisito{RU004}{Visualizar red en una interfaz gr�fica.}
  \Descripcion{Se debe mostrar en la interfaz gr�fica una representaci�n de la red (Un dibujo, etc) generada a partir de la informaci�n contenida en el archivo inp.}
  
  \Fuente{Jimmy Guti�rrez}
  \Prioridad{Moderada}
  \Estabilidad{Intransable}
  \FechaA{09/09/2019}
  \Estado{Cumple}
  \Incremento{3}
  \Tipo{Funcional}
\end{requisito}

\begin{requisito}[fig:RU002][Especificaci�n del requisito de usuario RU002.]
  \Requisito{RU002}{Resolver el problema monoobjetivo (\textit{Pipe Optimizing})  usando un Algoritmo Gen�tico.}
  \Descripcion{El Algoritmo Gen�tico debe ser aplicado para resolver el problema monoobjetivo que tiene como funci�n objetivo el costo de inversi�n y como variable de decisi�n el di�metro de las tuber�as.}
  
  \Fuente{Jimmy Guti�rrez}
  \Prioridad{Alta}
  \Estabilidad{Intransable}
  \FechaA{09/09/2019}
  \Estado{Cumple}
  \Incremento{2}
  \Tipo{Funcional}
\end{requisito}

\begin{requisito}[fig:RU020][Especificaci�n del requisito de usuario RU020.]
  \Requisito{RU020}{Permitir realizar simulaciones hidr�ulicas utilizando los valores por defectos que vienen en el archivo inp y visualizar los resultados.}
    \Descripcion{Utilizando los valores que vienen por defecto en el archivo inp se debe poder llevar a cabo la simulaci�n hidr�ulica de la red. Posteriormente, los resultados podr�n ser visualizados por el usuario.}
    
    \Fuente{Daniel Mora-Meli�}
    \Prioridad{Alta}
    \Estabilidad{Intransable}
    \FechaA{27/01/2020}
    \Estado{Cumple}
    \Incremento{5}
    \Tipo{Funcional}
\end{requisito}

La especificaci�n formal de los requisitos restantes se encuentra en el \textbf{Anexo~\ref{appendix:requisito}}.





