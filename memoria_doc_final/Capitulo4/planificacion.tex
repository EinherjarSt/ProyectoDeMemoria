\section{Planificaci�n}

Como se menciono en el cap�tulo anterior, para llevar a cabo el desarrollo del proyecto se opto por la metodolog�a iterativa e incremental. La planificaci�n resultante seguida durante el desarrollo del proyecto se muestra en el Cuadro~\ref{fig:planificacion}.

\begin{table}[H]
  \begin{center}
    \caption{Planificaci�n de las iteraciones}
    \begin{tabular}{||c| m{6cm}| m{2.5cm}| m{2.5cm}||} 
      \hline
      N� Iteraci�n & Tareas & Fecha Inicio & Fecha \newline t�rmino \\ [0.5ex] 
      \hline\hline
      1 & - Especificaci�n de requisitos \newline - Escoger arquitectura l�gica y f�sica & 26/08/2019 & 14/10/2019 \\ 
      \hline
      2 & - Implementar problema monoobjetivo (Pipe Optimizing) \newline - Implementar Algoritmo Gen�tico \newline - Implementar operadores de selecci�n y reproducci�n & 14/10/2019 & 11/11/2019 \\
      \hline
      3 & - Crear interfaces de usuario \newline - Guardar soluciones & 11/11/2019 & 20/01/2020 \\
      \hline
      4 & - Implementar problema multiobjetivo (Pumping Schedule) \newline - Implementar algoritmo NSGAII & 27/01/2020 & 24/02/2020 \\
      \hline
      5 & - Permitir realizar m�ltiples repeticiones de un algoritmo \newline - Permitir realizar la simulaci�n usando los valores por defecto de la red & 02/03/2020 & 04/05/2020 \\ [1ex] 
      \hline
      6 & - Agregar s�mbolos al dibujo de la red \newline - Exportar a excel \newline - Agregar menu de configuraci�n & 11/05/2020 & 08/06/2020 \\ [1ex] 
      \hline
    \end{tabular}
   \label{fig:planificacion}
 \end{center}
\end{table}