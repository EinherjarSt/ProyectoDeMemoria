\section{Alcance del proyecto}

Al final del per�odo de desarrollo la herramienta debe contar con las siguientes prestaciones.
\begin{itemize}
    \item El sistema permite la carga y la visualizaci�n de la red gr�ficamente.
    \item El sistema implementa por defecto dos algoritmos. Uno de ellos monoobjetivo y el otro multiobjetivo. El algoritmo monoobjetivo corresponde al Algoritmo Gen�tico (GA) mientras que el multiobjetivo a \textit{Non-Dominated Sorting Genetic Algorithm II} (NSGAII).
	\item El sistema proporciona dos problemas ya implementados uno monoobjetivo y el otro multiobjetivo. El Algoritmo Gen�tico se utiliza para generar soluciones para el problema monoobjetivo con el fin de optimizar el costo de inversi�n de las tuber�as. Por otro lado, el algoritmo NSGAII aborda el problema multiobjetivo con el objetivo de optimizar los costos energ�ticos y los costos de mantenimiento de los equipos de bombeo.
    \item El sistema permite visualizar y guardar las soluciones de los algoritmos en un archivo.
    \item El sistema permite que el usuario agregue nuevos algoritmos, operadores o problemas sin tener que modificar la interfaz de usuario.
\end{itemize}

Este proyecto no contempla la creaci�n de la red por lo que estas deber�n ser ingresadas como entradas al programa. La creaci�n de las redes puede realizarse a trav�s de EPANET. Adem�s, esta herramienta �nicamente podr� ser ocupada en equipos cuyo sistema operativo sea Windows de 64bits debido a que se realizan llamadas a librer�as nativas las cuales no son multiplataforma. Adicionalmente, el equipo debe contar con Java 1.8.
