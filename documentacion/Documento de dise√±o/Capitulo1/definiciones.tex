\section{Definiciones, siglas y abreviaturas}

\begin{description}
    \item [\textit{Metaheuristicas}:] Algoritmos que permiten resolver un amplio rango de problemas de optimizaci�n empleando t�cnicas con alg�n grado de aleatoriedad para encontrar soluciones a un problema. Estos algoritmos no garantizan que la soluci�n encontrada sea la �ptima, pero permiten obtener generalmente aproximaciones a esta~\cite{Yang2015, Boussaid2013, Luke2013}.
    \item [\textit{Genetic Algorithm (GA)}:] Estrategia de b�squeda de soluciones basada en la teor�a de la evoluci�n de Darwin. Para realizar esto, el algoritmo  parte desde un conjunto de soluciones denominada poblaci�n he iterativamente, lleva a cabo un proceso de reproducci�n, generando nuevas soluciones~\cite{Heiss-Czedik1997}.
    \item [\textit{Non-dominated Sorting Genetic Algorithm (NSGAII)}:] Algoritmo que utiliza el cruzamiento, mutaci�n y reproducci�n para encontrar un conjunto de soluciones �ptimas a problemas que cuentan con m�s de un objetivo~\cite{Deb2002}.
    \item [\textit{Java Reflection}:] Caracter�stica de java que permite que un programa se auto examine. Esta caracter�stica est� disponible a trav�s de la \textit{Java Reflection API}, la cual cuenta con m�todos para obtener los \textit{meta-object} de las clases, m�todos, constructores, campos o par�metros. Esta API tambi�n permite crear nuevos objetos cuyo tipo era desconocido al momento de compilar el programa~\cite{Braux1999}.
    \item [\textit{Java Annotation}:] Caracter�stica de java para agregar metadatos a elementos de java (clases, m�todos, par�metros, etc.)~\cite{Rocha2011}. Las anotaciones no tienen efecto directo sobre el c�digo, pero cuando son usadas junto con otras herramientas pueden llegar a ser muy �tiles. Estas herramientas pueden analizar estas anotaciones y realizar acciones en base a estas, por ejemplo, generar archivos adicionales como clases de java, archivos XML, entre otras. Adicionalmente, las anotaciones tambi�n pueden ser analizadas durante la ejecuci�n del programa v�a \textit{Java Reflection}, para crear objetos cuyo tipo no conocemos en tiempo de compilaci�n. 
    \item [\textit{GUI}:] Graphical User Interface.

\end{description}
