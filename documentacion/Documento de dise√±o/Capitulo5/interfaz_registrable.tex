\section{Interfaz Registrable}

Esta interfaz declara el m�todo \textit{build} y \textit{getParameters}. La declaraci�n del m�todo \textit{build}  corresponde a la siguiente:

\begin{lstlisting}
    R build(String inpPath) throws Exception;
\end{lstlisting}

Donde R corresponde al tipo de valor devuelto por la funci�n.

De esta clase se heredan dos subinterfaces. La primera corresponde a \textit{SingleObjectiveRegistrable} que debe ser usada para implementar los problemas monoobjetivos. En cuanto a la segunda, esta corresponde a \textit{MultiObjectiveRegistrable}, la cual se debe usar para los problemas multiobjetivos. El m�todo build sobreescrito por estas clases tiene la siguiente asignatura: 

\begin{lstlisting}
    Experiment<?> build(String inpPath) throws Exception;
\end{lstlisting}

\noindent %% remueve la identaci�n
donde Experiment consiste en una clase que almacena una lista de algoritmos (Del mismo tipo) y el problema a resolver por estos algoritmos.

Las nuevas clases deben implementar la interfaz \textit{SingleObjectiveRegistrable} o \textit{MultiObjectiveRegistrable}, dependiendo del tipo de problema a tratar. Las clases que implementen cualquiera de estas dos interfaces deben ser guardados en una estructura de datos, la cual ser� recorrida cuando se inicie la ejecuci�n del programa y analizada usando la \textit{Java Reflection API}. Este an�lisis consistir� en escanear y validar el cumplimiento de la convenci�n establecida para las clases que implementan est� interfaz. Esta convenci�n consiste en lo siguiente:

\begin{itemize}
    \item La clase debe contener un �nico constructor que use la anotaci�n \textit{@NewProblem.}
    \item Si el constructor requiere par�metros �stos deben estar descritos usando la anotaci�n \textit{@Parameters.}
    \item El constructor debe declarar los par�metros en el siguiente orden, de acuerdo con su tipo.
    \begin{enumerate}
        \item Object: Usado para inyectar los operadores. �stos pueden posteriormente ser casteados a su tipo correcto. La anotaci�n correspondiente es \textit{@OperatorInput}
        \item File: Usados cuando el problema requiere informaci�n adicional que se encuentra en un archivo diferente. La anotaci�n correspondiente es \textit{@FileInput}
        \item int, Integer, double o Double: Usado generalmente para configurar valores en el algoritmo o si el problema requiere otros valores que no fueron solicitados al crear los operadores. Las anotaciones correspondientes son \textit{@NumberInput y @NumberToggleInput.}
        \item El constructor debe solicitar la misma cantidad de par�metros que las descritas en la anotaci�n \textit{@Parameters.}
    \end{enumerate}
\end{itemize}

Si estas convenciones no se cumplen, entonces un error en tiempo de compilaci�n ser� emitido como se mencion� anteriormente en la secci�n anterior.

El orden en el que son inyectados los par�metros consiste en el siguiente:

\begin{enumerate}
    \item Par�metros descritos por \textit{@OperatorInput}
    \item Par�metros descritos por \textit{@FileInput}
    \item Par�metros descritos por \textit{@NumberInput}
    \item Par�metros descritos por \textit{@NumberToggleInput}
\end{enumerate}

Una vez que se haya configurado el problema a trav�s de la interfaz se crear� la instancia de la clase que hereda de Registrable y se llamar� a su m�todo build, para crear el experimento y comenzar su ejecuci�n.

La estructura de datos para registrar las clases que heredan de \textit{SingleObjectiveRegistrable} y \textit{MultiObjectiveRegistrable} se encontrar� en la clase \textit{RegistrableConfiguration}.
