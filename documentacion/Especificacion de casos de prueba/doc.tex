%% inicio, la clase del documento es iccmemoria.cls
\documentclass{especificacion}
\usepackage[onelanguage, commentsnumbered, linesnumbered, boxed, ruled]{algorithm2e}
\usepackage[draft=false,kerning=true]{microtype}
\usepackage{pdfpages}
\usepackage{amsfonts}
\usepackage{amssymb}
\usepackage{array}
\usepackage{longtable}
\usepackage{float}
\usepackage{especificacionutility}
\usepackage{setspace}
\usepackage{adjustbox}
\usepackage{xcolor}
\usepackage{listings}

\colorlet{punct}{red!60!black}
\definecolor{background}{HTML}{EEEEEE}
\definecolor{delim}{RGB}{20,105,176}
\colorlet{numb}{magenta!60!black}

\lstdefinelanguage{json}{
    basicstyle=\normalfont\ttfamily,
    breaklines=true,
}

%% Declare counter to TestID
\newcounter{testcase}[chapter]
%% Command to access the counter print in ATXXX format (Automatic test)
\newcommand{\atcounter}{
    AT%
    \refstepcounter{testcase}%
    \ifnum\value{testcase}<100 0\fi%
    \ifnum\value{testcase}<10 0\fi%
    \arabic{testcase}%
}
%% Command to access the counter print in MTXXX format (Manual test)
\newcommand{\mtcounter}{
    MT%
    \refstepcounter{testcase}%
    \ifnum\value{testcase}<100 0\fi%
    \ifnum\value{testcase}<10 0\fi%
    \arabic{testcase}%
}

\newcommand{\vb}{$\mid$} %command to print |


%% datos generales y para la tapa
\titulo{Documento de especificaci�n de casos de prueba}
\subtitulo{Herramienta para la optimizaci�n de redes de distribuci�n de agua potable}
\author{Gabriel Gonzalo Alexander Sanhueza Fuentes}


%% inicio de documento
\begin{document}

%% crea la tapa
\maketitle

\adddevelopment{Gabriel Sanhueza Fuentes}{Administrador, Analista, Dise�ador, Implementador y Tester.}{gsanhueza15@alumnos~.utalca.cl}

\addcounterpart{Jimmy H. Guti�rrez-Bahamondes}{Cliente/Profesor gu�a}{}
\addcounterpart{Jimmy H. Guti�rrez-Bahamondes}{Cliente/Profesor co-gu�a}{}

\addrevision{0.1}{07/09/2019}{}
\addrevision{1}{15/06/2020}{}


% \RevisionHistoryPage


%% indices
\tableofcontents
%\listoffigures
%\listoftables

%% abstract

%WRITE HERE
\chapter{Casos de pruebas automatizadas}

En este cap�tulo se presenta la especificaci�n formal de los casos de prueba automatizados. Los casos de prueba automatizados son aquellos que se realizan utilizando herramientas de automatizaci�n como el framework JUnit y que no necesitan interacci�n con un usuario durante su ejecuci�n.

\section{\textit{Pump}}
En esta secci�n se presentan las pruebas realizadas para la clase \textit{Pump}.

\begin{prueba}
    \TestID{\atcounter}
    \Titulo{Env�o de par�metro inv�lido al m�todo setProperty de la clase \textit{Pump} cuando se usa la clave \textit{PumpProperty.HEAD}.}
    \Caracteristica{Validar par�metro recibido por el m�todo \textit{setProperty}.}
    \Objetivo{Comprobar que al pasar un par�metro como valor de un tipo distinto a \textit{String}, cuando se usa la clave \textit{PumpProperty.HEAD}, se lanza una excepci�n.}
    \Configuracion{Instancia de la clase \textit{Pump} inicializada.}
    \DatosPrueba{%
    Clave: \textit{PumpProperty.HEAD}\\
    Valor: Un objeto diferente a un \textit{String}
    }
    \AccionesPrueba{%
        1. Inicializar instancia.\\
        2. Pasar una instancia de un objeto distinto de un \textit{String} al m�todo cuando se usa la llave \textit{PumpProperty.HEAD}.
    }
    \ResultadosEsperados{Excepci�n indicando que el tipo de instancia pasada no es v�lida.}
\end{prueba}

\begin{prueba}
    \TestID{\atcounter}
    \Titulo{Env�o de par�metro inv�lido al m�todo setProperty de la clase \textit{Pump} cuando se usa la clave \textit{PumpProperty.PATTERN}.}
    \Caracteristica{Validar par�metro recibido por el m�todo \textit{setProperty}.}
    \Objetivo{Comprobar que si se pasa un par�metro como valor de un tipo distinto a \textit{String}, cuando se usa la clave \textit{PumpProperty.PATTERN}, se lanza una excepci�n.}
    \Configuracion{Instancia de la clase \textit{Pump} inicializada.}
    \DatosPrueba{%
    Clave: \textit{PumpProperty.PATTERN}\\
    Valor: Un objeto diferente a un \textit{String}
    }
    \AccionesPrueba{%
        1. Inicializar instancia.\\
        2. Pasar una instancia de un objeto distinto de un \textit{String} al m�todo cuando se usa la llave \textit{PumpProperty.HEAD}.
    }
    \ResultadosEsperados{Excepci�n indicando que el tipo de instancia pasada no es v�lida.}
\end{prueba}

\begin{prueba}
    \TestID{\atcounter}
    \Titulo{Env�o de par�metro inv�lido al m�todo setProperty de la clase \textit{Pump} cuando se usa la clave \textit{PumpProperty.SPEED}.}
    \Caracteristica{Validar par�metro recibido por el m�todo \textit{setProperty}.}
    \Objetivo{Comprobar que si se pasa un par�metro como valor de un tipo distinto a \textit{Double}, cuando se usa la clave \textit{PumpProperty.SPEED}, el m�todo lanza una excepci�n.}
    \Configuracion{Instancia de la clase \textit{Pump} inicializada.}
    \DatosPrueba{%
    Clave: \textit{PumpProperty.SPEED}\\
    Valor: Un \textit{Integer} o alguna instancia de otro objeto distinto de Double.
    }
    \AccionesPrueba{%
        1. Inicializar instancia.\\
        2. Pasar una instancia de un objeto distinto de un \textit{Double} al m�todo cuando se usa la llave \textit{PumpProperty.SPEED}.
    }
    \ResultadosEsperados{Excepci�n indicando que el tipo de instancia pasada no es v�lida.}
\end{prueba}

\begin{prueba}
    \TestID{\atcounter}
    \Titulo{Env�o de par�metro inv�lido al m�todo setProperty de la clase \textit{Pump} cuando se usa la clave \textit{PumpProperty.POWER}.}
    \Caracteristica{Validar par�metro recibido por el m�todo \textit{setProperty}.}
    \Objetivo{Comprobar que si se pasa un par�metro como valor de un tipo distinto a \textit{Double}, cuando se usa la clave \textit{PumpProperty.POWER}, el m�todo lanza una excepci�n.}
    \Configuracion{Instancia de la clase \textit{Pump} inicializada.}
    \DatosPrueba{%
    Clave: \textit{PumpProperty.POWER}\\
    Valor: Un \textit{Integer} o alguna instancia de otro objeto distinto de Double.
    }
    \AccionesPrueba{%
        1. Inicializar instancia.\\
        2. Pasar una instancia de un objeto distinto de un \textit{Double} al m�todo cuando se usa la llave \textit{PumpProperty.POWER}.
    }
    \ResultadosEsperados{Excepci�n indicando que el tipo de instancia pasada no es v�lida.}
\end{prueba}

%

\begin{prueba}
    \TestID{\atcounter}
    \Titulo{Env�o de par�metro v�lido al m�todo setProperty de la clase \textit{Pump} cuando se usa la clave \textit{PumpProperty.HEAD}.}
    \Caracteristica{Validar par�metro recibido por el m�todo \textit{setProperty}.}
    \Objetivo{Comprobar que si se pasa un par�metro de tipo \textit{String} como valor, cuando se usa la clave \textit{PumpProperty.HEAD}, el m�todo finaliza sin error.}
    \Configuracion{Instancia de la clase \textit{Pump} inicializada.}
    \DatosPrueba{%
    Clave: \textit{PumpProperty.HEAD}\\
    Valor: Un \textit{String} no vac�o.
    }
    \AccionesPrueba{%
        1. Inicializar instancia.\\
        2. Pasar una instancia de un \textit{String} al m�todo cuando se usa la llave \textit{PumpProperty.HEAD}.
    }
    \ResultadosEsperados{M�todo ejecutado sin errores.}
\end{prueba}

\begin{prueba}
    \TestID{\atcounter}
    \Titulo{Env�o de par�metro v�lido al m�todo setProperty de la clase \textit{Pump} cuando se usa la clave \textit{PumpProperty.PATTERN}.}
    \Caracteristica{Validar par�metro recibido por el m�todo \textit{setProperty}.}
    \Objetivo{Comprobar que si se pasa un par�metro de tipo \textit{String} como valor, cuando se usa la clave \textit{PumpProperty.PATTERN}, el m�todo finaliza sin error.}
    \Configuracion{Instancia de la clase \textit{Pump} inicializada.}
    \DatosPrueba{%
    Clave: \textit{PumpProperty.PATTERN}\\
    Valor: Un \textit{String} no vac�o.
    }
    \AccionesPrueba{%
        1. Inicializar instancia.\\
        2. Pasar una instancia de un \textit{String} al m�todo cuando se usa la llave \textit{PumpProperty.HEAD}.
    }
    \ResultadosEsperados{M�todo ejecutado sin errores.}
\end{prueba}

\begin{prueba}
    \TestID{\atcounter}
    \Titulo{Env�o de par�metro v�lido al m�todo setProperty de la clase \textit{Pump} cuando se usa la clave \textit{PumpProperty.SPEED}.}
    \Caracteristica{Validar par�metro recibido por el m�todo \textit{setProperty}.}
    \Objetivo{Comprobar que si se pasa un par�metro de tipo \textit{Double} como valor, cuando se usa la clave \textit{PumpProperty.SPEED}, el m�todo finaliza sin error.}
    \Configuracion{Instancia de la clase \textit{Pump} inicializada.}
    \DatosPrueba{%
    Clave: \textit{PumpProperty.SPEED}\\
    Valor: Un valor \textit{Double}.
    }
    \AccionesPrueba{%
        1. Inicializar instancia.\\
        2. Pasar una instancia de un \textit{Double} al m�todo cuando se usa la llave \textit{PumpProperty.SPEED}.
    }
    \ResultadosEsperados{M�todo ejecutado sin errores.}
\end{prueba}

\begin{prueba}
    \TestID{\atcounter}
    \Titulo{Env�o de par�metro v�lido al m�todo setProperty de la clase \textit{Pump} cuando se usa la clave \textit{PumpProperty.POWER}.}
    \Caracteristica{Validar par�metro recibido por el m�todo \textit{setProperty}.}
    \Objetivo{Comprobar que si se pasa un par�metro de tipo \textit{Double} como valor, cuando se usa la clave \textit{PumpProperty.POWER}, el m�todo finaliza sin error.}
    \Configuracion{Instancia de la clase \textit{Pump} inicializada.}
    \DatosPrueba{%
    Clave: \textit{PumpProperty.POWER}\\
    Valor: Un valor \textit{Double}.
    }
    \AccionesPrueba{%
        1. Inicializar instancia.\\
        2. Pasar una instancia de un \textit{Double} al m�todo cuando se usa la llave \textit{PumpProperty.SPEED}.
    }
    \ResultadosEsperados{M�todo ejecutado sin errores.}

\end{prueba}

\section{\textit{GeneticAlgorithm}}
En esta secci�n se especifican las pruebas realizadas para la clase \textit{GeneticAlgorithm}.

\begin{prueba}
    \TestID{\atcounter}
    \Titulo{N�mero m�ximo de evaluaciones no v�lido.}
    \Caracteristica{Validar par�metro \textit{maxEvaluations}.}
    \Objetivo{Validar que el par�metro \textit{maxEvaluations} no sea negativo. Si el par�metro es negativo debe lanzarse una excepci�n.}
    \Configuracion{Instancia de la clase \textit{GeneticAlgorithm} inicializada.}
    \DatosPrueba{%
        Cualquier entero menor que 0.
    }
    \AccionesPrueba{%
        Llamar al m�todo \textit{setMaxEvaluations} con un argumento negativo.
    }
    \ResultadosEsperados{Una excepci�n.}
\end{prueba}

\begin{prueba}
    \TestID{\atcounter}
    \Titulo{N�mero m�ximo de evaluaciones sin mejora no v�lido.}
    \Caracteristica{Validar par�metro \textit{maxNumberOfEvaluationsWithoutImprovement}.}
    \Objetivo{Validar que el par�metro \textit{maxNumberOfEvaluationsWithoutImprovement} no sea negativo. Si el par�metro es negativo debe lanzarse una excepci�n.}
    \Configuracion{Instancia de la clase \textit{GeneticAlgorithm} inicializada.}
    \DatosPrueba{%
        Cualquier entero menor que 0.
    }
    \AccionesPrueba{%
        Llamar al m�todo \textit{setMaxNumberOfEvaluationsWithoutImprovement} con un argumento negativo.
    }
    \ResultadosEsperados{Una excepci�n.}
\end{prueba}

\begin{prueba}
    \TestID{\atcounter}
    \Titulo{Deshabilitar n�mero m�ximo de evaluaciones sin mejoras cuando se modifica el n�mero m�ximo de evaluaciones.}
    \Caracteristica{Validar par�metro \textit{maxEvaluations}.}
    \Objetivo{Validar que al modificar el par�metro \textit{maxEvaluations} el par�metro \textit{maxNumberOfEvaluationsWithoutImprovement} cambie a 0 (0 indica que esta deshabilitado).}
    \Configuracion{Instancia de la clase \textit{GeneticAlgorithm} inicializada.}
    \DatosPrueba{%
        Cualquier entero positivo.
    }
    \AccionesPrueba{%
        Llamar al m�todo \textit{setMaxEvaluations} con un entero positivo mayor a 0.
    }
    \ResultadosEsperados{Par�metro \textit{maxNumberOfEvaluationsWithoutImprovement} igual a 0.}
\end{prueba}

\begin{prueba}
    \TestID{\atcounter}
    \Titulo{Deshabilitar n�mero m�ximo de evaluaciones cuando se modifica el n�mero m�ximo de evaluaciones sin mejoras.}
    \Caracteristica{Validar par�metro \textit{maxNumberOfEvaluationsWithoutImprovement}.}
    \Objetivo{Validar que al modificar el par�metro \textit{maxNumberOfEvaluationsWithoutImprovement} el par�metro \textit{maxNumberOfEvaluationsWithoutImprovement} cambie a 0 (0 indica que esta deshabilitado).}
    \Configuracion{Instancia de la clase \textit{GeneticAlgorithm} inicializada.}
    \DatosPrueba{%
        Cualquier entero positivo.
    }
    \AccionesPrueba{%
        Llamar al m�todo \textit{setMaxNumberOfEvaluationsWithoutImprovement} con un entero positivo mayor a 0.
    }
    \ResultadosEsperados{Par�metro \textit{maxEvaluations} igual a 0.}
\end{prueba}

\section{\textit{IntegerRangeRandomMutation}}
En esta secci�n se especifican las pruebas realizadas sobre la clase \textit{IntegerRangeRandomMutation}.

\begin{prueba}
    \TestID{\atcounter}
    \Titulo{Probabilidad de mutaci�n no v�lida.}
    \Caracteristica{Validar par�metro \textit{mutationProbability}.}
    \Objetivo{Validar que el par�metro \textit{mutationProbability} no sea negativo. Si el par�metro es negativo debe lanzarse una excepci�n.}
    \Configuracion{Instancia de la clase \textit{IntegerRangeRandomMutation} inicializada.}
    \DatosPrueba{%
        Cualquier entero menor que 0.
    }
    \AccionesPrueba{%
        Crear instancia de la clase \textit{IntegerRangeRandomMutation} pasando como argumento para el par�metro \textit{mutationProbability} un valor negativo.
    }
    \ResultadosEsperados{Una excepci�n.}
\end{prueba}

\begin{prueba}
    \TestID{\atcounter}
    \Titulo{Rango no v�lido.}
    \Caracteristica{Validar par�metro \textit{range}.}
    \Objetivo{Validar que el par�metro \textit{range} no sea negativo. Si el par�metro es negativo debe lanzarse una excepci�n.}
    \Configuracion{Instancia de la clase \textit{IntegerRangeRandomMutation} inicializada.}
    \DatosPrueba{%
        Cualquier entero menor que 0.
    }
    \AccionesPrueba{%
        Crear instancia de la clase \textit{IntegerRangeRandomMutation} pasando como argumento para el par�metro \textit{range} un valor negativo.
    }
    \ResultadosEsperados{Una excepci�n.}
\end{prueba}

\begin{prueba}
    \TestID{\atcounter}
    \Titulo{Mutar variables.}
    \Caracteristica{Mutar variables cuando el n�mero generado por el \textit{randomGenerator} es menor que la probabilidad de mutaci�n.}
    \Objetivo{Validar que la mutaci�n suceda cuando un n�mero generado aleatoriamente sea menor que la probabilidad de mutaci�n.}
    \Configuracion{%
    Solucion con variables preestablecidas.\\
    Operador \textit{IntegerRangeRandomMutation} inicializado.
    }
    \DatosPrueba{%
        \textit{mutationProbability}: 0.3\\
        \textit{range}: 2\\
        \textit{solution}: [0, 2, 4]\\
        \textit{randomGenerator}: [0.2, 0.2, 0.4]\\
        \textit{boundedRandomGenerator}: [2, 0, 5]
    }
    \AccionesPrueba{%
        Pasar la soluci�n al m�todo \textit{execute}.
    }
    \ResultadosEsperados{Las variables en la soluci�n despues de realizar la mutaci�n deben ser [2, 0, 4] }
\end{prueba}

\begin{prueba}
    \TestID{\atcounter}
    \Titulo{No mutar variables.}
    \Caracteristica{No mutar variables cuando los n�meros generado por el \textit{randomGenerator} sean mayores que la probabilidad de mutaci�n.}
    \Objetivo{Validar que la mutaci�n no sucede si el \textit{randomGenerator} devuelve valores mayores a \textit{mutationProbability}.}
    \Configuracion{%
    Solucion con variables preestablecidas.\\
    Operador \textit{IntegerRangeRandomMutation} inicializado.
    }
    \DatosPrueba{%
        \textit{mutationProbability}: 0.3\\
        \textit{range}: 2\\
        \textit{solution}: [0, 2, 4]\\
        \textit{randomGenerator}: [0.5, 0.4, 0.4]\\
        \textit{boundedRandomGenerator}: [2, 0, 5]
    }
    \AccionesPrueba{%
        Pasar la soluci�n al m�todo \textit{execute}.
    }
    \ResultadosEsperados{Las variables en la soluci�n despues de realizar la llamada al m�todo \textit{execute} deben ser los valores originales [0, 2, 4].}
\end{prueba}

\begin{prueba}
    \TestID{\atcounter}
    \Titulo{No mutar variables cuando no hay un rango de mutaci�n.}
    \Caracteristica{No mutar variables cuando el rango de las variables a generar es 0.}
    \Objetivo{Validar que la mutaci�n no sucede si \textit{range} tiene asignado el valor 0.}
    \Configuracion{%
    Solucion con variables preestablecidas.\\
    Operador \textit{IntegerRangeRandomMutation} inicializado.
    }
    \DatosPrueba{%
        \textit{mutationProbability}: 0.3\\
        \textit{range}: 0\\
        \textit{solution}: [0, 2, 4]\\
        \textit{randomGenerator}: [0.1, 0.2, 0.3]
    }
    \AccionesPrueba{%
        Pasar la soluci�n al m�todo \textit{execute}.
    }
    \ResultadosEsperados{Las variables en la soluci�n despues de realizar la llamada al m�todo \textit{execute} deben ser los valores originales [0, 2, 4].}
\end{prueba}

\section{\textit{ReflectionUtils}}
En esta secci�n se especifican las pruebas realizadas sobre la clase \textit{ReflectionUtils}.

Durante esta secci�n nos referiremos a cualquier implementaci�n de las interfaz \textit{Registrable} o sus subinterfaces unicamente como \textit{Registrable}. Es por ello, que cuando se mencione implementar \textit{Registrable} se refiere a implementar ya sea \textit{SingleObjectiveRegistrable} o \textit{MultiobjectiveRegistrable}.

\begin{prueba}
    \TestID{\atcounter}
    \Titulo{Nombre del problema en anotaci�n \textit{@NewProblem}.}
    \Caracteristica{Obtener nombre del problema de la anotaci�n \textit{@NewProblem}.}
    \Objetivo{Validar que el m�todo \textit{getNameOfProblem} retorna el nombre asignado en la anotaci�n \textit{@NewProblem}.}
    \Configuracion{%
        Implementaci�n de \textit{Registrable} con la anotaci�n \textit{@NewProblem} en su constructor.
    }
    \DatosPrueba{%
        Nombre del problema: ``Test''
    }
    \AccionesPrueba{%
        Pasar el objeto \textit{Class$<$?$>$} que hace referencia al tipo \textit{Registrable} al m�todo \textit{getNameOfProblem}.
    }
    \ResultadosEsperados{``Test''}
\end{prueba}

\begin{prueba}
    \TestID{\atcounter}
    \Titulo{Nombre del algoritmo en anotaci�n \textit{@NewProblem}.}
    \Caracteristica{Obtener nombre del algoritmo de la anotaci�n \textit{@NewProblem}.}
    \Objetivo{Validar que el m�todo \textit{getNameOfAlgorithm} retorna el nombre asignado en la anotaci�n \textit{@NewProblem}.}
    \Configuracion{%
        Implementaci�n \textit{Registrable} con la anotaci�n \textit{@NewProblem} en su constructor.
    }
    \DatosPrueba{%
        Nombre del algoritmo: "NSGAII"
    }
    \AccionesPrueba{%
        Pasar el objeto \textit{Class$<$?$>$} que hace referencia al tipo \textit{Registrable} al m�todo \textit{getNameOfAlgorithm}.
    }
    \ResultadosEsperados{"NSGAII"}
\end{prueba}

\begin{prueba}
    \TestID{\atcounter}
    \Titulo{\textit{Registrable} sin anotaciones.}
    \Caracteristica{Validar las anotaciones y el tipo de par�metros utilizados en el constructor de la clase \textit{Registrable}.}
    \Objetivo{Validar que si en el constructor p�blico no tiene ni la anotaci�n \textit{@NewProblem} ni la anotaci�n \textit{@Parameters} se lanza una excepci�n.}
    \Configuracion{%
        Implementaci�n de \textit{Registrable} sin anotaciones.
    }
    \DatosPrueba{%
        Objeto \textit{Class$<$?$>$} que referencia al tipo \textit{Registrable}.
    }
    \AccionesPrueba{%
        Pasar el objeto \textit{Class} que hace referencia al tipo \textit{Registrable} al m�todo \textit{validateRegistrableProblem}.
    }
    \ResultadosEsperados{Una excepci�n.}
\end{prueba}

\begin{prueba}
    \TestID{\atcounter}
    \Titulo{\textit{Registrable} con la anotaci�n \textit{@NewProblem}.}
    \Caracteristica{Validar las anotaciones y el tipo de par�metros utilizados en el constructor de la clase \textit{Registrable}.}
    \Objetivo{Validar si en el constructor p�blico est� la anotaci�n \textit{@NewProblem}}
    \Configuracion{%
        Implementaci�n de \textit{Registrable} con la anotaci�n \textit{@NewProblem}.
    }
    \DatosPrueba{%
        Objeto \textit{Class$<$?$>$} que referencia al tipo \textit{Registrable}.
    }
    \AccionesPrueba{%
        Pasar el objeto \textit{Class} que hace referencia al tipo \textit{Registrable} al m�todo \textit{validateRegistrableProblem}.
    }
    \ResultadosEsperados{M�todo ejecutado sin errores.}
\end{prueba}

\begin{prueba}
    \TestID{\atcounter}
    \Titulo{\textit{Registrable} cuyo constructor recibe los par�metros en el orden correcto dependiendo de su tipo.}
    \Caracteristica{Validar las anotaciones y el tipo de par�metros utilizados en el constructor de la clase \textit{Registrable}.}
    \Objetivo{Validar que el constructor p�blico con las anotaciones recibe los par�metros en el siguiente orden: \textit{Object}, \textit{File}, \textit{(int \vb Integer \vb double \vb Double)}.}
    \Configuracion{%
        Implementaci�n de \textit{Registrable} con las anotaciones \textit{@NewProblem} y \textit{@Parameters}, y con los par�metros recibidos en el constructor en el orden esperado.
    }
    \DatosPrueba{%
        Objeto \textit{Class$<$?$>$} que referencia al tipo \textit{Registrable}.
    }
    \AccionesPrueba{%
        Pasar el objeto \textit{Class} que hace referencia al tipo \textit{Registrable} al m�todo \textit{validateRegistrableProblem}.
    }
    \ResultadosEsperados{M�todo ejecutado sin errores.}
\end{prueba}

\begin{prueba}
    \TestID{\atcounter}
    \Titulo{\textit{Registrable} cuyo constructor recibe los par�metros en un orden incorrecto.}
    \Caracteristica{Validar las anotaciones y el tipo de par�metros utilizados en el constructor de la clase \textit{Registrable}.}
    \Objetivo{Validar que el constructor p�blico con las anotaciones recibe los par�metros en un orden distinto a: \textit{Object}, \textit{File}, \textit{(int \vb Integer \vb double \vb Double)}.}
    \Configuracion{%
        Implementaci�n de \textit{Registrable} con las anotaciones \textit{@NewProblem} y \textit{@Parameters}, y con los par�metros recibidos en el constructor en un orden distinto al especificado.
    }
    \DatosPrueba{%
        Objeto \textit{Class$<$?$>$} que referencia al tipo \textit{Registrable}.
    }
    \AccionesPrueba{%
        Pasar el objeto \textit{Class} que hace referencia al tipo \textit{Registrable} al m�todo \textit{validateRegistrableProblem}.
    }
    \ResultadosEsperados{Una excepci�n.}
\end{prueba}

\begin{prueba}
    \TestID{\atcounter}
    \Titulo{\textit{Registrable} cuyo constructor recibe una cantidad de par�metros distintas a la esperada de acuerdo a la anotaci�n \textit{@Parameters.}}
    \Caracteristica{Validar las anotaciones y el tipo de par�metros utilizados en el constructor de la clase \textit{Registrable}.}
    \Objetivo{Validar que si el constructor p�blico recibe m�s par�metros que los indicados en \textit{@Parameters}, lanza una excepci�n.}
    \Configuracion{%
        Implementaci�n de \textit{Registrable} con las anotaciones \textit{@NewProblem} y \textit{@Parameters}, y con un par�metro extra en el constructor al indicado en la anotaci�n \textit{@Parameters}.
    }
    \DatosPrueba{%
        Objeto \textit{Class$<$?$>$} que referencia al tipo \textit{Registrable}.
    }
    \AccionesPrueba{%
        Pasar el objeto \textit{Class} que hace referencia al tipo \textit{Registrable} al m�todo \textit{validateRegistrableProblem}.
    }
    \ResultadosEsperados{Una excepci�n.}
\end{prueba}

\begin{prueba}
    \TestID{\atcounter}
    \Titulo{\textit{Registrable} con una anotaci�n extra en \textit{@Parameters}.}
    \Caracteristica{Validar las anotaciones y el tipo de par�metros utilizados en el constructor de la clase \textit{Registrable}.}
    \Objetivo{Validar que si \textit{@Parameters} tiene mas anotaciones que el n�mero de par�metros en el constructor, se lanza una excepci�n.}
    \Configuracion{%
        Implementaci�n de \textit{Registrable} con las anotaciones \textit{@NewProblem} y \textit{@Parameters}, �sta �ltima con una anotaci�n extra.
    }
    \DatosPrueba{%
        Objeto \textit{Class$<$?$>$} que referencia al tipo \textit{Registrable}.
    }
    \AccionesPrueba{%
        Pasar el objeto \textit{Class} que hace referencia al tipo \textit{Registrable} al m�todo \textit{validateRegistrableProblem}.
    }
    \ResultadosEsperados{Una excepci�n.}
\end{prueba}

\begin{prueba}
    \TestID{\atcounter}
    \Titulo{\textit{Registrable} con dos constructores.}
    \Caracteristica{Validar las anotaciones y el tipo de par�metros utilizados en el constructor de la clase \textit{Registrable}.}
    \Objetivo{Validar que si \textit{Registrable} tiene dos constructores se lanza una excepci�n.}
    \Configuracion{%
        Implementaci�n de \textit{Registrable} con dos constructor, uno de ellos con la anotaciones \textit{@NewProblem} y \textit{@Parameters}.
    }
    \DatosPrueba{%
        Objeto \textit{Class$<$?$>$} que referencia al tipo \textit{Registrable}.
    }
    \AccionesPrueba{%
        Pasar el objeto \textit{Class} que hace referencia al tipo \textit{Registrable} al m�todo \textit{validateRegistrableProblem}.
    }
    \ResultadosEsperados{Una excepci�n.}
\end{prueba}

\begin{prueba}
    \TestID{\atcounter}
    \Titulo{Id del grupo usado por \textit{@NumberToggleInput} secuencial.}
    \Caracteristica{Validar las anotaciones y el tipo de par�metros utilizados en el constructor de la clase \textit{Registrable}.}
    \Objetivo{Validar que si \textit{Registrable}, en el elemento \textit{numberToggle} de la anotaci�n \textit{@Parameters} recibe las anotaciones \textit{@NumberToggleInput} de manera secuencial no se lanza una excepci�n. Con secuencial se refiere a que \textit{@NumberToggleInput} con el mismo \textit{groupID} deben estar juntos.}
    \Configuracion{%
        Implementaci�n de \textit{Registrable} con las anotaciones \textit{@NewProblem} y \textit{@Parameters}. El elemento \textit{numberToggle} de \textit{@Parameters} tiene las anotaciones \textit{@NumberToggleInput} con el mismo \textit{groupID} juntos.
    }
    \DatosPrueba{%
        Objeto \textit{Class$<$?$>$} que referencia al tipo \textit{Registrable}.
    }
    \AccionesPrueba{%
        Pasar el objeto \textit{Class} que hace referencia al tipo \textit{Registrable} al m�todo \textit{validateRegistrableProblem}.
    }
    \ResultadosEsperados{M�todo termina sin error.}
\end{prueba}

\begin{prueba}
    \TestID{\atcounter}
    \Titulo{Id del grupo usado por \textit{@NumberToggleInput} no secuencial.}
    \Caracteristica{Validar las anotaciones y el tipo de par�metros utilizados en el constructor de la clase \textit{Registrable}.}
    \Objetivo{Validar que si \textit{Registrable}, en el elemento \textit{numberToggle} de la anotaci�n \textit{@Parameters} recibe las anotaciones \textit{@NumberToggleInput} de manera no secuencial se lanza una excepci�n. Con secuencial se refiere a que \textit{@NumberToggleInput} con el mismo \textit{groupID} deben estar juntos.}
    \Configuracion{%
        Implementaci�n de \textit{Registrable} con las anotaciones \textit{@NewProblem} y \textit{@Parameters}. El elemento \textit{numberToggle} de \textit{@Parameters} no tiene las anotaciones \textit{@NumberToggleInput} con el mismo \textit{groupID} juntos.
    }
    \DatosPrueba{%
        Objeto \textit{Class$<$?$>$} que referencia al tipo \textit{Registrable}.
    }
    \AccionesPrueba{%
        Pasar el objeto \textit{Class} que hace referencia al tipo \textit{Registrable} al m�todo \textit{validateRegistrableProblem}.
    }
    \ResultadosEsperados{M�todo termina sin error.}
\end{prueba}

\begin{prueba}
    \TestID{\atcounter}
    \Titulo{\textit{Registrable} con par�metros en el constructor que no est�n definidos en la anotaci�n \textit{@Parameters}.}
    \Caracteristica{Validar las anotaciones y el tipo de par�metros utilizados en el constructor de la clase \textit{Registrable}.}
    \Objetivo{Validar que si el constructor \textit{Registrable} tiene un p�rametro no correspondiente al indicado en la anotaci�n \textit{@Parameters} se lanza una excepci�n.}
    \Configuracion{%
        Implementaci�n de \textit{Registrable} con las anotaciones \textit{@NewProblem} y \textit{@Parameters}. El constructor indica que recibir� un \textit{File} mientras que la anotaci�n \textit{@Parameters} indica que en esa posici�n deber�a recibirse un \textit{Object}, el cual hace referencia a un operador para un algoritmo metaheur�stico.
    }
    \DatosPrueba{%
        Objeto \textit{Class$<$?$>$} que referencia al tipo \textit{Registrable}.
    }
    \AccionesPrueba{%
        Pasar el objeto \textit{Class} que hace referencia al tipo \textit{Registrable} al m�todo \textit{validateRegistrableProblem}.
    }
    \ResultadosEsperados{Una excepci�n.}
\end{prueba}

\section{\textit{ResultSimutation}}

En esta secci�n se especifica los test automatizados realizados sobre la clase \textit{ResultSimulation}.

\begin{prueba}
    \TestID{\atcounter}
    \Titulo{\textit{timeInSeconds} guardado correctamente.}
    \Caracteristica{Validar que la clase abstracta guarda el par�metro \textit{timeInSeconds} correctamente.}
    \Objetivo{Validar que al pasar el tiempo en segundos a la superclase \textit{ResultSimulation} se guarda el tiempo correctamente en el campo \textit{timeInSeconds}.}
    \Configuracion{%
        Clase que hereda de \textit{ResultSimulation} cuyo constructor delega el par�metro \textit{timeInSeconds} para que sea guardado en la superclase.
    }
    \DatosPrueba{%
        \textit{timeInSeconds}: 72000
    }
    \AccionesPrueba{%
        Pasar al constructor de la superclase abstracta \textit{ResultSimulation} el tiempo en segundo de la simulaci�n.
    }
    \ResultadosEsperados{\textit{timeInSeconds} igual a 72000.}
\end{prueba}

\begin{prueba}
    \TestID{\atcounter}
    \Titulo{\textit{String} que representa el valor de \textit{timeInSeconds} en formato HH:mm:ss.}
    \Caracteristica{Validar el m�todo \textit{getTimeString}.}
    \Objetivo{Validar que al pasar el tiempo en segundos a la superclase \textit{ResultSimulation} y llamar al m�todo \textit{getTimeString} se retorna un \textit{String} con la hora en formato HH:mm:ss.}
    \Configuracion{%
        Clase que hereda de \textit{ResultSimulation} cuyo constructor delega el par�metro \textit{timeInSeconds} para que sea guardado en la superclase.
    }
    \DatosPrueba{%
        \textit{timeInSeconds}: 72000, 0, 1, 1000, 86159, 86399.
    }
    \AccionesPrueba{%
        Pasar al constructor de la superclase abstracta \textit{ResultSimulation} el tiempo en segundo de la simulaci�n.
    }
    \ResultadosEsperados{Los \textit{String} "20:00:00", "00:00:00", "00:00:01", "00:30:00", "23:55:59","23:59:59".}
\end{prueba}

\begin{prueba}
    \TestID{\atcounter}
    \Titulo{\textit{timeInSeconds} fuera del rango de simulaci�n.}
    \Caracteristica{Validar el intervalo permitido para \textit{timeInSeconds}}
    \Objetivo{Validar que si \textit{timeInSeconds} esta fuera del rango de las 24 horas se lanza una excepci�n.}
    \Configuracion{%
        Clase que hereda de \textit{ResultSimulation} cuyo constructor delega el par�metro \textit{timeInSeconds} para que sea guardado en la superclase.
    }
    \DatosPrueba{%
        \textit{timeInSeconds}: -2, -1, 86400(24:00:00), 86401 (24:00:01).
    }
    \AccionesPrueba{%
        Pasar al constructor de la superclase abstracta \textit{ResultSimulation} el tiempo en segundo de la simulaci�n.
    }
    \ResultadosEsperados{Una excepci�n para cualquiera de los valores de \textit{timeInSeconds} usados.}
\end{prueba}

\section{\textit{JsonSimpleReader}}

En esta secci�n se especifican los casos de prueba para la clase \textit{JsonSimpleReader}.
% create and save a box to use in commands
\newsavebox{\js}
\begin{lrbox}{\js}
    \begin{minipage}{0.7\textwidth}
        \begin{lstlisting}[language=json]
{
    "int": 5,
    "double": 2.5,
    "ints": [0,1,2,3,4,5],
    "doubles": [0.1,0.2,2.3,2.5,2.5],
    "doubleMatrix": [[0.2,0.3,0.4],[1.2,1.3,1.5]],
    "intMatrix": [[0,1,2],[3,4,5]],
    "string": "A simple string",
    "boolean": true
}
        \end{lstlisting}%
      \end{minipage}%  
\end{lrbox}

\begin{prueba}
    \TestID{\atcounter}
    \Titulo{Leer un entero (\textit{int}) desde un json.}
    \Caracteristica{Leer valores desde un json.}
    \Objetivo{Validar que el m�todo \textit{getInt} retorna un \textit{int} cuando se lee un n�mero desde una propiedad en json.}
    \Configuracion{%
        Instancia de \textit{JsonSimpleReader} inicializado y listo para leer.
    }
    \DatosPrueba{%
        Un json con los siguientes valores:\\
        \usebox{\js}}
    \AccionesPrueba{%
        Llamar al m�todo \textit{getInt}.
    }
    \ResultadosEsperados{El entero leido desde el json con la clave ``\textit{int}''.}
\end{prueba}

\begin{prueba}
    \TestID{\atcounter}
    \Titulo{Leer un n�mero decimal (\textit{double}) desde un json.}
    \Caracteristica{Leer valores desde un json.}
    \Objetivo{Validar que el m�todo \textit{getDouble} retorna un \textit{double} cuando se lee un n�mero desde una propiedad en json.}
    \Configuracion{%
        Instancia de \textit{JsonSimpleReader} inicializado y listo para leer.
    }
    \DatosPrueba{%
        Un json con los siguientes valores:\\
        \usebox{\js}}
    \AccionesPrueba{%
        Llamar al m�todo \textit{getDouble}.
    }
    \ResultadosEsperados{El double leido desde el json con la clave ``\textit{double}''.}
\end{prueba}

\begin{prueba}
    \TestID{\atcounter}
    \Titulo{Leer un valor booleano (\textit{boolean}) desde un json.}
    \Caracteristica{Leer valores desde un json.}
    \Objetivo{Validar que el m�todo \textit{getBoolean} retorna un \textit{boolean} cuando se lee una propiedad booleana desde el json.}
    \Configuracion{%
        Instancia de \textit{JsonSimpleReader} inicializado y listo para leer.
    }
    \DatosPrueba{%
        Un json con los siguientes valores:\\
        \usebox{\js}}
    \AccionesPrueba{%
        Llamar al m�todo \textit{getBoolean}.
    }
    \ResultadosEsperados{El boleano leido desde el json con la clave ``\textit{boolean}''.}
\end{prueba}

\begin{prueba}
    \TestID{\atcounter}
    \Titulo{Leer un arreglo de enteros (\textit{int}[]) desde un json.}
    \Caracteristica{Leer valores desde un json.}
    \Objetivo{Validar que el m�todo \textit{getIntegerArray} retorna un \textit{int}[] cuando se lee un arreglo desde el json.}
    \Configuracion{%
        Instancia de \textit{JsonSimpleReader} inicializado y listo para leer.
    }
    \DatosPrueba{%
        Un json con los siguientes valores:\\
        \usebox{\js}}
    \AccionesPrueba{%
        Llamar al m�todo \textit{getIntegerArray}.
    }
    \ResultadosEsperados{El arreglo de enteros leidos desde el json con la clave ``\textit{ints}''.}
\end{prueba}

\begin{prueba}
    \TestID{\atcounter}
    \Titulo{Leer un arreglo de n�meros decimales (\textit{double}[]) desde un json.}
    \Caracteristica{Leer valores desde un json.}
    \Objetivo{Validar que el m�todo \textit{getDoubleArray} retorna un \textit{double}[] cuando se lee un arreglo desde el json.}
    \Configuracion{%
        Instancia de \textit{JsonSimpleReader} inicializado y listo para leer.
    }
    \DatosPrueba{%
        Un json con los siguientes valores:\\
        \usebox{\js}}
    \AccionesPrueba{%
        Llamar al m�todo \textit{getDoubleArray}.
    }
    \ResultadosEsperados{El arreglo de valores decimales leido desde el json con la clave ``\textit{doubles}''.}
\end{prueba}

\begin{prueba}
    \TestID{\atcounter}
    \Titulo{Leer un matriz de enteros (\textit{int}[][]) desde un json.}
    \Caracteristica{Leer valores desde un json.}
    \Objetivo{Validar que el m�todo \textit{getIntegerMatrix} retorna un \textit{int}[][] cuando se lee un matriz desde el json.}
    \Configuracion{%
        Instancia de \textit{JsonSimpleReader} inicializado y listo para leer.
    }
    \DatosPrueba{%
        Un json con los siguientes valores:\\
        \usebox{\js}}
    \AccionesPrueba{%
        Llamar al m�todo \textit{getIntegerMatrix}.
    }
    \ResultadosEsperados{La matriz de valores enteros leido desde el json con clave la ``\textit{intMatrix}''.}
\end{prueba}

\begin{prueba}
    \TestID{\atcounter}
    \Titulo{Leer un matriz de decimales (\textit{double}[][]) desde un json.}
    \Caracteristica{Leer valores desde un json.}
    \Objetivo{Validar que el m�todo \textit{getDoubleMatrix} retorna un \textit{double}[][] cuando se lee un matriz desde el json.}
    \Configuracion{%
        Instancia de \textit{JsonSimpleReader} inicializado y listo para leer.
    }
    \DatosPrueba{%
        Un json con los siguientes valores:\\
        \usebox{\js}}
    \AccionesPrueba{%
        Llamar al m�todo \textit{getDoubleMatrix}.
    }
    \ResultadosEsperados{La matriz de valores decimales leido desde el json con la clave ``\textit{doubleMatrix}''.}
\end{prueba}




\chapter{Casos de pruebas manuales}

En este cap�tulo se presenta la especificaci�n formal de los casos de prueba manuales que son realizados en el software. Estas pruebas son aquellas que se realizan con la ayuda de interacci�n humana.
%it3
\begin{prueba}
    \TestID{\mtcounter}
    \Titulo{Visualizaci�n de la red.}
    \Caracteristica{Mostrar visualizaci�n de la red.}
    \Objetivo{Confirmar que la red puede ser leida desde un archivo con extensi�n ``.inp'' y ser visualizada en la aplicaci�n.}
    \Configuracion{El equipo tiene la aplicaci�n JHawanetFramework lista para ejecutar.}
    \DatosPrueba{%
        inp: ``hanoi-Frankenstein.inp''.
    }
    \AccionesPrueba{%
        1. Abrir JHawanetFramework.\\
        2. Cargar archivo de red.
    }
    \ResultadosEsperados{El sistema muestra la red leida desde un archivo inp gr�ficamente en la aplicaci�n.}
\end{prueba}

\begin{prueba}
    \TestID{\mtcounter}
    \Titulo{Optimizaci�n monoobjetivo realizada completamente.}
    \Caracteristica{Realizar simulaci�n monoobjetivo.}
    \Objetivo{Confirmar que se puede llevar a cabo la resoluci�n del problema \textit{Pipe Optimizing} sobre la red abierta.}
    \Configuracion{El equipo tiene la aplicaci�n JHawanetFramework lista para ejecutar.}
    \DatosPrueba{%
        \textit{independentRun} = 10.\\
        Archivo inp: ``hanoi-Frankenstein.inp''.\\
        Archivo gama: ``hanoiHW.Gama''.
    }
    \AccionesPrueba{%
        1. Abrir JHawanetFramework.\\
        2. Cargar archivo de red.\\
        3. Seleccionar el problema \textit{Pipe Optimizing} del men�.\\
        4. Configurar el problema usando la ventana de configuraci�n.\\
        5. Realizar la optimizaci�n.
    }
    \ResultadosEsperados{Al terminar la optimizaci�n el sistema muestra una interfaz con las soluciones generadas por la optimizaci�n. Debe haber tantas soluciones como el numero de configuraciones independientes (\textit{independentRun}) establecidas.}
\end{prueba}

\begin{prueba}
    \TestID{\mtcounter}
    \Titulo{Optimizaci�n monoobjetivo cancelada.}
    \Caracteristica{Realizar simulaci�n monoobjetivo.}
    \Objetivo{Confirmar que se puede llevar a cabo la resoluci�n del problema \textit{Pipe Optimizing} sobre la red abierta y que se puede cancelar el proceso  cerrando la ventana o pulsando cancelar.}
    \Configuracion{El equipo tiene la aplicaci�n JHawanetFramework lista para ejecutar.}
    \DatosPrueba{%
        Archivo inp: ``hanoi-Frankenstein.inp''.\\
        Archivo gama: ``hanoiHW.Gama''.
    }
    \AccionesPrueba{%
        1. Abrir JHawanetFramework.\\
        2. Cargar archivo de red.\\
        3. Seleccionar el problema \textit{Pipe Optimizing} del men�.\\
        4. Configurar el problema usando la ventana de configuraci�n.\\
        5. Realizar la optimizaci�n.
    }
    \ResultadosEsperados{Al cancelar la b�squeda de soluciones la ventana de estado indica que la optimizaci�n ha sido detenida.}
\end{prueba}

%it 4

\begin{prueba}
    \TestID{\mtcounter}
    \Titulo{Optimizaci�n multiobjetivo realizada completamente.}
    \Caracteristica{Realizar simulaci�n multiobjetivo.}
    \Objetivo{Confirmar que se puede llevar a cabo la resoluci�n del problema \textit{Pumping Schedule} sobre la red abierta.}
    \Configuracion{El equipo tiene la aplicaci�n JHawanetFramework lista para ejecutar.}
    \DatosPrueba{%
        Archivo inp: ``Vanzyl.inp''.\\
        Archivo gama: ``VanzylConfiguration.json''.
    }
    \AccionesPrueba{%
        1. Abrir JHawanetFramework.\\
        2. Cargar archivo de red.\\
        3. Seleccionar el problema \textit{Pumping Schedule} del men�.\\
        4. Seleccionar el algoritmo NSGAII.\\
        5. Configurar el problema usando la ventana de configuraci�n.\\
        6. Realizar la optimizaci�n.
    }
    \ResultadosEsperados{Al terminar la optimizaci�n el sistema muestra una interfaz con las soluciones generadas por la optimizaci�n. Estas soluciones corresponden a la Frontera de Pareto del Experimento.}
\end{prueba}

\begin{prueba}
    \TestID{\mtcounter}
    \Titulo{Optimizaci�n multiobjetivo cancelada.}
    \Caracteristica{Realizar simulaci�n multiobjetivo.}
    \Objetivo{Confirmar que se puede llevar a cabo la resoluci�n del problema \textit{Pumping Schedule} sobre la red abierta y que se puede cancelar el proceso cerrando la ventana o pulsando cancelar.}
    \Configuracion{El equipo tiene la aplicaci�n JHawanetFramework lista para ejecutar.}
    \DatosPrueba{%
        Archivo inp: ``Vanzyl.inp''.\\
        Archivo gama: ``VanzylConfiguration.json''.
    }
    \AccionesPrueba{%
        1. Abrir JHawanetFramework.\\
        2. Cargar archivo de red.\\
        3. Seleccionar el problema \textit{Pumping Schedule} del men�.\\
        4. Seleccionar el algoritmo NSGAII.\\
        5. Configurar el problema usando la ventana de configuraci�n.\\
        6. Realizar la optimizaci�n.
    }
    \ResultadosEsperados{Al cancelar la b�squeda de soluciones la ventana de estado indica que la optimizaci�n ha sido detenida.}
\end{prueba}

\begin{prueba}
    \TestID{\mtcounter}
    \Titulo{Ver soluciones de los problemas monoobjetivos gr�ficamente a medida que se ejecuta la optimizaci�n.}
    \Caracteristica{Visualizar gr�ficamente las soluciones.}
    \Objetivo{Comprobar que a medida que algoritmo va generando soluciones �stas pueden ser visualizadas. El gr�fico es Objetivo vs Numero de generaciones.}
    \Configuracion{El equipo tiene la aplicaci�n JHawanetFramework lista para ejecutar.}
    \DatosPrueba{%
        Archivo inp: ``hanoi-Frankenstein.inp''.\\
        Archivo gama: ``hanoiHW.Gama''.}
    \AccionesPrueba{%
        1. Abrir JHawanetFramework.\\
        2. Cargar archivo de red.\\
        3. Escoger el problema monoobjetivo.\\
        4. Configurar el problema monoobjetivo y ejecutar.\\
        5. Ir a la ventana del gr�fico.
    }
    \ResultadosEsperados{
        Un gr�fico de dos dimensiones en el que se muestre los resultados de las soluciones generadas por cada generaci�n de cada una de las ejecuciones independientes.
    }
\end{prueba}

\begin{prueba}
    \TestID{\mtcounter}
    \Titulo{Ver soluciones del problema multiobjetivos, con dos objetivos, gr�ficamente a medida que se ejecuta la optimizaci�n.}
    \Caracteristica{Visualizar gr�ficamente las soluciones}
    \Objetivo{Comprobar que a medida que algoritmo va generando soluciones estas pueden ser visualizadas. El gr�fico es Objetivo1 vs Objetivo2.}
    \Configuracion{El equipo tiene la aplicaci�n JHawanetFramework lista para ejecutar.}
     \DatosPrueba{%
        Archivo inp: ``Vanzyl.inp''.\\
        Archivo gama: ``VanzylConfiguration.json''.
    }
    \AccionesPrueba{%
        1. Abrir JHawanetFramework.\\
        2. Cargar archivo de red.\\
        3. Escoger el problema monoobjetivo.\\
        4. Configurar el problema monoobjetivo y ejecutar.\\
        5. Ir a la ventana del gr�fico.
    }
    \ResultadosEsperados{
        Un gr�fico de dos dimensiones en el que se muestre los resultados de las soluciones generadas por cada generaci�n de cada una de las ejecuciones independientes.
    }
\end{prueba}

\begin{prueba}
    \TestID{\mtcounter}
    \Titulo{Guardar la gr�fica de las soluciones del problema monoobjetivo como png.}
    \Caracteristica{Guardar gr�fica de soluciones.}
    \Objetivo{Confirmar que se puede guardar el gr�fico de soluciones como un archivo png.}
    \Configuracion{El equipo tiene la aplicaci�n JHawanetFramework lista para ejecutar.}
    \DatosPrueba{%
        Archivo inp: ``hanoi-Frankenstein.inp''.\\
        Archivo gama: ``hanoiHW.Gama''.
    }
    \AccionesPrueba{%
        1. Abrir JHawanetFramework.\\
        2. Cargar archivo de red.\\
        3. Escoger el problema monoobjetivo.\\
        4. Configurar el problema monoobjetivo y ejecutar.\\
        5. Ir a la ventana del gr�fico.\\
        6. Pulsar el boton para guardar una captura del gr�fico.\\
        7. Configurar donde guardar la captura.
    }
    \ResultadosEsperados{
        Generaci�n de un archivo png en el equipo.
    }
\end{prueba}

\begin{prueba}
    \TestID{\mtcounter}
    \Titulo{Guardar la gr�fica de soluciones de problemas de 2 objetivos como png.}
    \Caracteristica{Guardar gr�fica de soluciones.}
    \Objetivo{Confirmar que se puede guardar el gr�fico de soluciones como un png.}
    \Configuracion{El equipo tiene la aplicaci�n JHawanetFramework lista para ejecutar.}
    \DatosPrueba{%
        Archivo inp: ``Vanzyl.inp''.\\
        Archivo gama: ``VanzylConfiguration.json''.
    }
    \AccionesPrueba{%
        1. Abrir JHawanetFramework.\\
        2. Cargar archivo de red.\\
        3. Escoger el problema monoobjetivo.\\
        4. Configurar el problema monoobjetivo y ejecutar.\\
        5. Ir a la ventana del gr�fico.\\
        6. Pulsar el boton para guardar una captura del gr�fico.\\
        7. Configurar donde guardar la captura.
    }
    \ResultadosEsperados{
        Generaci�n de un archivo png en el equipo.
    }
\end{prueba}

\begin{prueba}
    \TestID{\mtcounter}
    \Titulo{Guardar la solucion seleccionada como inp.}
    \Caracteristica{Guardar resultados de la optimizaci�n.}
    \Objetivo{Confirmar que se puede exportar los resultados de la soluci�n sobre el archivo de red (inp).}
    \Configuracion{El equipo tiene la aplicaci�n JHawanetFramework lista para ejecutar.}
    \DatosPrueba{%
        Archivo inp: ``hanoi-Frankenstein.inp''\\
        Archivo gama: ``hanoiHW.Gama''
    }
    \AccionesPrueba{%
        1. Abrir JHawanetFramework.\\
        2. Cargar archivo de red.\\
        3. Escoger el problema monoobjetivo.\\
        4. Configurar el problema monoobjetivo y ejecutar.\\
        5. Esperar que la ejecucion finalice.\\
        6. Seleccionar una soluci�n.\\
        7. Pulsar el boton guardar como inp.\\
        8. Configurar donde guardar el archivo.
    }
    \ResultadosEsperados{
        Generaci�n del archivo de configuraci�n de la red (inp) con la soluci�n aplicada. 
    }
\end{prueba}

\begin{prueba}
    \TestID{\mtcounter}
    \Titulo{Guardar las soluciones en archivos tsv.}
    \Caracteristica{Guardar resultados de la optimizaci�n.}
    \Objetivo{Confirmar que se puede exportar las soluciones a dos archivos tsv. Uno para las variables y el otro para los objetivos.}
    \Configuracion{El equipo tiene la aplicaci�n JHawanetFramework lista para ejecutar.}
    \DatosPrueba{%
        Archivo inp: ``hanoi-Frankenstein.inp''.\\
        Archivo gama: ``hanoiHW.Gama''.
    }
    \AccionesPrueba{%
        1. Abrir JHawanetFramework.\\
        2. Cargar archivo de red.\\
        3. Escoger el problema monoobjetivo.\\
        4. Configurar el problema monoobjetivo y ejecutar.\\
        5. Esperar que la ejecucion finalice.\\
        6. Pulsar el boton guardar tabla.\\
        7. Configurar donde guardar los archivo.
    }
    \ResultadosEsperados{
        Generaci�n del 2 archivos .tsv. Uno con el prefijo FUN\_ y otro con el prefijo VAR\_. El archivo FUN
        contiene el valor de los objetivos de las soluciones. El archivo VAR contiene las variables de la soluciones.
    }
\end{prueba}

\begin{prueba}
    \TestID{\mtcounter}
    \Titulo{Guardar las soluciones en un Excel.}
    \Caracteristica{Guardar resultados de la optimizaci�n.}
    \Objetivo{Confirmar que se puede exportar la tabla de resultados completa a un Excel.}
    \Configuracion{El equipo tiene la aplicaci�n JHawanetFramework lista para ejecutar.}
    \DatosPrueba{%
        Archivo inp: ``hanoi-Frankenstein.inp''.\\
        Archivo gama: ``hanoiHW.Gama''.
    }
    \AccionesPrueba{%
        1. Abrir JHawanetFramework.\\
        2. Cargar archivo de red.\\
        3. Escoger el problema monoobjetivo.\\
        4. Configurar el problema monoobjetivo y ejecutar.\\
        5. Esperar que la ejecucion finalice.\\
        6. Pulsar el boton guardar tabla como Excel.\\
        7. Configurar donde guardar el archivo.
    }
    \ResultadosEsperados{
        Un archivo Excel con los mismos datos que la tabla de resultados de la aplicaci�n.
    }
\end{prueba}

\begin{prueba}
    \TestID{\mtcounter}
    \Titulo{Simulaci�n hidr�ulica de la red en un solo per�odo de tiempo.}
    \Caracteristica{Realizar simulaci�n hidr�ulica con los valores establecidos en el archivo de configuraci�n de red.}
    \Objetivo{Confirmar que se puede realizar la simulaci�n utilizando los valores del archivo de configuraci�n de red.}
    \Configuracion{El equipo tiene la aplicaci�n JHawanetFramework lista para ejecutar.}
    \DatosPrueba{%
        Archivo inp: ``hanoi-Frankenstein.inp''.\\
        Archivo gama: ``hanoiHW.Gama''.
    }
    \AccionesPrueba{%
        1. Abrir JHawanetFramework.\\
        2. Cargar archivo de red.\\
        3. Pulsar bot�n \textit{Execute} de la ventana principal.
    }
    \ResultadosEsperados{
        Ejecuci�n realizada sin ningun error.
    }
\end{prueba}

\begin{prueba}
    \TestID{\mtcounter}
    \Titulo{Ver resultados de la simulaci�n hidr�ulica de un solo per�odo de tiempo}
    \Caracteristica{Realizar simulaci�n hidr�ulica con los valores establecidos en el archivo de configuraci�n de red.}
    \Objetivo{Confirmar que se puede visualizar los resultados de la simulaci�n realizada.}
    \Configuracion{El equipo tiene la aplicaci�n JHawanetFramework lista para ejecutar.}
    \DatosPrueba{%
        Archivo inp: ``hanoi-Frankenstein.inp''.\\
        Archivo gama: ``hanoiHW.Gama''.
    }
    \AccionesPrueba{%
        1. Abrir JHawanetFramework.\\
        2. Cargar archivo de red.\\
        3. Pulsar bot�n \textit{Execute} de la ventana principal.\\
        4. Pulsar bot�n \textit{Results} de la ventana principal.
    }
    \ResultadosEsperados{
        Una interfaz que permite seleccionar si se quiere ver los resultados para los enlaces o los nodos.
    }
\end{prueba}

\begin{prueba}
    \TestID{\mtcounter}
    \Titulo{Simulaci�n hidr�ulica de la red de m�s de un per�odo de tiempo de simulaci�n.}
    \Caracteristica{Realizar simulaci�n hidr�ulica con los valores establecidos en el archivo de configuraci�n de red.}
    \Objetivo{Confirmar que se puede realizar la simulaci�n utilizando los valores del archivo de configuraci�n de red.}
    \Configuracion{El equipo tiene la aplicaci�n JHawanetFramework lista para ejecutar.}
    \DatosPrueba{%
        Archivo inp: ``Vanzyl.inp''.\\
        Archivo gama: ``VanzylConfiguration.json''.
    }
    \AccionesPrueba{%
        1. Abrir JHawanetFramework.\\
        2. Cargar archivo de red.\\
        3. Pulsar boton \textit{Execute} de la ventana principal.
    }
    \ResultadosEsperados{
        Ejecuci�n realizada sin ningun error.
    }
\end{prueba}

\begin{prueba}
    \TestID{\mtcounter}
    \Titulo{Ver resultados de la simulaci�n hidr�ulica de mas de un per�odo de tiempo}
    \Caracteristica{Realizar simulaci�n hidr�ulica con los valores establecidos en el archivo de configuraci�n de red.}
    \Objetivo{Confirmar que se puede visualizar los resultados de la simulaci�n realizada.}
    \Configuracion{El equipo tiene la aplicaci�n JHawanetFramework lista para ejecutar.}
     \DatosPrueba{%
        Archivo inp: ``Vanzyl.inp''.\\
        Archivo gama: ``VanzylConfiguration.json''.
    }
    \AccionesPrueba{%
        1. Abrir JHawanetFramework.\\
        2. Cargar archivo de red.\\
        3. Pulsar bot�n \textit{Execute} de la ventana principal.\\
        4. Pulsar bot�n \textit{Results} de la ventana principal.
    }
    \ResultadosEsperados{
        Una interfaz que permite seleccionar si se quieren ver los resultados para los enlaces o los nodos. Adicionalmente, permite escoger el tiempo de simulaci�n y listar los resultados para todos los tiempos de un elemento de la red espec�fico.
    }
\end{prueba}

%% ambiente glosario
%\begin{glosario}
%  \item[RDA] Este es el significado del primer t�rmino, realmente no se bien lo que significa pero podr�a haberlo averiguado si hubiese tenido un poco mas de tiempo.
%  \item[GA] Este si se lo que significa pero me da lata escribirlo...
%\end{glosario}


%% genera las referencias
%\bibliography{refs}


%% comienzo de la parte de anexos
%\appendixpart

%% contenido del primer anexo
%%\appendix{Apendix}

\end{document}

   

