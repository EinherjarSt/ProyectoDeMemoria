\section{Alcance del proyecto}

Al final del periodo de desarrollo la herramienta contara con las siguientes prestaciones.

\begin{itemize}
    \item	El sistema permitir� la carga y la visualizaci�n de la red gr�ficamente.
    \item	El sistema solo resolver� dos clases de problemas de optimizaci�n, uno mono-objetivo y el otro multiobjetivo. El problema mono-objetivo ser� el de los costos de inversi�n. En cuanto al problema multiobjetivo, este ser� el de los costos energ�ticos y el n�mero de encendidos y apagado de las bombas.  
    \item	El sistema �nicamente contara con dos algoritmos implementados los cuales ser�n el algoritmo gen�tico y NSGAII. El algoritmo gen�tico ser� el usado para tratar el problema mono-objetivo, mientras que NSGAII ser� aplicado al multiobjetivo.
    \item	El sistema permitir� visualizar y guardar las soluciones de los algoritmos en un archivo.
    \item	El sistema permitir� que el usuario agregue nuevos algoritmos, operadores o problemas sin tener que modificar la interfaz de usuario.
\end{itemize}

Este proyecto no contempla la creaci�n de la red por lo que estas deber�n ser ingresadas como entradas al programa. 

Adem�s, esta herramienta �nicamente podr� ser ocupada en equipos cuyo sistema operativo sea Windows debido a que se realizan llamadas a librer�as nativas.
