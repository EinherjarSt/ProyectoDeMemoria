\section{Contexto}

Este sistema ser� desarrollado utilizando el lenguaje de programaci�n java. Debido a que este es un lenguaje ampliamente utilizado y que cuenta con un gran soporte y comunidad que lo utilizan.

Como motor de c�lculo para llevar a cabo las simulaciones se utilizar� una librer�a desarrollada en c, que cuenta con funciones para llevar a cabo simulaciones de redes de agua potable. El nombre de esta librer�a es epanet2.dll. Las funciones que incorpora esta librer�a se encuentran explicadas en~\cite{Rossman2017}.

Desde lenguaje se realizar�n llamadas a librer�as nativas usando la librer�a JNA existente en java. Esta librer�a cuenta con las clases y m�todos necesarios para poder acoplar este sistema a la librer�a epanet2.dll desarrollada en c y que ser� usada como motor de c�lculo para llevar a cabo las simulaciones.
Puesto que una de las funcionalidades del sistema es permitir la ejecuci�n de algoritmos metaheur�sticos, se toma como base la arquitectura presentada por el framework JMetal.

JMetal es un framework para la optimizaci�n multiobjetivo con metaheur�sticas. Su arquitectura inicial~\cite{Durillo2010} involucraba una serie de problemas y dificultaban la realizaci�n de ciertas acciones que eran recurrentes. Adem�s, esta no hacia uso de las novedades incorporadas por Java como los gen�ricos. Es por esto, que posteriormente fue redise�ada, haciendo uso de patrones de dise�o, principios de la programaci�n orientada a objetos y aprovechando las caracter�sticas del lenguaje Java. Este redise�o se presenta en~\cite{Nebro2015}.

El contexto en el que se desenvolver� este sistema ser� en ambientes universitario, de investigaci�n y en el ambiente laboral. 
