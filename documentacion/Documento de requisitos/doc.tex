%% inicio, la clase del documento es iccmemoria.cls
\documentclass{especificacion}
\usepackage[onelanguage, commentsnumbered, linesnumbered, boxed, ruled]{algorithm2e}
\usepackage[draft=false,kerning=true]{microtype}
\usepackage{pdfpages}
\usepackage{amsfonts}
\usepackage{amssymb}
\usepackage{array}
\usepackage{longtable}
\usepackage{float}
\usepackage{especificacionutility}
\usepackage{setspace}
\usepackage{adjustbox}
\usepackage[linktocpage, hidelinks]{hyperref}

%% datos generales y para la tapa
\titulo{Documento de especificaci�n de requisitos}
\subtitulo{Herramienta para la optimizaci�n de redes de distribuci�n de agua potable}
\author{Gabriel Gonzalo Alexander Sanhueza Fuentes}


%% inicio de documento
\begin{document}

%% crea la tapa
\maketitle

\adddevelopment{Gabriel Sanhueza Fuentes}{Administrador, Analista, Dise�ador, Implementador y Tester.}{gsanhueza15@alumnos~.utalca.cl}

\addcounterpart{Jimmy H. Guti�rrez-Bahamondes}{Cliente/Profesor gu�a}{}
\addcounterpart{Jimmy H. Guti�rrez-Bahamondes}{Cliente/Profesor co-gu�a}{}

\addrevision{0.1}{07/09/2019}{Primer borrador}
\addrevision{1}{15/06/2020}{Actualizaci�n documento. Agregados los requisitos faltantes.}


\RevisionHistoryPage


%% indices
\tableofcontents
%\listoffigures
%\listoftables

%% abstract

%WRITE HERE

\chapter{Introducci�n}
El proyecto que se desarrollara consiste en la creaci�n de una herramienta que haga uso de algoritmos metaheur�sticos para tratar y minimizar problemas existentes en redes de distribuci�n de agua potable.

Este proyecto solo abarcara dos problemas existentes dentro de las redes de distribuci�n de agua potable.

Para el desarrollo de este proyecto se usar�n librer�as ya existentes con el fin de reducir el tiempo de desarrollo. Estas librer�as son epanet2.dll, desarrollada en c, y JNA, librer�a en java para funciones nativas.


\section{Prop�sito del sistema}

El proyecto consiste en el desarrollo de un sistema que permita simular y buscar soluciones a problemas presentes en las redes de distribuci�n de agua potable haciendo uso de algoritmos metaheur�sticos. Adem�s, este sistema ser� desarrollado de tal forma que sea escalable con el fin de que otros desarrolles o investigadores sean capaces de extender su funcionalidad f�cilmente agregando nuevos algoritmos metaheur�sticos y problemas a tratar en el contexto de la distribuci�n de redes de agua potable.

\section{Alcance del proyecto}

Al final del periodo de desarrollo la herramienta contara con las siguientes prestaciones.
\begin{itemize}
    \item El sistema permitir� la carga y la visualizaci�n de la red gr�ficamente.
    \item El sistema inicialmente solo resolver� dos clases de problemas de optimizaci�n, uno monoobjetivo y el otro multiobjetivo. El problema monoobjetivo ser� el de los costos de inversi�n. En cuanto al problema multiobjetivo, este ser� el de los costos energ�ticos y el n�mero de encendidos y apagado de las bombas.  
    \item El sistema inicialmente �nicamente contara con dos algoritmos implementados los cuales ser�n el algoritmo gen�tico y NSGAII. El algoritmo gen�tico ser� el usado para tratar el problema monoobjetivo, mientras que NSGAII ser� aplicado al multiobjetivo.
\end{itemize}

Este proyecto no contempla la creaci�n de la red por lo que estas deber�n ser ingresadas como entradas al programa. La creaci�n de las redes puede realizarse a trav�s de EPANET. Adem�s, esta herramienta �nicamente podr� ser ocupada en equipos cuyo sistema operativo sea Windows debido a que se realizan llamadas a librer�as nativas.

\section{Contexto del proyecto} 
La escasez de agua potable es sin duda una problem�tica a nivel mundial y la optimizaci�n de los sistemas que permiten su distribuci�n es cada d�a m�s relevante. Existe una serie de problem�ticas asociadas a la determinaci�n de las condiciones �ptimas de operaciones y las caracter�sticas adecuadas para su construcci�n.

Las RDA son redes que pueden ser muy extensas y complejas. Forman parte de la estructura principal de cualquier ciudad. Deben ser capaces de adaptarse a los cambios y asegurar niveles m�nimos de servicios durante las 24 horas del d�a~\cite{Pereyra2017}. Adicionalmente, dependiendo de su topolog�a, las RDA integran sistemas de bombeo que requieren gran cantidad de energ�a en horarios determinados.

La optimizaci�n de estos sistemas, a la vez, involucra la participaci�n de m�ltiples criterios que deben ser tomados en cuenta a la hora de decidir. Sin embargo, la incorporaci�n de �stos, involucra la generaci�n de modelos cada vez m�s complejos~\cite{JHawanet-2019}.

Por lo anteriormente mencionado, esta �rea, ha llamado la atenci�n de muchos investigadores que han creado diversos m�todos para resolver la problem�tica desde diferentes enfoques. Sin embargo, a�n existen muy pocas aplicaciones computacionales que permitan emplear los nuevos modelos y t�cnicas de forma pr�ctica. �sto supone un gran problema para los interesados en aplicar estos conocimientos en un contexto real. Generalmente se trata de personas instruidas en tem�ticas relacionadas con la hidr�ulica pero que poseen un escaso manejo de t�cnicas computacionales de optimizaci�n.

En este trabajo se pretende dar respuesta a esa necesidad creciente a trav�s del dise�o e implementaci�n de una aplicaci�n de escritorio. Este nuevo sistema, permitir� a a los usuarios resolver dos de los principales problemas en la optimizaci�n de RDA. En el caso de los problemas con solo un objetivo se utilizar� un algoritmo gen�tico, mientras que en los problemas multiobjetivos se utilizar� NSGAII.
\section{Definiciones, siglas y abreviaturas}

\begin{description}
    \item [\textit{Genetic Algorithm (GA)}:] Estrategia de b�squeda de soluciones basada en la teor�a de la evoluci�n de Darwin. Para realizar esto, el algoritmo  parte desde un conjunto de soluciones denominada poblaci�n he iterativamente, lleva a cabo un proceso de reproducci�n, generando nuevas soluciones~\cite{Heiss-Czedik1997}.
    \item [\textit{Non-dominated Sorting Genetic Algorithm (NSGAII)}:] Algoritmo que utiliza el cruzamiento, mutaci�n y reproducci�n para encontrar un conjunto de soluciones optimas a problemas que cuentan con m�s de un objetivo~\cite{Deb2002}.
    \item [\textit{Metaheuristicas}:] Algoritmos que permiten resolver un amplio rango de problemas de optimizaci�n empleando t�cnicas con alg�n grado de aleatoriedad para encontrar soluciones a un problema. Estos algoritmos no garantizan que la soluci�n encontrada sea la �ptima, pero permiten obtener generalmente aproximaciones a esta~\cite{Yang2015, Boussaid2013, Luke2013}.
    \item [\textit{Java Reflection}:] Caracter�stica de java que permite que un programa se auto examine. Esta caracter�stica est� disponible a trav�s de la \textit{Java Reflection API}, la cual cuenta con m�todos para obtener los \textit{meta-object} de las clases, m�todos, constructores, campos o par�metros. Esta API tambi�n permite crear nuevos objetos cuyo tipo era desconocido al momento de compilar el programa~\cite{Braux1999}.
    \item [\textit{Java Annotation}:] Caracter�stica de java para agregar metadatos a elementos de java (clases, m�todos, par�metros, etc.)~\cite{Rocha2011}. Las anotaciones no tienen efecto directo sobre el c�digo, pero cuando son usadas junto con otras herramientas pueden llegar a ser muy �tiles. Estas herramientas pueden analizar estas anotaciones y realizar acciones en base a estas, por ejemplo, generar archivos adicionales como clases de java, archivos XML, entre otras; ser analizadas durante la ejecuci�n del programa v�a \textit{Java Reflection}, para crear objetos cuyo tipo no conocemos en tiempo de compilaci�n; etc. 
    \item [\textit{GUI}:] Graphical User Interface.

\end{description}

%% genera las referencias
\bibliography{refs}
\chapter{Descripci�n General}
En esta secci�n se describe las caracter�sticas de los usuarios que har�n uso del sistema. Adem�s, se menciona el ambiente operacional de la soluci�n y la relaci�n que este proyecto tiene con otros proyectos. Finalmente, tambi�n se menciona las restricciones generales de la herramienta a desarrollar.

\section{Caracter�sticas de los Usuarios}

Este sistema solo cuenta con un tipo de usuario el cual tendr� acceso a todas las funcionalidades. Se espera que el usuario que este sistema sean ingenieros, investigadores u personas cuenten con el conocimiento b�sico acerca de redes de distribuci�n de agua potable y metaheur�sticas, que les servir� para interpretar los resultados generados por los algoritmos.
\section{Ambiente operacional de la soluci�n}

El ambiente operacional en el que se desarrolla el sistema es el siguiente:
\begin{itemize}
    \item Intel(R) Core(TM) i7-7700HQ CPU @ 2.80Ghz 2.8Ghz
    \item RAM 16GB DDR4
    \item HDD 7200rpm 1T
    \item SSD 256GB PCIe NVME M.2 
    \item Windows 10 x64
    \item NVIDIA GeForce GTX 1050
\end{itemize}


\section{Relaci�n con otros proyectos}

El sistema depende de la librer�a nativa epanet2\_64bit.dll, ya que usara esta librer�a como motor de c�lculo. La librer�a cuenta con 54 funciones dentro de las cuales se encuentran funciones para correr las simulaciones, modificar y obtener configuraciones de la red, modificar los elementos que conforman la red y generar reportes.

Las llamadas desde java a la librer�a nativa ser�n realizadas a trav�s de la librer�a JNA.

Adicionalmente, el sistema toma la arquitectura utilizada por JMETAL como base para agregar los algoritmos, operadores y problemas. Esta arquitectura ser� modificada seg�n se necesite para satisfacer los requisitos del sistema.

\section{Restricciones Generales}

\begin{itemize}
    \item La red ser� ingresada como entrada al programa a trav�s de un archivo .inp.
    \item La herramienta solo estar� disponible para el sistema operativo Window.
\end{itemize}




\chapter{Requisitos}
En esta cap�tulo se presentan los requisitos capturados, los cuales est�n sujetos a cambios a medida que se avanza en las iteraciones.

\section{Requisitos de usuario}

A continuaci�n, se presentar�n los requisitos de usuarios que han sido obtenidos para el desarrollo de este proyecto

Durante la presentaci�n de estos requisitos se hace referencia al archivo de configuraci�n de red. �ste debe ser generado utilizando la aplicaci�n Epanet y guardado con la extensi�n ``inp'' a partir de ahora nos referiremos al archivo de configuraci�n de red simplemente como inp o archivo inp.

\begin{requisito}
    \Requisito{RU001}{Cargar una red.}
    \Descripcion{La red que es representada por el archivo .inp debe ser cargada en el programa para poder llevar a cabo la simulaci�n.}
    
    \Fuente{Jimmy Guti�rrez}
    \Prioridad{Alta}
    \Estabilidad{Intransable}
    \FechaA{09/09/2019}
    \Estado{Cumple}
    \Incremento{1}
    \Tipo{Funcional}
\end{requisito}
    
\begin{requisito}
    \Requisito{RU002}{Resolver el problema monoobjetivo (\textit{Pipe Optimizing})  usando un Algoritmo Gen�tico.}
    \Descripcion{El Algoritmo Gen�tico debe ser aplicado para resolver el problema monoobjetivo que tiene como funci�n objetivo el costo de inversi�n y como variable de decisi�n el di�metro de las tuber�as.}
    
    \Fuente{Jimmy Guti�rrez}
    \Prioridad{Alta}
    \Estabilidad{Intransable}
    \FechaA{09/09/2019}
    \Estado{Cumple}
    \Incremento{2}
    \Tipo{Funcional}
\end{requisito}
    
\begin{requisito}
    \Requisito{RU003}{Resolver el problema multiobjetivo (\textit{Pumping Schedule}) usando un Algoritmo NSGAII.}
    \Descripcion{El algoritmo NSGAII debe ser aplicado al problema multiobjetivo cuyas funciones a optimizar son los costos energ�ticos y los costos de mantenimiento de los equipos de bombe  (\textit{Pumping Schedule}).}
    
    \Fuente{Jimmy Guti�rrez}
    \Prioridad{Alta}
    \Estabilidad{Intransable}
    \FechaA{09/09/2019}
    \Estado{Cumple}
    \Incremento{4}
    \Tipo{Funcional}
\end{requisito}
    
\begin{requisito}
    \Requisito{RU004}{Visualizar red en una interfaz gr�fica.}
    \Descripcion{Se debe mostrar en la interfaz gr�fica una representaci�n de la red (Un dibujo, etc) generada a partir de la informaci�n contenida en el archivo inp.}
    
    \Fuente{Jimmy Guti�rrez}
    \Prioridad{Moderada}
    \Estabilidad{Intransable}
    \FechaA{09/09/2019}
    \Estado{Cumple}
    \Incremento{3}
    \Tipo{Funcional}
\end{requisito}
    
\begin{requisito}
    \Requisito{RU005}{Exportar los resultados de los algoritmos aplicados en dos archivos, uno para las variables y otro para los objetivos.}
    \Descripcion{Se deben poder exportar las soluciones generadas por la ejecuci�n de los algoritmos sobre un problema a un conjunto de archivos. Espec�ficamente, serian 2 archivos. El primer archivo debe guardar las variables de las soluciones mientras que el segundo archivo los objetivos de las soluciones.}
    
    \Fuente{Jimmy Guti�rrez}
    \Prioridad{Moderada}
    \Estabilidad{Intransable}
    \FechaA{09/09/2019}
    \Estado{Cumple}
    \Incremento{3}
    \Tipo{Funcional}
\end{requisito}
    
\begin{requisito}
    \Requisito{RU006}{Implementar el operador IntegerSBXCrossover.}
    \Descripcion{El operador IntegerSBXCrossover es uno de los operadores de cruzamiento. Este operador en base a c�lculos probabil�sticos combina dos soluciones para crear dos nuevas soluciones hijas.}
    
    \Fuente{Jimmy Guti�rrez}
    \Prioridad{Alta}
    \Estabilidad{Intransable}
    \FechaA{14/10/2019}
    \Estado{Cumple}
    \Incremento{2}
    \Tipo{Funcional}
\end{requisito}
    
\begin{requisito}
    \Requisito{RU007}{Implementar el operador IntegerSinglePointCrossover.}
    \Descripcion{El operador IntegerSinglePointCrossover es un operador de cruzamiento. Viendo la soluci�n como un vector, este operador toma dos soluciones y elige un punto a partir del cual los valores de una soluci�n se intercambiar�n con los valores de la otra soluci�n. Este operador usa una probabilidad de cruzamiento y solamente realiza el intercambio de los valores en la soluci�n cuando un n�mero generado aleatoriamente es menor que la probabilidad de cruzamiento.}
    
    \Fuente{Jimmy Guti�rrez}
    \Prioridad{Alta}
    \Estabilidad{Intransable}
    \FechaA{14/10/2019}
    \Estado{Cumple}
    \Incremento{2}
    \Tipo{Funcional}
\end{requisito}
    
\begin{requisito}
    \Requisito{RU008}{Implementar el operador IntegerPolynomialMutation.}
    \Descripcion{El operador IntegerPolynomialMutation es un operador de mutaci�n. Este operador de mutaci�n usa c�lculos probabil�sticos para mutar algunos variables de decisi�n que forman parte de la soluci�n.}
    
    \Fuente{Jimmy Guti�rrez}
    \Prioridad{Alta}
    \Estabilidad{Intransable}
    \FechaA{14/10/2019}
    \Estado{Cumple}
    \Incremento{2}
    \Tipo{Funcional}
\end{requisito}
    
\begin{requisito}
    \Requisito{RU009}{Implementar el operador IntegerSimpleRandomMutation.}
    
    \Descripcion{El operador IntegerSimpleRandomMutation es un operador de mutaci�n. Este operador muta una variable de decisi�n cuando un n�mero generado aleatoriamente es menor que la probabilidad de mutaci�n establecida. El operador recorre cada variable de decisi�n realizando lo descrito anteriormente. La mutaci�n mencionada por este operador consiste en cambiar el valor de la variable de decisi�n por otro valor aleatorio. }
    
    \Fuente{Jimmy Guti�rrez}
    \Prioridad{Alta}
    \Estabilidad{Intransable}
    \FechaA{14/10/2019}
    \Estado{Cumple}
    \Incremento{2}
    \Tipo{Funcional}
\end{requisito}
    
    
\begin{requisito}
    \Requisito{RU010}{Implementar el operador IntegerRangeRandomMutation.}
    
    \Descripcion{El operador IntegerRangeRandomMutation es un operador de mutaci�n. Este operador muta una variable de decisi�n cuando un n�mero generado aleatoriamente es menor que la probabilidad de mutaci�n establecida. El operador recorre cada variable de decisi�n realizando lo descrito anteriormente. La mutaci�n realizada por este operador consiste en cambiar el valor de la variable de decisi�n por otro valor aleatorio que se encuentre entre un rango establecido.
    \singlespacing
    Ejemplo:\singlespacing

    Variable de decisi�n: 3\\
    Rango: 2\\
    La variable de decisi�n despu�s de aplicado el operador puede tomar un valor entre [1, 5].
    }
    \Fuente{Jimmy Guti�rrez}
    \Prioridad{Alta}
    \Estabilidad{Intransable}
    \FechaA{14/10/2019}
    \Estado{Cumple}
    \Incremento{2}
    \Tipo{Funcional}
\end{requisito}
    
    
\begin{requisito}
    \Requisito{RU011}{Implementar el operador UniformSelection.}
    \Descripcion{El operador UniformSelection~\cite{Iglesias-2004} es un operador de selecci�n. Este operador de selecci�n ordena la poblaci�n y asigna una probabilidad m�xima y m�nima a la mejor y peor soluci�n, respectivamente. A las soluciones que se encuentran entre la mejor y la peor soluci�n se le asigna una probabilidad entre el m�ximo y m�nimo obtenido anteriormente. Si la probabilidad de la soluci�n es mayor a 1.5 entonces la soluci�n se duplica en la nueva poblaci�n. Si la probabilidad esta entre 0.5 y 1.5, entonces en la nueva poblaci�n se agrega la soluci�n solo una vez. Las soluciones cuya probabilidad es menor que 0.5 no aparecen en la nueva poblaci�n.
    
    La probabilidad m�xima puede se calcula como $p_{max} = \beta/N_c$ mientras que la probabilidad m�nima se calcula de acuerdo a $p_{min} = (2-\beta)/N_c$. Donde $\beta$ es un numero entre 1.5 y 2, y $N_c$ es el n�mero de soluciones presentes en el conjunto sobre el que se realizar� la selecci�n.

    La probabilidad de las soluciones intermedias se calcula de acuerdo a $$p_i=p_{min}+(p_{max}-p_{min})\times((N_c-i)/(N_c-1))$$}
    \Fuente{Jimmy Guti�rrez}
    \Prioridad{Alta}
    \Estabilidad{Intransable}
    \FechaA{14/10/2019}
    \Estado{Cumple}
    \Incremento{2}
    \Tipo{Funcional}
\end{requisito}
    
\begin{requisito}
    \Requisito{RU012}{Crear archivo inp de la soluci�n generada.}
    \Descripcion{Al ejecutar un algoritmo metaheur�stico este devuelve una o un conjunto de soluciones. A partir de alguna de estas soluciones se debe crear un archivo inp en el que se vean reflejados los resultados de la soluci�n.}
    
    \Fuente{Jimmy Guti�rrez}
    \Prioridad{Moderada}
    \Estabilidad{Intransable}
    \FechaA{15/10/2019}
    \Estado{Cumple}
    \Incremento{3}
    \Tipo{Funcional}
\end{requisito}
    
\begin{requisito}
    \Requisito{RU013}{Mostrar las soluciones de los algoritmos en la interfaz de usuario.}
    \Descripcion{Mostrar los resultados de la ejecuci�n del algoritmo, es decir, las variables y los valores objetivos resultantes. Adicionalmente, �stas deben poder ser guardadas en el equipo del usuario. Las variables de decisi�n se guardar�n en un archivo (VAR), mientras que los valores de los objetivos se guardar�n en otro (FUN).}
    
    \Fuente{Jimmy Guti�rrez}
    \Prioridad{Alta}
    \Estabilidad{Transable}
    \FechaA{30/11/2019}
    \Estado{Cumple}
    \Incremento{3}
    \Tipo{Funcional}
\end{requisito}
    
\begin{requisito}
    \Requisito{RU014}{Mostrar las caracter�sticas de la red.}
    \Descripcion{Mostrar las caracter�sticas que posee la red. Esto puede ser realizado cuando se presiona el elemento de la red o agregando alg�n componente que muestre los elementos que conforman la red.}
    
    \Fuente{Jimmy Guti�rrez}
    \Prioridad{Baja}
    \Estabilidad{Transable}
    \FechaA{30/11/2019}
    \Estado{Cumple}
    \Incremento{3}
    \Tipo{Funcional}
\end{requisito}
    
    
\begin{requisito}
    \Requisito{RU015}{Graficar las soluciones.}
    \Descripcion{Mostrar en un plano cartesiano las soluciones que se van obteniendo a medida que se ejecuta el algoritmo. Solo considerar hasta 2 objetivos.}
    
    \Fuente{Jimmy Guti�rrez}
    \Prioridad{Baja}
    \Estabilidad{Transable}
    \FechaA{30/11/2019}
    \Estado{Cumple}
    \Incremento{3}
    \Tipo{Funcional}
\end{requisito}
    
\begin{requisito}
    \Requisito{RU016}{Hacer el programa f�cil de extender.}
    \Descripcion{El programa debe poder f�cilmente agregar nuevos algoritmos, operadores y problemas.}
    
    \Fuente{Jimmy Guti�rrez}
    \Prioridad{Alta}
    \Estabilidad{Transable}
    \FechaA{30/11/2019}
    \Estado{Cumple}
    \Incremento{3}
    \Tipo{No Funcional}
\end{requisito}
    
\begin{requisito}
    \Requisito{RU017}{Mostrar en la interfaz de usuario los problemas, agrupando los algoritmos que pueden ser utilizados para resolverlos.}
    \Descripcion{En el men� donde se muestran los problemas que pueden ser resueltos debe aparecer el nombre del problema y en �ste se deben agrupar los algoritmos que pueden ser utilizados para resolverlo.
    \singlespacing
    EJ:
    \singlespacing
    Pumping Schedule \\
    $>$ NSGAII\\
    $>$ SPA2

    }
    \Fuente{Daniel Mora-Meli�}
    \Prioridad{Baja}
    \Estabilidad{Intransable}
    \FechaA{27/01/2020}
    \Estado{Cumple}
    \Incremento{5}
    \Tipo{Funcional}
\end{requisito}
    
\begin{requisito}
    \Requisito{RU018}{Permitir realizar m�ltiples simulaciones independientes para resolver el problema multiobjetivo.}
    \Descripcion{Durante la resoluci�n del problema multiobjetivo se debe poder escoger cuantas veces se correra el algoritmo, independientemente de otra ejecuci�n. La soluci�n final ser� la Frontera de Pareto del conjunto formado por todas las soluciones generadas a partir de cada ejecuci�n independiente del algoritmo. Este concepto se conoce en algunos framework como Experimentos.}
    
    \Fuente{Jimmy Guti�rrez}
    \Prioridad{Alta}
    \Estabilidad{Intransable}
    \FechaA{27/01/2020}
    \Estado{Cumple}
    \Incremento{5}
    \Tipo{Funcional}
\end{requisito}
    
\begin{requisito}
    \Requisito{RU019}{Guardar los resultados temporales por cada simulaci�n independiente del problema multiobjetivo y generar los archivos al final de todas las simulaciones con los mejores resultados obtenidos.}
    \Descripcion{Las simulaciones multiobjetivos permiten realizar m�ltiples simulaciones independientes, entregando cada ejecuci�n como resultado el conjunto de soluciones que conforman su Frontera de Pareto. Sin embargo, una vez se terminan todas las repeticiones y se genera la soluci�n final, se pierden los resultados intermedios de cada repetici�n. Es por ello, que mientras se van terminando cada simulaci�n independiente, se debe guardar un respaldo de los resultados de cada repetici�n del algoritmo.}
    
    \Fuente{Jimmy Guti�rrez}
    \Prioridad{Baja}
    \Estabilidad{Intransable}
    \FechaA{27/01/2020}
    \Estado{Cumple}
    \Incremento{5}
    \Tipo{Funcional}
\end{requisito}
    
\begin{requisito}
    \Requisito{RU020}{Permitir realizar simulaciones hidr�ulicas utilizando los valores por defectos que vienen en el archivo inp y visualizar los resultados.}
    \Descripcion{Utilizando los valores que vienen por defecto en el archivo inp se debe poder llevar a cabo la simulaci�n hidr�ulica de la red. Posteriormente, los resultados podr�n ser visualizados por el usuario.}
    
    \Fuente{Daniel Mora-Meli�}
    \Prioridad{Alta}
    \Estabilidad{Intransable}
    \FechaA{27/01/2020}
    \Estado{Cumple}
    \Incremento{5}
    \Tipo{Funcional}
\end{requisito}
    
\begin{requisito}
    \Requisito{RU021}{Agregar el algoritmo multiobjetivo SMPSO.}
    \Descripcion{Con el fin de comprobar que se pueden acoplar nuevos algoritmos se solicita por parte del usuario que se agregue el algoritmo SMPSO. }
    
    \Fuente{Jimmy Guti�rrez}
    \Prioridad{Baja}
    \Estabilidad{Intransable}
    \FechaA{13/05/2020}
    \Estado{Cumple}
    \Incremento{6}
    \Tipo{Funcional}
\end{requisito}
    
\begin{requisito}
    \Requisito{RU022}{Agregar el algoritmo multiobjetivo SPA2.}
    \Descripcion{Con el fin de comprobar que se pueden acoplar nuevos algoritmos se solicita por parte del usuario que se agregue el algoritmo SPA2. }
    
    \Fuente{Jimmy Guti�rrez}
    \Prioridad{Baja}
    \Estabilidad{Intransable}
    \FechaA{13/05/2020}
    \Estado{Cumple}
    \Incremento{6}
    \Tipo{Funcional}
\end{requisito}
    
\begin{requisito}
    \Requisito{RU023}{Incluir en el dibujo de la red s�mbolos para diferenciar los elementos que componen la red.}
    \Descripcion{Se requiere que se agreguen en la representaci�n gr�fica de la red s�mbolos para distinguir los distintos elementos que la componen.}
    
    \Fuente{Jimmy Guti�rrez}
    \Prioridad{Baja}
    \Estabilidad{Intransable}
    \FechaA{13/05/2020}
    \Estado{Cumple}
    \Incremento{6}
    \Tipo{Funcional}
\end{requisito}
    
\begin{requisito}
    \Requisito{RU024}{Permitir realizar m�ltiples simulaciones independientes para resolver el problema monoobjetivo.}
    \Descripcion{Durante la resoluci�n del problema monoobjetivo se debe poder escoger cuantas veces se repetir� el algoritmo, siendo independiente una repetici�n de la otra. Al finalizar cada repetici�n se solicita mostrar los resultados de cada repetici�n.}
    
    \Fuente{Jimmy Guti�rrez}
    \Prioridad{Alta}
    \Estabilidad{Intransable}
    \FechaA{13/05/2020}
    \Estado{Cumple}
    \Incremento{6}
    \Tipo{Funcional}
\end{requisito}
    
\begin{requisito}
    \Requisito{RU025}{Agregar un men� de configuraci�n.}
    \Descripcion{Agregar un men� de configuraciones donde poder establecer el tama�o de los puntos, si mostrar o no el gr�fico de resultados y limitar el n�mero de resultados retornados por el problema multiobjetivo.}
    
    \Fuente{Jimmy Guti�rrez}
    \Prioridad{Baja}
    \Estabilidad{Intransable}
    \FechaA{13/05/2020}
    \Estado{Cumple}
    \Incremento{6}
    \Tipo{Funcional}
\end{requisito}
    
\begin{requisito}
    \Requisito{RU026}{Usar en el gr�fico, para cada repetici�n del algoritmo en un Experimento, un color distinto.}
    \Descripcion{Para facilitar la distinci�n entre cada repetici�n de un algoritmo en un experimento se solicita variar que se cambie el color mostrado para cada repetici�n.}
    
    \Fuente{Jimmy Guti�rrez}
    \Prioridad{Moderada}
    \Estabilidad{Transable}
    \FechaA{13/05/2020}
    \Estado{Cumple}
    \Incremento{6}
    \Tipo{Funcional}
\end{requisito}
    
\begin{requisito}
    \Requisito{RU027}{Mostrar una leyenda que pueda ser activada y desactivada con los s�mbolos mostrados sobre el dibujo de la red.}
    \Descripcion{Se requiere que en la ventana de representaci�n de la red se muestre una leyenda en la que se presenten los s�mbolos utilizados por la aplicaci�n para representar cada componente de la red.}
    
    \Fuente{Jimmy Guti�rrez}
    \Prioridad{Baja}
    \Estabilidad{Intransable}
    \FechaA{13/05/2020}
    \Estado{Cumple}
    \Incremento{6}
    \Tipo{Funcional}
\end{requisito}
    
    
\begin{requisito}
    \Requisito{RU028}{Mostrar en la ventana de configuraci�n informaci�n sobre el problema, como los objetivos, el nombre del algoritmo a utilizar y el nombre del problema que se quiere resolver.}
    \Descripcion{Al momento de querer realizar una optimizaci�n no se entrega suficiente informaci�n acerca de los objetivos del problema. Es por esto, que se requiere poder agregar alguna descripci�n que pueda ser visualizada cuando se abra la ventana de configuraci�n del experimento.}
    
    \Fuente{Jimmy Guti�rrez}
    \Prioridad{Alta}
    \Estabilidad{Intransable}
    \FechaA{13/05/2020}
    \Estado{Cumple}
    \Incremento{6}
    \Tipo{Funcional}
\end{requisito}
    
\begin{requisito}
    \Requisito{RU029}{A�adir en la ventana de resultados del problema columnas extras que muestren las configuraciones utilizadas con el problema.}
    \Descripcion{Se solicita que se pueda agregar mas informaci�n a la ventana de resultados mostrada cuando se termina una optimizaci�n. Se propone usar un mapa con los valores adicionales que se quieran mostrar, en donde la clave sea el nombre de la columna y el valor sea lo mostrado en la celda.}
    
    \Fuente{Jimmy Guti�rrez}
    \Prioridad{Alta}
    \Estabilidad{Intransable}
    \FechaA{13/05/2020}
    \Estado{Cumple}
    \Incremento{6}
    \Tipo{Funcional}
\end{requisito}
    
    
\begin{requisito}
    \Requisito{RU030}{Exportar los resultados de los algoritmos aplicados como un Excel.}
    \Descripcion{Se requiere exportar la table de resultados de la optimizaci�n a un archivo Excel.}
    
    \Fuente{Jimmy Guti�rrez}
    \Prioridad{Moderada}
    \Estabilidad{Intransable}
    \FechaA{13/05/2020}
    \Estado{Cumple}
    \Incremento{6}
    \Tipo{Funcional}
\end{requisito}
    
\begin{requisito}
    \Requisito{RU031}{Exportar el gr�fico utilizado para mostrar visualmente las soluciones a una imagen (PNG o SVG)}
    \Descripcion{Se requiere exportar el gr�fico de resultados mostrado durante la ejecuci�n de un experimento, ya sea para el problema monoobjetivo como el multiobjetivo. Se requiere idealmente SVG, en caso de que este formato no sea posible se acepta la utilizaci�n de PNG.}
    
    \Fuente{Jimmy Guti�rrez}
    \Prioridad{Baja}
    \Estabilidad{Transable}
    \FechaA{13/05/2020}
    \Estado{Cumple}
    \Incremento{6}
    \Tipo{Funcional}
\end{requisito}
    
\begin{requisito}
    \Requisito{RU032}{Permitir indicar valores por defecto a los operadores y a los problemas.}
    \Descripcion{Se solicita que se puedan ingresar valores por defectos que puedan ser visualizados en la ventana de configuraci�n del problema antes de llevar a cabo la resoluci�n del algoritmo. Los valores por defecto deben ser agregados donde se esperen recibir valores num�ricos.}
    
    \Fuente{Jimmy Guti�rrez}
    \Prioridad{Moderada}
    \Estabilidad{Intransable}
    \FechaA{13/05/2020}
    \Estado{Cumple}
    \Incremento{6}
    \Tipo{Funcional}
\end{requisito}
    
\section{Requisitos de sistema}

En esta secci�n se presentar�n los requisitos de sistema obtenidos a partir de los requisitos de usuarios.

\begin{requisito}
    \Requisito{RS001}{Leer red desde el archivo inp.}
    \Descripcion{Leer un archivo inp desde la aplicaci�n.}
    \Fuente{Jimmy Guti�rrez}
    \Prioridad{Alta}
    \Estabilidad{Intransable}
    \FechaA{09/09/2019}
    \Estado{Cumple}
    \Incremento{1}
    \Tipo{Funcional}
\end{requisito}

\begin{requisito}
    \Requisito{RS002}{Cargar red dentro del programa.}
    \Descripcion{Generar una representaci�n de la red en el programa a partir del archivo le�do, es decir, crear una jerarqu�a de clases para mantener la informaci�n de los componentes y su configuraci�n leidos desde el archivo inp.}
    \Fuente{Jimmy Guti�rrez}
    \Prioridad{Alta}
    \Estabilidad{Intransable}
    \FechaA{09/09/2019}
    \Estado{Cumple}
    \Incremento{1}
    \Tipo{Funcional}
\end{requisito}

\begin{requisito}
    \Requisito{RS003}{Implementar Algoritmo Gen�tico.}
    \Descripcion{Implementar el Algoritmo Gen�tico. La versi�n del Algoritmo Gen�tico a ser implementado consiste en el Algoritmo Gen�tico Generacional. }
    \Fuente{Jimmy Guti�rrez}
    \Prioridad{Alta}
    \Estabilidad{Intransable}
    \FechaA{09/09/2019}
    \Estado{Cumple}
    \Incremento{2}
    \Tipo{Funcional}
\end{requisito}

\begin{requisito}
    \Requisito{RS004}{Implementar el problema multiobjetivo \textit{Pipe Optimizing}.}
    \Descripcion{Implementar la clase que lleva a cabo la evaluaci�n de las soluciones generadas por los algoritmos.}\Fuente{Jimmy Guti�rrez}
    \Prioridad{Alta}
    \Estabilidad{Intransable}
    \FechaA{09/09/2019}
    \Estado{Cumple}
    \Incremento{2}
    \Tipo{Funcional}
\end{requisito}

\begin{requisito}
    \Requisito{RS005}{Utilizar Algoritmo Gen�tico para resolver el problema \textit{Pipe Optimizing} desde el enfoque monoobjetivo.}
    \Descripcion{Utilizando el Algoritmo Gen�tico y la clase que lleva a cabo la evaluaci�n de las soluciones (La clase \textit{Problem}) se debe realizar la evaluaci�n y optimizaci�n de los objetivos.}
    \Fuente{Jimmy Guti�rrez}
    \Prioridad{Alta}
    \Estabilidad{Intransable}
    \FechaA{09/09/2019}
    \Estado{Cumple}
    \Incremento{2}
    \Tipo{Funcional}
\end{requisito}

\begin{requisito}
    \Requisito{RS006}{Evaluar soluciones al problema monoobjetivo (Pipe optimizing) usando la librer�a Epanet.}
    \Descripcion{Se deben evaluar el cumplimiento de las restricciones establecidas para el problema. La restricci�n establecida para el problema \textit{Pipe Optimizing} es cumplir con un nivel de presi�n m�nimo.}\Fuente{Jimmy Guti�rrez}
    \Prioridad{Alta}
    \Estabilidad{Intransable}
    \FechaA{09/09/2019}
    \Estado{Cumple}
    \Incremento{2}
    \Tipo{Funcional}
\end{requisito}

\begin{requisito}
    \Requisito{RS007}{Implementar NSGAII.}
    \Descripcion{Implementar el algoritmo NSGAII. El cual es usado para tratar con problemas multiobjetivo.}
    \Fuente{Jimmy Guti�rrez}
    \Prioridad{Moderada}
    \Estabilidad{Intransable}
    \FechaA{01/10/2019}
    \Estado{Cumple}
    \Incremento{4}
    \Tipo{Funcional}
\end{requisito}

\begin{requisito}	
    \Requisito{RS008}{Implementar el problema multiobjetivo \textit{Pumping Schedule}.}
    \Descripcion{Implementar la clase \textit{Problem} que lleva a cabo la evaluaci�n de las soluciones generadas por los algoritmos.}\Fuente{Jimmy Guti�rrez}
    \Prioridad{Moderada}
    \Estabilidad{Intransable}
    \FechaA{01/10/2019}
    \Estado{Cumple}
    \Incremento{4}
    \Tipo{Funcional}
\end{requisito}

\begin{requisito}
    \Requisito{RS009}{Utilizar el algoritmo NSGAII para resolver el problema \textit{Pumping Schedule} desde el enfoque multiobjetivo.}
    \Descripcion{Utilizando el NSGAII y la clase que lleva a cabo la evaluaci�n de las soluciones (La clase \textit{Problem}) se debe realizar la evaluaci�n y optimizaci�n de los objetivos.}\Fuente{Jimmy Guti�rrez}
    \Prioridad{Moderada}
    \Estabilidad{Intransable}
    \FechaA{01/10/2019}
    \Estado{Cumple}
    \Incremento{4}
    \Tipo{Funcional}
\end{requisito}

\begin{requisito}
    \Requisito{RS010}{Evaluar soluciones al problema multiobjetivo (\textit{Pumping Schedule}) usando la librer�a Epanet.}
    \Descripcion{Se deben evaluar el cumplimiento de las restricciones establecidas para el problema. }\Fuente{Jimmy Guti�rrez}
    \Prioridad{Moderada}
    \Estabilidad{Intransable}
    \FechaA{01/10/2019}
    \Estado{Cumple}
    \Incremento{4}
    \Tipo{Funcional}
\end{requisito}

\begin{requisito}
    \Requisito{RS011}{Visualizar red dentro de un canvas.}
    \Descripcion{Se debe mostrar la representaci�n gr�fica de la red utilizando un Canvas.}
    \Fuente{Jimmy Guti�rrez}
    \Prioridad{Moderada}
    \Estabilidad{Intransable}
    \FechaA{09/09/2019}
    \Estado{Cumple}
    \Incremento{3}
    \Tipo{Funcional}
\end{requisito}



\begin{requisito}
    \Requisito{RS012}{Crear archivo TSV para guardar los valores de los objetivos.}
    \Descripcion{Se debe crear un archivo TSV (archivo de texto separado por tabuladores) con los objetivos de las soluciones obtenidas.}\Fuente{Jimmy Guti�rrez}
    \Prioridad{Baja}
    \Estabilidad{Intransable}
    \FechaA{09/09/2019}
    \Estado{Cumple}
    \Incremento{3}
    \Tipo{Funcional}
\end{requisito}

\begin{requisito}
    \Requisito{RS013}{Crear archivo TSV para guardar los valores de las variables.}
    \Descripcion{Se debe crear un archivo TSV (archivo de texto separado por tabuladores) con las variables de las soluciones obtenidas.}\Fuente{Jimmy Guti�rrez}
    \Prioridad{Baja}
    \Estabilidad{Intransable}
    \FechaA{09/09/2019}
    \Estado{Cumple}
    \Incremento{3}
    \Tipo{Funcional}
\end{requisito}

\begin{requisito}
    \Requisito{RS014}{Implementar el operador IntegerSBXCrossover.}
    \Descripcion{El operador IntegerSBXCrossover es uno de los operadores de cruzamiento. En base a c�lculos probabil�sticos combina dos soluciones para crear unas dos nuevas soluciones hijas.}
    \Fuente{Jimmy Guti�rrez}
    \Prioridad{Alta}
    \Estabilidad{Intransable}
    \FechaA{14/10/2019}
    \Estado{Cumple}
    \Incremento{2}
    \Tipo{Funcional}
\end{requisito}

\begin{requisito}
    \Requisito{RS015}{Implementar el operador IntegerSinglePointCrossover.}
    \Descripcion{El operador IntegerSinglePointCrossover es un operador de cruzamiento. Viendo la soluci�n como un vector, este operador toma dos soluciones y elige un punto a partir del cual los valores de una soluci�n se intercambiar�n con los valores de otra soluci�n. Este operador usa una probabilidad de cruzamiento y solamente realiza el intercambio de los valores en la soluci�n cuando un n�mero generado aleatoriamente es menor que la probabilidad de cruzamiento.}
    \Fuente{Jimmy Guti�rrez}
    \Prioridad{Alta}
    \Estabilidad{Intransable}
    \FechaA{14/10/2019}
    \Estado{Cumple}
    \Incremento{2}
    \Tipo{Funcional}
\end{requisito}

\begin{requisito}
    \Requisito{RS016}{Implementar el operador IntegerPolynomialMutation.}
    \Descripcion{El operador IntegerPolynomialMutation es un operador de mutaci�n. Este operador de mutaci�n usa c�lculos probabil�sticos para mutar algunos variables de decisi�n que forman parte de la soluci�n.}
    \Fuente{Jimmy Guti�rrez}
    \Prioridad{Alta}
    \Estabilidad{Intransable}
    \FechaA{14/10/2019}
    \Estado{Cumple}
    \Incremento{2}
    \Tipo{Funcional}
\end{requisito}

\begin{requisito}
    \Requisito{RS017}{Implementar el operador IntegerSimpleRandomMutation.}

    \Descripcion{El operador IntegerSimpleRandomMutation es un operador de mutaci�n. Este operador muta una variable de decisi�n cuando un n�mero generado aleatoriamente es menor que la probabilidad de mutaci�n establecida. El operador recorre cada variable de decisi�n realizando lo descrito anteriormente. La mutaci�n realizada por este operador consiste en cambiar el valor de la variable de decisi�n por otro valor aleatorio. }
    \Fuente{Jimmy Guti�rrez}
    \Prioridad{Alta}
    \Estabilidad{Intransable}
    \FechaA{14/10/2019}
    \Estado{Cumple}
    \Incremento{2}
    \Tipo{Funcional}
\end{requisito}


\begin{requisito}
    \Requisito{RS018}{Implementar el operador IntegerRangeRandomMutation.}
    \Descripcion{El operador IntegerRangeRandomMutation es un operador de mutaci�n. Este operador muta una variable de decisi�n cuando un n�mero generado aleatoriamente es menor que la probabilidad de mutaci�n establecida. El operador recorre cada variable de decisi�n realizando lo descrito anteriormente. La mutaci�n realizada por este operador consiste en cambiar el valor de la variable de decisi�n por otro valor aleatorio que se encuentre entre un rango establecido.
    \singlespacing
Ejemplo:
    \singlespacing
Variable de decisi�n: 3\\
Rango: 2\\
La variable de decisi�n despu�s de aplicado el operador puede tomar un valor entre [1, 5].
}
    \Fuente{Jimmy Guti�rrez}
    \Prioridad{Alta}
    \Estabilidad{Intransable}
    \FechaA{14/10/2019}
    \Estado{Cumple}
    \Incremento{2}
    \Tipo{Funcional}
\end{requisito}


\begin{requisito}
    \Requisito{RS019}{Implementar el operador UniformSelection.}
    \Descripcion{El operador UniformSelection~\cite{Iglesias-2004} es un operador de selecci�n. Este operador de selecci�n ordena la poblaci�n y asigna una probabilidad m�xima y m�nima a la mejor y peor soluci�n respectivamente. A las soluciones que se encuentran entre la mejor y la peor soluci�n se le asigna una probabilidad entre el m�ximo y m�nimo obtenido anteriormente. Si la probabilidad de la soluci�n es mayor a 1.5 entonces la soluci�n se duplica en la nueva poblaci�n. Si la probabilidad esta entre 0.5 y 1.5, entonces en la nueva poblaci�n se agrega la soluci�n solo una vez. Las soluciones cuya probabilidad es menor que 0.5 no aparecen en la nueva poblaci�n.
    
    La probabilidad m�xima puede se calcula como $p_{max} = \beta/N_c$ mientras que la probabilidad m�nima se calcula de acuerdo a $p_{min} = (2-\beta)/N_c$. Donde $\beta$ es un numero entre 1.5 y 2, y $N_c$ es el n�mero de soluciones presentes en el conjunto sobre el que se realizar� la selecci�n.

    La probabilidad de las soluciones intermedias se calcula de acuerdo a $$p_i=p_{min}+(p_{max}-p_{min})\times((N_c-i)/(N_c-1))$$}
    \Fuente{Jimmy Guti�rrez}
    \Prioridad{Alta}
    \Estabilidad{Intransable}
    \FechaA{14/10/2019}
    \Estado{Cumple}
    \Incremento{2}
    \Tipo{Funcional}
\end{requisito}

\begin{requisito}
    \Requisito{RS020}{Aplicar una soluci�n al objeto que representa la red.}
    \Descripcion{Cuando el algoritmo haya generado soluciones para el problema, se debe poder seleccionar una de ellas y aplicarlas sobre un objeto \textit{Network} el cual podr� ser exportado posteriormente a un archivo inp.
    
    Aplicar la soluci�n sobre un objeto \textit{Network} significa tomar los valores de las variables de decisi�n y modificar las configuraciones de la red (instancia de \textit{Network}) donde corresponda..
    }
    \Fuente{Jimmy Guti�rrez}
    \Prioridad{Baja}
    \Estabilidad{Intransable}
    \FechaA{15/10/2019}
    \Estado{Cumple}
    \Incremento{3}
    \Tipo{Funcional}
\end{requisito}

\begin{requisito}
    \Requisito{RS021}{Exportar el objeto red a un archivo .inp.}
    \Descripcion{Se debe exportar el objeto \textit{Network} a un archivo con extensi�n inp, de acuerdo al formato establecido por Epanet. El archivo inp generado debe poder ser cargado en el programa de simulaci�n Epanet.}
    \Fuente{Jimmy Guti�rrez}
    \Prioridad{Baja}
    \Estabilidad{Intransable}
    \FechaA{15/10/2019}
    \Estado{Cumple}
    \Incremento{3}
    \Tipo{Funcional}
\end{requisito}

\begin{requisito}
    \Requisito{RS022}{Crear la pesta�a que permite visualizar los resultados.}
    \Descripcion{Cuando el experimento haya terminado su ejecuci�n se debe crear un nuevo componente para mostrar los resultados.}
    \Fuente{Jimmy Guti�rrez}
    \Prioridad{Moderada}
    \Estabilidad{Transable}
    \FechaA{30/11/2019}
    \Estado{Cumple}
    \Incremento{3}
    \Tipo{Funcional}
\end{requisito}

\begin{requisito}
    \Requisito{RS023}{Mostrar los resultados de la optimizaci�n.}
    \Descripcion{En la pesta�a de resultados se deben mostrar los resultados de la optimizaci�n.}
    \Fuente{Jimmy Guti�rrez}
    \Prioridad{Moderada}
    \Estabilidad{Transable}
    \FechaA{30/11/2019}
    \Estado{Cumple}
    \Incremento{3}
    \Tipo{Funcional}
\end{requisito}

\begin{requisito}
    \Requisito{RS024}{Implementar un componente que permita mostrar los elementos que componen la red.}
    \Descripcion{El componente debe estar filtrado por el tipo de elemento que compone la red.}
    \Fuente{Jimmy Guti�rrez}
    \Prioridad{Moderada}
    \Estabilidad{Transable}
    \FechaA{30/11/2019}
    \Estado{Cumple}
    \Incremento{3}
    \Tipo{Funcional}
\end{requisito}

\begin{requisito}
    \Requisito{RS025}{Implementar componente que permitan mostrar las caracter�sticas de un elemento seleccionado de la red.}
    \Descripcion{Se debe crear una ventana o un componente en el que se muestres la configuraci�n de los elementos de la red. Este componente debe actualizarse cada vez que se seleccione un nuevo elemento, ya sea desde la lista o desde la representaci�n gr�fica de la red. }
    \Fuente{Jimmy Guti�rrez}
    \Prioridad{Moderada}
    \Estabilidad{Transable}
    \FechaA{30/11/2019}
    \Estado{Cumple}
    \Incremento{3}
    \Tipo{Funcional}
\end{requisito}



\begin{requisito}
    \Requisito{RS026}{Permitir seleccionar en la lista de elementos de la red el componente del que se quieren mostrar sus caracter�sticas.}
    \Descripcion{En la lista de elementos de la red, al hacer doble click, se debe mostrar el componente que muestra las configuraciones que est�n establecidas para el elemento seleccionado.}
    \Fuente{Jimmy Guti�rrez}
    \Prioridad{Moderada}
    \Estabilidad{Transable}
    \FechaA{30/11/2019}
    \Estado{Cumple}
    \Incremento{3}
    \Tipo{Funcional}
\end{requisito}

\begin{requisito}
    \Requisito{RS027}{Permitir seleccionar del dibujo de la red el componente del que se van a mostrar sus caracter�sticas.}
    \Descripcion{Al hacer doble click sobre un elemento, en el canvas que representa la red, se debe abrir el componente para ver la configuraci�n que tiene establecida dicho elemento.  }
    \Fuente{Jimmy Guti�rrez}
    \Prioridad{Moderada}
    \Estabilidad{Transable}
    \FechaA{30/11/2019}
    \Estado{Cumple}
    \Incremento{3}
    \Tipo{Funcional}
\end{requisito}

\begin{requisito}
    \Requisito{RS028}{Implementar un componente que muestre un plano cartesiano.}
    \Descripcion{Este componente gr�fico debe tener un plano cartesiano. El plano solo puede ser ocupado cuando la optimizaci�n es uno o dos objetivos.}
    \Fuente{Jimmy Guti�rrez}
    \Prioridad{Baja}
    \Estabilidad{Transable}
    \FechaA{30/11/2019}
    \Estado{Cumple}
    \Incremento{3}
    \Tipo{Funcional}
\end{requisito}

\begin{requisito}
    \Requisito{RS029}{Mostrar las soluciones del experimento monoobjetivo en el plano cartesiano.}
    \Descripcion{El componente de gr�ficos debe permitir agregar el valor de la soluci�n al plano cartesiano. El plano cartesiano tendr� en el eje y el objetivo, y en el eje x el n�mero de generaciones.}
    \Fuente{Jimmy Guti�rrez}
    \Prioridad{Moderada}
    \Estabilidad{Transable}
    \FechaA{30/11/2019}
    \Estado{Cumple}
    \Incremento{3}
    \Tipo{Funcional}
\end{requisito}

\begin{requisito}
    \Requisito{RS030}{Mostrar las soluciones del experimento multiobjetivo en el plano cartesiano.}
    \Descripcion{El componente de gr�ficos debe permitir agregar el valor de la soluci�n al plano cartesiano. El plano cartesiano usar� en el eje x el objetivo 1, mientras que el objetivo 2 estar� en el eje y.  }
    \Fuente{Jimmy Guti�rrez}
    \Prioridad{Moderada}
    \Estabilidad{Transable}
    \FechaA{30/11/2019}
    \Estado{Cumple}
    \Incremento{3}
    \Tipo{Funcional}
\end{requisito}

\begin{requisito}
    \Requisito{RS031}{Implementar una jerarqu�a de clases que a trav�s de la implementaci�n de interfaces se pueda extender los algoritmos.}
    \Descripcion{Implementar una jerarqu�a de clases que permita agregar nuevos algoritmos o modificar los ya existentes f�cilmente. Entendiendo por f�cilmente, que solo se necesita implementar una interfaz para agregar un nuevo algoritmo o extender alguna clase de un algoritmo ya existente para modificar su comportamiento.}
    \Fuente{Jimmy Guti�rrez}
    \Prioridad{Alta}
    \Estabilidad{Intransable}
    \FechaA{30/11/2019}
    \Estado{Cumple}
    \Incremento{3}
    \Tipo{No Funcional}
\end{requisito}


\begin{requisito}
    \Requisito{RS032}{Implementar una jerarqu�a de clases que a trav�s de la implementaci�n de interfaces se pueda extender los problemas.}
    \Descripcion{Implementar una jerarqu�a de clases que permita agregar nuevos problemas f�cilmente. Entendiendo por f�cilmente, que solo se necesita implementar una interfaz para agregar un nuevo problema o extender alguna clase de un problema ya existente para modificar su comportamiento.}
    \Fuente{Jimmy Guti�rrez}
    \Prioridad{Alta}
    \Estabilidad{Intransable}
    \FechaA{30/11/2019}
    \Estado{Cumple}
    \Incremento{3}
    \Tipo{No Funcional}
\end{requisito}

\begin{requisito}
    \Requisito{RS033}{Implementar una jerarqu�a de clases que a trav�s de la implementaci�n de interfaces se pueda extender los operadores.}
    \Descripcion{Implementar una jerarqu�a de clases que permita agregar nuevos operadores o modificar los ya existentes f�cilmente. Entendiendo por f�cilmente, que solo se necesita implementar una interfaz para agregar un nuevo operador o extender alguna clase de un operador ya existente para modificar su comportamiento.}
    \Fuente{Jimmy Guti�rrez}
    \Prioridad{Alta}
    \Estabilidad{Intransable}
    \FechaA{30/11/2019}
    \Estado{Cumple}
    \Incremento{3}
    \Tipo{No Funcional}
\end{requisito}

\begin{requisito}
    \Requisito{RS034}{Agrupar para cada problema mostrado en el men� de la interfaz de usuario los algoritmos que pueden ser usados.}
    \Descripcion{Dentro del men� de optimizaci�n cada problema ser� un nuevo men�, los cuales indicaran los algoritmos que pueden ser usados.}
    \Fuente{Daniel Mora-Meli�}
    \Prioridad{Baja}
    \Estabilidad{Intransable}
    \FechaA{27/01/2020}
    \Estado{Cumple}
    \Incremento{5}
    \Tipo{Funcional}
\end{requisito}

\begin{requisito}
    \Requisito{RS035}{Implementar mecanismo que permita realizar m�ltiples repeticiones del mismo algoritmo para un problema espec�fico desde el enfoque multiobjetivo.}
    \Descripcion{Durante la resoluci�n del problema multiobjetivo se debe poder escoger cuantas veces se ejecutar� el algoritmo de manera independiente de otras ejecuciones. La soluci�n final ser� la Frontera de Pareto del conjunto formado por todas las soluciones generadas a partir de cada ejecuci�n independiente del algoritmo. Este concepto se conoce en algunos framework como Experimentos.}
    \Fuente{Jimmy Guti�rrez}
    \Prioridad{Alta}
    \Estabilidad{Intransable}
    \FechaA{27/01/2020}
    \Estado{Cumple}
    \Incremento{5}
    \Tipo{Funcional}
\end{requisito}

\begin{requisito}
    \Requisito{RS036}{Mantener los resultados de cada repetici�n de un algoritmo cuando se resuelve un problema multiobjetivo.}
    \Descripcion{Al resolver un problema multiobjetivo se deben mantener los resultados intermedios generados cuando se termina cada repetici�n del algoritmo que conforma el Experimento. }
    \Fuente{Jimmy Guti�rrez}
    \Prioridad{Baja}
    \Estabilidad{Intransable}
    \FechaA{27/01/2020}
    \Estado{Cumple}
    \Incremento{5}
    \Tipo{Funcional}
\end{requisito}

\begin{requisito}
    \Requisito{RS037}{Guardar los resultados almacenados de cada repetici�n del algoritmo multiobjetivo.}
    \Descripcion{Los resultados intermedios generados por cada ejecucion independiente del algoritmo metaheur�stico multiobjetivo, deben ser guardados en una carpeta indicada por el usuario.}
    \Fuente{Jimmy Guti�rrez}
    \Prioridad{Baja}
    \Estabilidad{Intransable}
    \FechaA{27/01/2020}
    \Estado{Cumple}
    \Incremento{5}
    \Tipo{Funcional}
\end{requisito}

\begin{requisito}
    \Requisito{RS038}{A partir de los resultados de cada repetici�n del algoritmo, para el problema multiobjetivo obtener las mejores soluciones.}
    \Descripcion{A partir de los resultados intermedios generados por cada repetici�n del algoritmo multiobjetivo se deben obtener las soluciones no dominadas, es decir, aquellas que conforman la Frontera de Pareto.}
    \Fuente{Jimmy Guti�rrez}
    \Prioridad{Alta}
    \Estabilidad{Intransable}
    \FechaA{27/01/2020}
    \Estado{Cumple}
    \Incremento{5}
    \Tipo{Funcional}
\end{requisito}

\begin{requisito}
    \Requisito{RS039}{Guardar las mejores soluciones obtenidas a partir de cada una de las repeticiones del algoritmo.}
    \Descripcion{Guardar las soluciones no dominadas obtenidas al unir los resultados de cada repetici�n del algoritmo dentro de un Experimento.}
    \Fuente{Jimmy Guti�rrez}
    \Prioridad{Baja}
    \Estabilidad{Intransable}
    \FechaA{27/01/2020}
    \Estado{Cumple}
    \Incremento{5}
    \Tipo{Funcional}
\end{requisito}

\begin{requisito}
    \Requisito{RS040}{Realizar la simulaci�n hidr�ulica utilizando los valores por defecto de la red que vienen en el archivo inp.}
    \Descripcion{Se debe poder realizar la simulaci�n hidr�ulica utilizando �nicamente los valores preconfigurados en el archivo inp.}
    \Fuente{Daniel Mora-Meli�}
    \Prioridad{Alta}
    \Estabilidad{Intransable}
    \FechaA{27/01/2020}
    \Estado{Cumple}
    \Incremento{5}
    \Tipo{Funcional}
\end{requisito}

\begin{requisito}
    \Requisito{RS041}{Obtener los resultados de la simulaci�n hidr�ulica desde el simulador.}
    \Descripcion{A medida que se van realizando la simulaci�n hidr�ulica se deben recuperar los resultados del simulador y almacenarlos en memoria para posteriormente ser presentados al usuario.}
    \Fuente{Daniel Mora-Meli�}
    \Prioridad{Alta}
    \Estabilidad{Intransable}
    \FechaA{27/01/2020}
    \Estado{Cumple}
    \Incremento{5}
    \Tipo{Funcional}
\end{requisito}

\begin{requisito}
    \Requisito{RS042}{Mostrar los resultados de la simulaci�n hidr�ulica en la interfaz de usuario.}
    \Descripcion{Mostrar al usuario los resultados obtenidos al realizar la simulaci�n.}\Fuente{Daniel Mora-Meli�}
    \Prioridad{Alta}
    \Estabilidad{Intransable}
    \FechaA{27/01/2020}
    \Estado{Cumple}
    \Incremento{5}
    \Tipo{Funcional}
\end{requisito}

\begin{requisito}
    \Requisito{RS043}{Implementar el algoritmo SMPSO.}
    \Descripcion{Agregar el algoritmo SMPSO (Multiobjetivo) y utilizarlo para resolver el problema Pumping Scheduling.}
    \Fuente{Jimmy Guti�rrez}
    \Prioridad{Baja}
    \Estabilidad{Intransable}
    \FechaA{13/05/2020}
    \Estado{Cumple}
    \Incremento{6}
    \Tipo{Funcional}
\end{requisito}

\begin{requisito}
    \Requisito{RS044}{Implementar el algoritmo SPA2.}
    \Descripcion{Agregar el algoritmo SPA2 (Multiobjetivo) y utilizarlo para resolver el problema Pumping Scheduling.}\Fuente{Jimmy Guti�rrez}
    \Prioridad{Baja}
    \Estabilidad{Intransable}
    \FechaA{13/05/2020}
    \Estado{Cumple}
    \Incremento{6}
    \Tipo{Funcional}
\end{requisito}

\begin{requisito}
    \Requisito{RS045}{Especificar los s�mbolos a utilizar para cada componente de la red.}
    \Descripcion{Establecer s�mbolos para representar los elementos de la red. Se pueden usar los mismos que el software de simulaci�n hidr�ulica Epanet. }
    \Fuente{Jimmy Guti�rrez}
    \Prioridad{Baja}
    \Estabilidad{Intransable}
    \FechaA{13/05/2020}
    \Estado{Cumple}
    \Incremento{6}
    \Tipo{Funcional}
\end{requisito}

\begin{requisito}
    \Requisito{RS046}{Agregar los s�mbolos de los componentes cuando se muestra la red.}
    \Descripcion{Agregar sobre cada elemento mostrado en la representaci�n de la red el s�mbolo que lo representa y que permite distinguirlo de otros elementos.}
    \Fuente{Jimmy Guti�rrez}
    \Prioridad{Baja}
    \Estabilidad{Intransable}
    \FechaA{13/05/2020}
    \Estado{Cumple}
    \Incremento{6}
    \Tipo{Funcional}
\end{requisito}

\begin{requisito}
    \Requisito{RS047}{Implementar mecanismo que permita realizar m�ltiples repeticiones del mismo algoritmo para un problema espec�fico desde el enfoque monoobjetivo.}
    \Descripcion{Durante la resoluci�n del problema monoobjetivo se debe poder escoger cuantas veces se repetir� el algoritmo, siendo independiente una repetici�n de la otra. Al finalizar cada repetici�n se solicita mostrar los resultados de cada repetici�n. As� como se hace para problemas multiobjetivos.}

    \Fuente{Jimmy Guti�rrez}
    \Prioridad{Alta}
    \Estabilidad{Intransable}
    \FechaA{13/05/2020}
    \Estado{Cumple}
    \Incremento{6}
    \Tipo{Funcional}
\end{requisito}

\begin{requisito}
    \Requisito{RS048}{Crear una interfaz de usuario para mostrar las configuraciones.}
    \Descripcion{Implementar una interfaz de usuario para presentar ciertas opciones que el usuario puede configurar al utilizar la aplicaci�n.}
    \Fuente{Jimmy Guti�rrez}
    \Prioridad{Baja}
    \Estabilidad{Intransable}
    \FechaA{13/05/2020}
    \Estado{Cumple}
    \Incremento{6}
    \Tipo{Funcional}
\end{requisito}

\begin{requisito}
    \Requisito{RS049}{Establecer los valores de la aplicaci�n que se permitir� que el usuario configure.}
    \Descripcion{Establecer cu�les ser�n los valores permitidos para que el usuario configure en la ventana de configuraci�n.}
    \Fuente{Jimmy Guti�rrez}
    \Prioridad{Baja}
    \Estabilidad{Intransable}
    \FechaA{13/05/2020}
    \Estado{Cumple}
    \Incremento{6}
    \Tipo{No Funcional}
\end{requisito}

\begin{requisito}
    \Requisito{RS050}{Mostrar al usuario la ventana de configuraci�n.}
    \Descripcion{Agregar la funcionalidad para abrir la ventana de configuraci�n cuando el usuario la requiera.}
    \Fuente{Jimmy Guti�rrez}
    \Prioridad{Baja}
    \Estabilidad{Intransable}
    \FechaA{13/05/2020}
    \Estado{Cumple}
    \Incremento{6}
    \Tipo{Funcional}
\end{requisito}

\begin{requisito}
    \Requisito{RS051}{Aplicar al sistema las configuraciones establecidas en la ventana de configuraciones.}
    \Descripcion{Hacer efectivas las configuraciones configuradas en la ventana de configuraci�n en el programa.}
    \Fuente{Jimmy Guti�rrez}
    \Prioridad{Baja}
    \Estabilidad{Intransable}
    \FechaA{13/05/2020}
    \Estado{Cumple}
    \Incremento{6}
    \Tipo{Funcional}
\end{requisito}

\begin{requisito}
    \Requisito{RS052}{Escoger una paleta de colores a partir de la cual elegir el color usado para mostrar cada iteraci�n.}
    \Descripcion{Establecer una paleta de colores entre las que poder elegir el color a ser usado por cada iteraci�n dentro de un experimento.}
    \Fuente{Jimmy Guti�rrez}
    \Prioridad{Baja}
    \Estabilidad{Transable}
    \FechaA{13/05/2020}
    \Estado{Cumple}
    \Incremento{6}
    \Tipo{No Funcional}
\end{requisito}

\begin{requisito}
    \Requisito{RS053}{Asignar a cada soluci�n de una misma iteraci�n un color dentro de la paleta de colores.}
    \Descripcion{Implementar la funcionalidad para que cada iteraci�n a ser mostrada en el gr�fico de resultados tome un color diferente.}
    \Fuente{Jimmy Guti�rrez}
    \Prioridad{Baja}
    \Estabilidad{Transable}
    \FechaA{13/05/2020}
    \Estado{Cumple}
    \Incremento{6}
    \Tipo{Funcional}
\end{requisito}

\begin{requisito}
    \Requisito{RS054}{Mostrar los s�mbolos usados en la representaci�n de la red y que significan cada uno de ellos.}
    \Descripcion{Implementar c�digo para que se muestre una leyenda con los s�mbolos de la red existentes y a qu� pertenecen.}
    \Fuente{Jimmy Guti�rrez}
    \Prioridad{Baja}
    \Estabilidad{Intransable}
    \FechaA{13/05/2020}
    \Estado{Cumple}
    \Incremento{6}
    \Tipo{Funcional}
\end{requisito}

\begin{requisito}
    \Requisito{RS055}{Agregar una opci�n que permita activar y desactivar la leyenda.}
    \Descripcion{Agregar en el men� de configuraciones la posibilidad de activar y desactivar la leyenda de s�mbolos.}
    \Fuente{Jimmy Guti�rrez}
    \Prioridad{Baja}
    \Estabilidad{Intransable}
    \FechaA{13/05/2020}
    \Estado{Cumple}
    \Incremento{6}
    \Tipo{Funcional}
\end{requisito}

\begin{requisito}
    \Requisito{RS056}{Implementar un mecanismo que permite indicar una descripci�n del problema a resolver.}
    \Descripcion{Implementar alguna manera de poder agregar una descripci�n al problema a tratar que debe ser mostrada en la interfaz de configuraci�n del problema como una pesta�a. Pueden ser usadas las mismas anotaciones para este fin.}
    \Fuente{Jimmy Guti�rrez}
    \Prioridad{Alta}
    \Estabilidad{Intransable}
    \FechaA{13/05/2020}
    \Estado{Cumple}
    \Incremento{6}
    \Tipo{Funcional}
\end{requisito}

\begin{requisito}
    \Requisito{RS057}{Mostrar en la ventana de configuraci�n informaci�n del problema a resolver.}
    \Descripcion{Modificar la ventana de configuraci�n del problema para que se muestre la descripci�n del problema escogido.}
    \Fuente{Jimmy Guti�rrez}
    \Prioridad{Alta}
    \Estabilidad{Intransable}
    \FechaA{13/05/2020}
    \Estado{Cumple}
    \Incremento{6}
    \Tipo{Funcional}
\end{requisito}

\begin{requisito}
    \Requisito{RS058}{Implementar un mecanismo para indicar el valor de las columnas adicionales que quieren ser mostrados en la interfaz de resultados.}
    \Descripcion{Implementar c�digo para a�adir informaci�n adicional a la ventana de resultados. Esta informaci�n adicional ser� especificada por el usuario que implemente el problema. A�adir esta informaci�n debe ser opcional. Si esta informaci�n no se a�ade entonces mostrar los resultados y objetivos.}
    \Fuente{Jimmy Guti�rrez}
    \Prioridad{Alta}
    \Estabilidad{Intransable}
    \FechaA{13/05/2020}
    \Estado{Cumple}
    \Incremento{6}
    \Tipo{Funcional}
\end{requisito}

\begin{requisito}
    \Requisito{RS059}{Agregar los datos adicionales que desean ser mostrados en la ventana de resultados.}
    \Descripcion{Mostrar los par�metros adicionales en la ventana de resultados.}
    \Fuente{Jimmy Guti�rrez}
    \Prioridad{Alta}
    \Estabilidad{Intransable}
    \FechaA{13/05/2020}
    \Estado{Cumple}
    \Incremento{6}
    \Tipo{Funcional}
\end{requisito}

\begin{requisito}
    \Requisito{RS060}{Exportar todos los datos de la ventana de resultados a un archivo Excel.}
    \Descripcion{Implementar c�digo para exportar los valores presentes en la ventana de resultados a un archivo Excel.}
    \Fuente{Jimmy Guti�rrez}
    \Prioridad{Moderada}
    \Estabilidad{Intransable}
    \FechaA{13/05/2020}
    \Estado{Cumple}
    \Incremento{6}
    \Tipo{Funcional}
\end{requisito}

\begin{requisito}
    \Requisito{RS061}{Configurar la carpeta en la que se guardar�n la imagen del gr�fico.}
    \Descripcion{Mostrar una ventana para que el usuario seleccione donde quiere guardar la imagen del gr�fico.}
    \Fuente{Jimmy Guti�rrez}
    \Prioridad{Baja}
    \Estabilidad{Intransable}
    \FechaA{13/05/2020}
    \Estado{Cumple}
    \Incremento{6}
    \Tipo{Funcional}
\end{requisito}

\begin{requisito}
    \Requisito{RS062}{Guardar el gr�fico de resultados en el equipo del usuario en formato PNG.}
    \Descripcion{Guardar el gr�fico en la carpeta seleccionada por el usuario.}
    \Fuente{Jimmy Guti�rrez}
    \Prioridad{Baja}
    \Estabilidad{Intransable}
    \FechaA{13/05/2020}
    \Estado{Cumple}
    \Incremento{6}
    \Tipo{Funcional}
\end{requisito}

\begin{requisito}
    \Requisito{RS063}{Agregar un mecanismo para ingresar valores por defecto para los operadores.}
    \Descripcion{Implementar funcionalidad para agregar valores por defecto a los par�metros que pueden ser configurados desde la ventana de configuraci�n del problema. Estos valores por defecto deben ser para los par�metros num�ricos (int o double).}
    \Fuente{Jimmy Guti�rrez}
    \Prioridad{Moderada}
    \Estabilidad{Intransable}
    \FechaA{13/05/2020}
    \Estado{Cumple}
    \Incremento{6}
    \Tipo{Funcional}
\end{requisito}

\begin{requisito}
    \Requisito{RS064}{Mostrar en la ventana de configuraci�n de los problemas los valores por defecto.}
    \Descripcion{Al abrir la ventana de configuraci�n de problemas mostrar los valores establecidos por defecto.}
    \Fuente{Jimmy Guti�rrez}
    \Prioridad{Moderada}
    \Estabilidad{Intransable}
    \FechaA{13/05/2020}
    \Estado{Cumple}
    \Incremento{6}
    \Tipo{Funcional}
\end{requisito}

\section{Matriz de Trazado Requisitos de Usuario vs. Requisitos de Software}
La matriz de trazabilidad de los requisitos de usuario y de sistema que se presenta a continuaci�n permite ver la relaci�n y dependencia que un requisito de sistema tiene con los requisitos de usuario.

\begin{figure}[H]
    \centering
    \adjincludegraphics[width=\textwidth, trim={0 0 0 {0.3\height}},clip, rotate = 180]{Capitulo3/assets/matriz_req-u_req-s.eps}
\end{figure}

\begin{figure}[H]
    \centering
    \adjincludegraphics[width=\textwidth, trim={0 {0.7\height} 0 0},clip, rotate = 180]{Capitulo3/assets/matriz_req-u_req-s.eps}
      \caption{Matriz de requisito de usuario versus requisitos de sistema.}
      \label{fig:matriz_req}
\end{figure}

%% ambiente glosario
%\begin{glosario}
%  \item[RDA] Este es el significado del primer t�rmino, realmente no se bien lo que significa pero podr�a haberlo averiguado si hubiese tenido un poco mas de tiempo.
%  \item[GA] Este si se lo que significa pero me da lata escribirlo...
%\end{glosario}


%% genera las referencias
%\bibliography{refs}


%% comienzo de la parte de anexos
%\appendixpart

%% contenido del primer anexo
%%\appendix{Apendix}

\end{document}

   

